\documentclass{article}

\usepackage{listings}
\usepackage{xcolor}
\lstset { %
    language=C++,
    backgroundcolor=\color{black!5}, % set backgroundcolor
    basicstyle=\footnotesize,% basic font setting
}

\usepackage{amsfonts}
\usepackage{amsmath}
\usepackage{graphicx}
\usepackage{hyperref}
\usepackage{caption}
\usepackage{fancyhdr}
\usepackage{geometry}
	\geometry{height=24 cm}
	\geometry{left=2.5 cm}
	\geometry{right=2.5 cm}
	\geometry{top=2 cm}
	\geometry{headheight=1 cm}

\setcounter{secnumdepth}{2}

\title{\vspace{2cm}\textbf{Domande Reti di Calcolatori}}
\author{\vspace{3mm}4 ottobre 2022}
\date{\vspace{3mm} \textbf{Rosso Carlo}}

\begin{document}
\tableofcontents
\newpage

\section{Serie di Fourier}
Un segnale che ha una durata finita può essere immaginato semplicemente come la
ripetizione infinita dell'intero schema, ossia come una rappresentazione dove
l'intervallo da $T$ a $2T$ è uguale all'intervallo da $0$ a $T$. La serie di
Fourier è infatti, in matematica, la rappresentazione di una funzione periodica
tramite la combinazione lineare di funzioni sinusoidali (seno e coseno). \\ 
È possibile quindi rappresentare un segnale tramite delle funzioni di questo
tipo, le quali permettono una analisi e modellazione molto efficacie. \\
Su questo principio si basano tutte le reti, ed in generale il passaggio di dati
tramite mezzi di trasmissione $>$ purtroppo, nella pratica, questi ultimi
attenuano in modo non uniforme i componenti della serie di Fourier, generando
così una distorsione del segnale. \\
Per ovviare a questa distorsione, le ampiezze fino ad una certa frequenza
vengono trasmesse senza modifiche, da quella frequenza in poi vengono attenuate
$>$ l'intervallo di frequenze trasmesse senza una forte attenuazione è chiamato
banda passante (generalmente viene indicata come banda passante quella compresa
tra 0 e la frequenza dove la potenza è attenuata del 50\%).

\paragraph{Pregi} scomporre un segnale in più componenti permette uno studio più
preciso del segnale.

\paragraph{Difetti} none

\paragraph{Ambiti d'uso} viene ampiamente usata nelle comunicazioni in generale,
per la trasmissione dei dati $>$ la scomposizione del segnale in più sinusoidi
migliora la comprensione delle onde. Si passerà poi alle varie modulazioni per
ovviare ai problemi dovuti alle attenuazioni o alle distorsioni.

\section{Bit rate o baud rate}

Il \textit{bitrate} è una quantità di informazioni digitali che viene trasferita
o registrata in una certa unità di tempo. \\
Si tratta quindi della velocità di trasmissione, espressa in \textit{bit/s}. \\
La velocità di trasmissione è anche detta banda e dipende dal tipo di mezzo
trasmissivo utilizzato e dalle sue condizioni fisiche al momento dell'uso. \\
Il baudrate invece rappresenta il numero di "simboli" che viene trasmesso in un
secondo, ossia un determinato e fisso numero di bit (che differisce in base alle
techiche di modulazione utilizzate). Non va confusa con il sopracitato bitrate,
in qunato misurano unità differenti ($bitrate = baudrate \cdot n$ dove $n$ è la
cardinalità dell'alfebo utilizzato).

\paragraph{Pregi} grazie a queste unità di misura è possibile dare una
rappresentazione quantitativa della velocità di trasmissione del mezzo.

\paragraph{Difetti} none

\paragraph{Ambiti d'uso} queste metriche vengono utilizzate nelle reti wireless
e cablate.

\section{Satelliti}

Esistono 3 tipi principali di satelliti: i LEO, i MEO, i GEO, rispettivamente
\textit{lwo earth orbit}, \textit{medium earth orbit} e \textit{geostationary
earth orbit}. \\
In generale, i satelliti comunicano tra loro ad alte frequenze, questo per
diminuire la dispersione ed aumentare la banda, possono infatti essere pensati
come dei grandi ripetitori di microonde posti nel cielo. Ciò può essere fonte di
complicanze nel caso vi siano condizioni meteorologiche avverse, che potrebbero
creare interferenze.

\begin{itemize}

\item \textbf{LEO} sono posti inferiormente alla fascia di Van Allen inferiore,
di conseguenza sono i più bassi dei 3 tipi. Vengono usati principalmente per le
telecomunicazioni, hanno una bassa latenza e necessitano di meno potenza per
trasmettere rispetto agli altri tipi (ciò vale da entrambi i lati della
comunicazione, cioè anche per gli apparecchi terrestri che trasmettono al
satellite). Il costo per mandarli in orbita, inoltre, è più basso rispetto ai MEO
e GEO. Alcuni satelliti LEO sono Iridium (gruppo di 66 satelliti che ruotano
attorno al globo e forniscono servizi di fax, telefonia vocale exx ovunque nel
mondo mare e in volo compresi, creando in questo modo una rete di
telecomunicazioni nello spazio e trasmettono il segnale ricevuto), Globalstar
(formato da molti meno satelliti rispetto a Iridium, i quali non creano
propriamente una rete nello spazio, in quanto fungono solo da ripetitori del
segnale che viene trasmesso in realtà dalla rete terrestre) ma anche la più
recente Starlink (progetto fondato dalla agenzia spaziale SpaceX con lo scopo di
portare la connessione internet satellitare a bassa latenza ovunque nel mondo e
con costi contenuti rispetto alle alternative già esistenti). \\
In generale, i satelliti LEO sono molto utilizzati in ambito militare, per le
telecomunicazioni ma anche per i sistemi di telerilevamento di sensori. 

\item \textbf{MEO} compresi tra la fascia di Van Allen inferiore e la fascia di
Van Allen superiore troviamo i satelliti \textit{medium earth orbit}. Si tratta
infatti di satelliti in orbita media, ciò comporta che il costo per lanciarli in
orbita si ainferiore rispetto ai satelliti geostazionari, ma allo stesso tempo
non ne siano necessari così tanti come i satelliti LEO. \\
Necessitano tuttavia di più potenza di questi ultimi per trasmettere, siccome
sono più distanti, olter ad avere potenzialmente più rischi di interferenze.
Inoltre, rispetto ai satelliti GEO, si perde la comodità del punto fisso)
rispetto all'equatore terrestre). In questa fascia troviamo Sputnik, il primo
satellite lanciato in orbita della storia e soprattutto i satelliti che si
occupano del sistema GPS. Quest'ultimo era all'inizio una esclusiva militare,
tuttavia si è poi deciso di aprirne l'utilizzo al pubblico )in principio con una
precisione di 100 metri, poi ridotta a 20 metri). In generale, a causa del fatto
che prima di triangolare la posizione sarebbe necessario per i dispositivi sulla
terra attendere il passaggio e la trasmissione del segnale da parte di 3 o 4
satelliti, il "tempo di fix" per ottenere la posizione dovrebbe essere molto
lungo, anche di qualche minuto. Questo problema è tuttavia stato risolto dal
sistema A-GPS $>$ si tratta di un potente computer centrale che viene utilizzato
dagli ISP / operatori di servizi mobile, il quale ha la funzione di calcolare
costantemente la posizione dei satelliti del GPS, per poi fornirla direttamente
ai telefoni cellulari che richiedono una calibrazione della posizione. \\

\item \textbf{GEO} i satelliti \textit{geostationary earth orbit} sono collocati
al di sopra della fasci di Van Allen Superiore, si tratta di satelliti definiti
come "geostazionari", in quanto la loro posizione risulta fissa rispetto alla
linea equatoriale terrestre. Vengono collocati infatti a distanza di 2 gradi nel
pinao equatoriale fra loro, ne deriva che c'è spazio solo per 180 satelliti di
questo tipo $>$ ciò comporta che il loro lancio sia spesso fonte di collisioni e
dispute fra i diversi paesi. \\
In generale, si tratta di satelliti di grossa dimensione, la cui funzione
principale è la trasmissione satellitare di contenuti televisivi, oltre che
essere molto utilizzati in ambito meteorologico. I lanci dei satelliti GEO sono
molto costosi, tuttavia il loro vantaggio principale è la stabilità (utile per
questioni sia di osservazione ma anche di orientamento delle antenne). Possiamo
contare sui satelliti geostazionari della seconda metà del XX secolo, il primo
esemplare "commerciale" è stato Intelsat I, un satellite realizzato da Comsat.
\end{itemize}

\paragraph{Pregi} rispetto alla fibra, i satelliti danno una maggiore garanzia
di copertura e permettono di connettere/raggiungere con la stessa velocità
l'intera superficie del globo. Per questo motivo i satelliti vengono infatti
molto utilizzati in luoghi inospitali o difficilmente raggiungibili da
connessioni cablate.

\paragraph{Difetti} costi maggiori per l'installazione di un sistema satellitare
rispettoa all'installazione della fibra ottica, economa di mantenimento molto
esosa, soprattutto per quanto riguarda i satelliti in orbita più alta
(attualmente infatti per le telecomunicazioni si è orientati ad utilizzare la
fibra piuttosto che la connessione satellitare).

\paragraph{Ambiti d'uso} sono utilizzati nell'ambito delle telecomunicazioni
(vocali  e via fax), i MEO per la gestione del sistema GPS, i GEO in ambito
meteorologico ed in generale tutti i sateliti in ambito militare.

\section{Modem e modulazione}

Per inviare segnali digitali attraverso una linea telefonica il computer deve
convertire i dati in forma analogica per poterle trasmetterere attraverso
l'ultimo miglio. Questa conversione avveien attraverso un dispositivo chimato
modem.\\
I problmei pincipali delle linee di trasmissione però:
\begin{enumerate}
	\item la perdita di energia per via della propagazione del segnale verso
		l'esterno, dipendente dalla frequenza del segnale. Questa perdita è
		anche detta \textbf{attenuazione};

	\item distorsione, causata dal fatto che ogni componente di Fourier si
		propaga a velocità diverse rispetto al cavo;

	\item rumore, ovvero l'energia indesiderata generate da sorgenti esterne al
		trasmittore. Per minimizzare questi problemi viene utilizzata la
		trasmissione AC che introduce un tono continuo, chiamato portale d'onda
		sinusoidale, la cui ampiezza, frequenza o fase possono essere modulate
		per trasmettere informazioni. Un apparecchio che accetta un flusso
		seriale di bit in ingresso e produce una portata modulata è appunto il
		modem. 
\end{enumerate}

\paragraph{Modulazione FSK} si modula la frequenza in maniera proporzionale
all'ampiezza che si vuole trasmettere, ovvero cambia la frequenza in base al
simbolo che si vuole trasmettere.

\paragraph{Modulazione AM} modulazione in ampiezza in proporzione all'ampiezza
da trasmettere, viene usata principalmente nelle trasmissioni radio. 

\paragraph{Modulazione PSK} modulazione di fase, cambio fase del segnale a
seconda del simbolo da trasmettere.

\section{FDM, TDM, CDM (algoritmi di multiplexing e selezione banda)}

Ne esistono 3 tipi diversi:
\begin{itemize}
	\item FDM (frequency division multiplexing), divide lo spettro in varie
		bande di frequenza e ad ogni utente viene assegnata una porzione della
		banda con uso esclusivo. Vengono usati 12 canali voce uniti in
		multiplexing nella banda tra 60 e 108 MHz. Viene usato dal GSM.

	\item TDM (time division multiplexing), l'intera banda viene assegnata a
		tutti gli utenti a turno per un tempo limitato secondo Round Robin.
		Usato da Bluetooth, GPRS, GSM.

	\item CDM (code division multiplexing), anche conosciuto come CDMA, è il
		protocollo di accesso multiplo a canale condiviso. È realizzato
		moltiplicando in trasmissione l'informazione generate per un'opportuna
		parola detta "chip", questa sequenza in uscita sarà successivamente
		modulata e trasmessa sul canale. In ricezione il segnale sarà costituito
		dalla somma vettoriale di tutti i segnali trasmessi dalle singole
		stazioni.
\end{itemize}

\section{Modulazione di frequenza}

Durante l'invio di informazioni, il segnale può subilre attenuazione,
distorsione o vnire in generale disturbato dal rumore $>$ per questo motivo si
tende ad evitare l'uso di un largo intervallo di frequenze. Questi effetti
rendono la trasmissione in banda base (DC) una soluzione adatta solo a velocità
basse e distanze brevi. Piuttosto, per risolvere questi problemi viene
utilizzata la trasmissione AC, ossia un tono continuo (portata d'onda
sinusoidale) nell'intervallo compreso tra 100hZ e 200hZ, lavorando sulla
modulazione della sua ampiezza (modulazione AM), frequenza (modulazione FM) o
fase.\\
La modulazione di frequenza (FM) non è altro quindi che una tecnica di
trasmissione utilizzata per trasmettere informazioni usando la variazione di
frequenza dell'onda portante. Rispetto all modulazione in ampiezza (AM) ha il
vantaggio di essere molto meno sensibile ai disturbi e permette una trasmissione
di miglior qualità. Ha inoltre un'efficienza energetica molto maggiore dato che
la potenza del segnale modulato FM è esclusivamente quello della portante.

\paragraph{pregi} permette di ridurre i problemi di attenuazione e distorsione
della linea.

\paragraph{difetti} necessita di circuiti complessi sia per la generazione del
segnale sia per la sua ricezione. Questi problemi sono stati, per la maggior
parte, superati dalle attuali tecnologie. 

\paragraph{ambiti d'uso} la modulazione FM viene utilizzata soprattuto in ambito
di broadcasting commerciale ed è in generale molto più comune della modulazione
in ampiezza. 

\section{Modulazione Delta}

La modulazione delta è un tipo di compressione e digitalizzazione dei dati di un
segnale analogico. Si tratta di una variazione del differntial pulse coe
modulation (DPCM) ed è stato introdotto con la generazione 2G dei telefoni
mobili. A differenza del segnale analogico, il senglae digitale ricavato dalla
modulazione delta utilizza molta meno banda, ma perde una discreta quantità di
informazioni essendo un tipo di compressione lossy.\\
A differenza di altre tecniche, piuttosto che quantizzare il valore della forma
d'onda analogica in ingresso, la modulazione delta quantizza la differenza tra
la fase corrente e quella precedente. Nella sua forma più semplice, la
variazione dell'onda per ogni unità di tempo può essere descritta da un singolo
bit 1 od un singolo bit 0, rispettivamente se il segnale in ingresso è positivo
(aumento nel grafico che descrive l'onda) o negativo (diminuizione nel grafico
che descrive l'onda).\\
Grazie a questa tecnica, è possibile comprimere e digitalizzare il segnale
analogico, con il risultato di gravare meno sulle linee e permettere
comunicazioni più veloci, a discapito della precisione. Per questo motivo si
possono presentare problemi in ambiti nei quali il segnale cambia troppo
velocemente (è infatti l'utilizzo della modulazione delta per comprimere e
digitalizzare la musica), in quanto si perderebbero molti dati. Se il
cambiamento del segnale analogico è troppo ampio, si ha quindi una perdita di
dati, ma in ambito telefonico (più precisamente nelle chiamate), questo non è un
problema.

\paragraph{ambiti d'uso} comunicazioni satellitari o nelle comunicazioni
voce/mobile.

\section{QAM, QAM16, QAM64 e QPSK}

QAM è una modalità di modulazione in fase ed in ampiezza (sia digitale che
analogica) per questo motivo si può rappresentare in "costellazioni". Le
portanti sono sinusoidi. Ogni pallina equivale ad uno spostamento sull'asse
temporale dell'onda che sta trasmettendo $>$ questo consente di aumentare il
bitrate, tuttavia, quendo questi spostamenti sono troppo piccoli è possbile che
a causa di interferenze vengano scambiati i simboli che si stanno trasmettendo.
Proprio per questo motivo, QPSK è il tipo di modulazione più robusta (pur
essendo solo di fase) $>$ più si aumentano i puntini, più aumenta il data rate,
ma anche la fragilità della connesione.\\
QAM è quindi un sistema di modulazione numerica di ampiezza in quadratura, sia
digitale che analogica. Il termine quadratura indica che differiscono di 90
gradi.\\
Il segnale in ingresso viene suddiviso e modulato per l'ampiezza. Nel caso di
segnali digitali, si sommano i segnali modulati e si ottiene una forma d'onda
che risulta una combinazione della modulazione di fase e qualla di ampiezza.
Ciascun tipo di modulazione QAM è caratterizzato da un diagramma (costellazione)
su vui sono rappresentati tutti gli stati della portante.\\
La QAM, rispetto alla PSK (phase shift keying), migliora l'immunità di rumore.\\
QAM16, QAM64, ecc invece non sono latro che un tipo di costellazione QAM che
utilizza rispettivamente 4 ampiezza e 4 fasi (QAM16) oppure 16 ampiezza e 16
fasi (QAM64), per un totale di 16 o 61 punti diversi nella costellazione.\\
In realtà, il miglior spostamento sarebbe quello di tipo circolare, tuttavia i
circular QAM, pur essendo ottimali dal punto di vista della robustezza, non
vengono utilizzati a causa della difficoltà nella generazione e decodifica dei
messaggi.\\
In generale, nelle modulazioni QAM non viene utilizzata la modulazione in
frequenza in quanto è così più facile generare il segnale e facilita la
sincronizzazione dello stesso.

\paragraph{pregi} permette di aumentare il data rate.

\paragraph{difetti} ogni modem ha un suo schema di costellazione e può
comunicare solo con altri modem che possiedono lo stesso, anche se la maggior
parte dei modem più recenti è comunque in grado di emulare costellazioni più
lente della propria.

\paragraph{ambiti d'uso} fa parte dello stato fisico, questo tipo di modulazione
è utilizzato come standard nei modem telefonici.

\section{Codifica Manchester}

Si tratta di una tecnica per trasmettere codici binari e determinare gli 1 e gli
0, è utilizzata nel protocollo Ethernet a causa della sua capacità di
auto-sincronizzazione (la codifica ethernet non necessita di un segnale esterno
di sicronia).\\
Inizialmente la codifica Manchester era stata giudicata da molti una scelta
sbagliata e veniva reputata carente rispetto ad altre tecniche di trasmissione
di dati binari, questo a causa del fatto che per utilizzarla era richiesto il
doppio della banda rispetto alla pura conversione binaria 0-5 volt. Tuttavia,
grazie anche all'avanzamento tecnologico ed all'aumentare della velocità media
di trasmissione dei dati, si è rivelata nel corso degli anni una scelta
corretta, essendo la codifica Manchester una tecnica molto precisa, oltre che ad
essere vantaggiosa dal punto di vista economico (l'hw non è molto costoso).\\
Nella codifica Manchester, i bit vengono indentificati da 2 picchi invece che 1
(ed è questo il motivo per cui 50\% della banda va persa), infatti in questo
tipo di codifica è la direzione del cambio di energia che identifica gli 1 e gli
0, e non il picco in sè.

\section{Tecniche di stuffing}

Lo scopo dello strato fisico è quello di prendere un flusso di bit e cercare di
portarlo il più integralmente possibile agli strati superiori $>$ di norma,
tuttavia, non esiste alcuna garanzia riguardo la correttezza dei dati. Il modo
migliore per farlo è quello di suddividere il flusso di bit in dei frame, per
poterli così controllare in maniera migliore.\\
Uno dei metodi più utilizzati è il byte stuffing, che consiste nell'utilizzare
un particolare byte conosciuto come flag byte (FLAG) che delimiterà l'inizio e
la fine di un frame $>$ in questo modo, se il destinatario dovesse perdere la
sincronizzazione, può semplicemente ricercare il flag byte per capire dov'è
l'inizio e la fine del frame corrente.\\
C'è tuttavia un problema: nel caso si volesse trasmettere informazioni/bit che
coincidono con i bit scelti come flag byte, il destinatario potrebbe cadere in
confusione. Nel byte stuffing, la soluzione che è stata addottata è quella di
utilizzare un ulteriore byte conosciuto come escape byte (ESC), da aggiungere ad
ogni occorrenza reale dei bit corrispondenti al FLAG (intesi quindi come bit da
trasmettere e non come fine/inizio di un frame). Questa procedura di aggiunta di
byte aggiuntivi per determinare le caratteristiche dei frame, in generale, è
conosciuta come stuffing.

\paragraph{pregi} risolve il problema della sincronizzazione dei frame,
permettendo una suddivisione dei bit in frame.

\paragraph{difetti} [ legato all-uso di caratteri da 8 bit (1 frame), che non è
supportata da tutte le codifiche. Inoltre, il numero di bit superflui (ESC byte,
FLAG byte, ecc) è notevole.\\
La soluzione che è stata adottata per questi difetti è l'utilizzo di una
procedura di stuffing diversa, conosciuta come bit stuffing $>$ si tratta di una
tecnica che permette di creare data frame con un numero arbitrario di frame,
oltre che permettere l'utilizzo di un numero arbitrario di bit negli stessi (e
quindi non solo 8 come nel byte stuffing).\\
Il bit stuffing è molto più efficace: inserisce infatti solo un bit ogni volta
che trova una sequenza dei primi n-1 bit uguale al FLAG di n bit. Siccome questo
bit viene inserito sempre a parte nei FLAG, ricostruire il messaggio originale
risulta molto semplice, basterà infatti togliere il bit superfluo ogni volta che
si incontra la seguenza di n-1 bit.

\paragraph{ambiti d'uso} il bit stuffing viene utilizzato nel protocollo HDLC,
il byte stuffing nel protocollo Ethernet (per migliorare la sincronizzazione ad
elevatissime velocità).

\section{ALOHA}

In telecomunicazioni ALOHA è un protocollo di rete atto a garantire le
funzionalità di accesso multiplo al mezzo di trasmissione dati condiviso tra più
utenti $>$ si tratat quindi di un protocollo multiaccesso, ciò significa che
indipendentemente dal numero di utenti connessi alla rete, la banda sfruttata
rimane stabile e non si avvicina allo 0 all'aumentare di essi.\\
ALOHA puro $>$ nella sua variante base, in ALOHA ogni utente spedisce i propri
pacchetti senza sapere se il canale trasmissivo è occupato (manca il carrier
sense) e quindi senza sincronizzazione con gli altri utenti della rete. Se il
pacchetto dovesse collidere con altri, viene semplicemente fatto attendere un
tempo casuale (opportunamente limitato) per poi riprovare l'invio. Con ALOHA
puro, la banda massima sfruttata del canale trasmissivo è intorno al 18\%.

\paragraph{pregi} permette l'accesso multiplo di più utenti allo stesso mezzo
trasmissivo e favorisce le comunicazioni broadcast.

\paragraph{svantaggi} è molto poco efficiente in telecomunicazioni con tante
stazioni.

\paragraph{ambiti d'uso} protocollo utilizzato a livello di indirizzi MAC, viene
usato dalle varie stazioni nelle comunicazioni broeadcast per condividere lo
stesso mezzo trasmissivo tra più utenti.

\subsection{ALOHA slotted}
Aggiungendo la temporizzazione si ottiene ALOHA slotted, in questo protocollo,
che è un migliorameno ed aggiornamento dell'ALOHA base. Esistono degli slot di
tempo che determinano l'accesso al mezzo trasmissivo, infatti, gli utenti non
possono trasmettere a cavallo di questi. Con ALOHA slotted, la banda massima
sfruttata dael canale trasmissivo raddoppia (circa 36\%).

\paragraph{pregi} miglioramento di ALOHA puro, raddoppia il grado di successo.

\paragraph{svantaggi} è richiesto un segnale di sicronia esterno che gestisce
gli slot di tempo ed indichi alle stazioni quando trasmettere.

\section{CSMA}

Il CSMA è una tecnica di trasmissione dati per la quale ogni dispositivo perima
di avviare la trasmissione dei dati deve verificare se latri nodi stanno
trasmettendo sullo stesso canale, rilevando quindi la presenza di portanti. Solo
se il canale risulta livero, il dispositivo inizia la trasmissione, altrimenti è
tenuto ad attendere un tempo arbitrario e diverso in base allo standard CSMA
utilizzato prima di riprovare.

\paragraph{CSMA 1-persistente} primo tra i protocolli CSMA, ha la particolarità
di inviare con probabilità 1 sul canale in caso di nessun rilevamento. La
possibilità di inviare non appena il canale risulta libero non lo rende immune
da collisioni, le quali comunque potrebbero accadere nel caso di stazioni che
controllano nello stesso momento un canale vuoto ed inviano contemporaneamente.

\paragraph{CSMA non-persistente} prima di trasmettere, ogni stazione controlla
il canale. Se lo trova libero, inizia ad inviare dati, altrimenti se risulata
occupato, la stazione non esegue un controllo continuo per tasmettere subito il
proprio frame, ma attende un intervallo di tempo casuale prima di ripetere
l'algoritmo (allunga quindi il delay).

\paragraph{CSMA p-persistente} variante che si applica su canali divisi in
intervalli temporali. Quando è pronta a trasmettere ogni stazione controlla il
canale. Se lo trova libero, trasmette subtio con probabilità p, e rimanda fino
all'intervallo successivo con probabilità 1=1-p. Nel caso in cui anche
quell'intervallo risultasse libero, la stazione trasmette oppure rimanda
un'altra volta. Il processo si ripete finchè il frame non è stato trasmesso.

\paragraph{pregi} migliora di molto le prestazioni di ALOHA ed ALOHA slotted,
oltre che ad avere il vantaggio di annullare la propria trasmissione in caso di
collisione (così da risparmiare tempo e banda).

\paragraph{difetti} none

\paragraph{ambiti d'uso} utilizzato prevalentemente in connesioni dove il
rilevamento delle collisioni non è realizzabile (es. connessioni wireless).
Utilizzato nello standard IEEE 802.3 (Ethernet), oltre che nelle reti LAN
Ethernet (nella variante migliorata CSMA/CD).

\section{Protocolli a contesa limitata / Adapdive Tree Walk}

CSMA/CD non è sufficiente se le stazioni trasmittenti sono tante, siccome
aumenta in tal caso il content period, onoltre non è adatto ad un carico di rete
basso. CSMA invce ha caratterisctiche inverse. Cercando infatti di trovare un
metodo che combini le caratteristich emigliori di uno e dell'altro si è capito
che l'unico modo per ridurre le collisioni (anche dei piccoli pacchetti
appositi) è ridurre i contendenti $>$ su questa constatazione si basano i
protocolli a contesa limitata.\\
In questi ultimi, le stazioni vengono divise in gruppi (non necessariamente
disgiunti) ed ongi gruppo si contende uno slot. La suddivisione in gruppi,
inoltre, cambia in base al diverso carico di rete, dato che con un carico alto
c'è più probabilità di collisioni (gruppi piccoli), viceversa con un carico
basso c'è meno probabilità di collisioni.\\
Il protocollo Adaptive Tree Walk segue queste modalità: idelamente partendo 1
singolo gruppo che contende per ogni slot, ed in caso di collisione viene diviso
in due (contenimento x 2 slot diversi). Queto avviene ricorsivamente sui
sottogruppi. Chiaramente (organizzando la suddivisione in gruppi come in grafo
ad albero), in una situazione reale non si parte dal livello 1 ma dal livello
$log(2q)$, dove $q$ è il nmero di stazioni previste che vogliono trasmettere (in
base al traffico dati precedente) in modo tale da distribuirle una per gruppo
(si spera) ed evitare le collisioni (se si partisse più in alto, la possibilità
di collisioni sarebbe molto alta, rendendo di fatto inutile partire da così in
alto).

\paragraph{pregi} garantisce un ritardo limitato in caso di basso carico ed una
buona efficienza in caso di carico più elevato.

\paragraph{ambiti d'uso} viene utilizzato nelle connessioni con più stazioni che
vogliono trasmettere nello stesso mezzo di trasmissione.

\section{Stazione nascosta}

Quello della stazione nascosta è un problema che si verifica nelle comunicazioni
wireless ed è dovuto al fatto che i dispositivi non conoscono l'intera topologia
della rete di cui fanno parte. Questo porta inevitabilmente al crearsi di zone
di interferenza nelle quali si verificano delle collisioni. La stazione nascosta
quindi non è altro che una stazione che vuole inviare un segnale ma non riesce a
ricevere i segnali dei concorrenti a causa della distanza.\\
Supponiamo per esempio di avere tre dispositivi A, B e C: A vuole inviare a B,
prima di farlo, ascolta se ci sono altre connessioni, altrimenti procede
all'invio. C però è troppo distante da A e quest'ultimo non riesce a sentirlo,
ma è abbastanza vicino a B per inviargli dati $>$ lo fa e va in conflitto con
l'invio di A.\\
Il problema della stazione nascosta è stato risolto dal MACA, ossia il
protocollo Multiple Access with Collosion Avoidance ed all'utilizzo di pacchetti
RTS (request to send) e CTS (clear to sent). Questi ultimi, tra l'altro
risolvono anche il problema della stazione esposta.

\section{stazione esposta}

Il problema della stazione esposta invece è l'inverso di quello della stazione
nascosta. Supponiamo di avere 4 dispositivi A, B, C e D: B trasmette ad A, C
vuole inviare un pacchetto a D e controlla la presenza di portante sul mezzo di
trasmissione, rilevando che B non intralcerebbe la trasmissione di C, ma questo
non lo può sapere, e di conseguenza si genera uno stallo inutile.\\

\section{Descrivere i vari tipi di cavo e confronto}

I principali tipi di cavo utilizzato nelle telecomunicazioni sono: il doppino (o
cavo annodato / UTP), il cavo coassiale e la fibra ottica:
\begin{itemize}
	\item Cavo annodato (UTP), cavo formato da una coppia di fili annodati da
		qui la dicitura "twisted" il quale serve a limitare l'interferenza
		reciproca (crosstalk). I due fili sono detti anche "doppini" e sono
		spessi circa 1mm ciscuno. Possono estendersi per diversi km senza
		chiedere amplificazione del segnale e vengono usati per trasmettere dati
		analogici e digitali. UTP3 Bandwidth250MHz, UTP5 Bandwidth600MHz;

	\item Cavo coasiale, essendo pi\ schermato del cavo annodato si pu;
		estendere per distanze maggiori e a velocit' pi' elevate. Viene
		utilizzato maggiormente per la tv via cavo e le MAN e la sua Bandwidth è
		di circa 1GHz;

	\item Cavi in fibra ottica, un sistema ottico è formato da tre componenti
		fondamentali: la sorgente luminosa, il mezzo di trasmissione e il
		ricevitore. Un impulso di luce indica il valore 1 e l'assenza indica il
		valore 0. Il mezzo di trasmissione è una fibra di vetro sottilissima in
		silicio. Quando viene colpito dalla luce, il rilevatore genera un
		impulso elettrico. Collegando ad un estremo una sorgente di luce e un
		rilevatore dall'altro si crea un sistema di trasmissione unidirezionale
		che accetta un segnale elettrico, lo converte e lo trasmette sotto forma
		di impulso luminoso; all'altra estremità della fibra converte nuovamente
		output in segnale elettrico. I cavi in fibra sono molto simili a quelli
		coassiali solamente che appunto non sono avvolti da una calzatura
		conduttrice ma da un rivestimento in silicio. Nei cavi in fibra al
		centro si trova il nucleo di vetro attraverso il quale si propaga la
		luce, diametro di 50 micron per quelle multimodali, mentre dagli 8 ai 10
		micron per quelle monolocali (monomodali). Il nucleo è circondato da un
		rivestimento di vetro chiamato cladding che ha un indice di rifrazione
		più basso. Le fibre si possono collegare in tre modi diversi: connettori
		$>$ perdono circa il 10-20\% del segnale, ma semplificano la
		riconfigurazione dei sistemi; meccanicamente, le estremità sono
		appoggiate tra loro; a fusione, formando una connessione solida.

\end{itemize}

\subsection{Confronto}
Le fibre ottiche rispetto ai cavi in rame sono notevolmente più costose e sono
meno pieghevoli, inoltre più laboriose da unire. D'altro canto però la fibra ha
più bandwidth e tiene meglio il segnale, inoltre è più piccola e leggera. La
fibra è dielettrica e dunque nelle situazioni di maletempo non vi si presentano
problemi di alte interferenze elettriche e inoltre sono più difficili da
intercettare, infatti richiede un intervento fisico sul cavo (derivazioni).

\section{AMPS}

L'AMPS tratta dei telefoni di prima generazione. In AMPS le aree geografiche
sono celle ampie 10-20km, ognuna delle quali utilizza frequenze non utilizzate
dalle celle vicine. Vengono utilizzate celle relativamente piccole e il
riutilizzo delle frequenze di trasmissione delle celle vicine ma non adiacenti.
Nei casi in cui vi sono delle celle il numero di persone [ auntato fino a
sovraccaricare il sistema, viene ridotta la potenza e le celle sovraccaricate
vengono divise in celle più piccole chiamate microcelle.\\
Al centro di ogni celle è presente una stazione di base, la quale comunica con
tutti i telefoni presenti nella cella, questa base è costituita da un computer e
un trasmettitore, ricevitore collegato ad un'antenna. Il sistema AMPS utilizza
832 canali full duplex ognuno costituito da una coppia di canali simplex, ampi
ciascuno 30kHz.

\section{GSM}

Il GSM tratta dei telefoni di seconda generazione, quelli a voce digitale. La
sua struttura è formata da 4 tipi di celle: macro, micro, pico e umbrella.
\begin{itemize}
	\item macro, sono le più grandi, sopraelevate rispetto agli edifici e hanno
		un raggio massimo di 35 km;

	\item micro, come si può intuire sono più piccole delle macro e coprono
		un'ampiezza pari agli edifici;

	\item pico, sono molto piccole, usate nelle aree molto dense, tipicamente
		indoor;

	\item umbrella, è una piccola estensione usata per coprire i buchi tra
		quelle appena citate.
\end{itemize}

Il GSM sfrutta multiplexing a divisione di frequenza, con ogni apparecchio che
trasmette su una frequenza e riceve su una più alta, una singola coppia di
frequenza è divisa in slot temporali e condivisa tra più utenti. Un sistema GSM
ha 124 coppie di canali simplex e supporta otto connessioni separate mediante
multiplexing a divisione di tempo.\\
A ogni stazione attava è assegnato uno slot su una coppai di canali.
Trasmissione e ricezione non avvengono nello stesso intervallo temporale perché
GSM non è in grado di fare entrambe le azioni contemporaneamente. Introducendo
pure l'utilizzo della SIM card, in cui vengono salvati i dati descrittivi
dell'abbonato e ha la funzione di autenticazione e autorizzazione all'utilizzo
della rete.

\paragraph{pregi} interoperabilità tra reti diverse che fanno capo ad un unico
standard internazionale. Introduzione della comunicazione di tipo digitale.

\paragraph{difetti} none

\paragraph{ambiti d'uso} viene utilizzato appunto nella seconda generazione di
telefoni cellulari, ed è il sistema più diffuso al mondo.

\section{CDMA} 

Il CDMA è un sistema di comunicazione mobile introdotto dalla seconda
generazione di telefoni cellulari. Si tratta di un protocollo completamente
diverso da AMPS e GSM, in quanto questi ultimi utilizzano TDM ed FDM, che
suddividono in spazi dedicati ad un singolo utente rispettivamente il tempo e lo
spettro di frequenze disponibili.\\
Nel caso di CDMa, ogni utente trasmette utilizzando l'intero spettro,
contemporaneamente agli altri utenti $>$ cià è possibile perchè ad ogni stazione
viene assegnata una sequenza di bit che codifica l'1 ed una speciale che
codifica lo 0, ed essendo le sequenze delle varie stazioni ortogonali tra loro,
i vari segnali sis sommano linearmente (al posto di collidere ed oscurarsi
completamente) durante la trasmissione. Per estrarre un segnale dagli altri è
sufficiente fare il prodotto scalare della somma dei dsegnali per la sequenza
assegnata alla stazione.\\
Questo procedimento aumente anche la sicurzza in quanto per distinguere il
segnale è necessario conoscere la sequenza di bit assegnata alla stazione che si
vuole ascoltare.\\
CDMA rispetto a GSM e D-AMPS opera in una banda di 1,25 MHz, permettendo agli
utenti di avere un'ampiezza di banda considerevole.

\paragraph{pregi} Grazie all maggiore efficienza spettrale, il CDMA garantisce
uan maggior velocità di trasmissione dati. Provvede inoltre a garantire una
maggior sicurezza rispetto ai predecessori, in quanto il demux è fattibile solo
grazie all conoscenza delle parole di codice.
Migliora l'efficienza di uso della banda: se dei dispositivi non trasmettono ci
saranno minori interferenze. Migliora l'handoff, in quanto non è necessario
cambiare frequenza. Celle vicine percepiranno il medesimo segnale.

\paragraph{difetti} none

\paragraph{ambiti d'uso} viene utilizzato nei telefoni di seconda generazione,
come alternativa a D-AMPS e GSM.

\section{Handoff}
Nell'ambito della telefonia mobile, con handoff si intende la procedura tramite
la quale un terminale cambia canale (frequenza e slot di tempo) durante una
comunicazione.\\
Ogni area geografica è divis in celle, al centro di ogni cella si trava una
stazione base, che comunica con tutti i telefoni che si trovano all'interno
della cella stessa. Quando un telefono mobile inizia ad abandonare fisicamente
l'area descritta da una cella, la stazione base di quest'ultima si accorge del
movimento del terminale e controlla il livello potenza del segnale ricevuto
dalle stazioni adiacenti. Una volta decretato quella più forte, trasferisce il
terminale a quella cella. Il telefono viene informato della nuova centrale di
cambiamento e forzato al cambiamento. Questa procedura è conosciuta come
handoff.

\paragraph{soft handoff} Il telefono è acquisito dalla nuova stazione base prima
che venga interrotto il segnale dalla stazione precedente (non vi è nessuna
perdita di continuità, ma il dispositivo deve essere in grado di gestire più
frequenze contemporaneamente).

\paragraph{hard handoff} la vecchia stazione di base rilascia il telefono prima
che la nuova riesca ad acquisirlo.

\section{UDP}

L'UDP è un protocollo a livello trasporto che aggiunge al protoccolo IP (livello
network) il concetto di porte, ovvero consente di usare socket, cioè
associazioni IP-porta. Si tratta di un protocollo che nasce dall'esigenza di
distinguere i processi e servizi operanti all'interno di una macchina, siccole
di questi ultimi ne possono esistere molti e con compiti anche tanto diversi fra
loro all'intero di un singolo computer conneso in rete $>$ la necessità di
distinguerli nasce proprio per questo motivo.\\
È importante ricordare che all'interno del protocollo UDP non sono previsti
controlli di flusso/gestione (questi sono invece forniti dal protocollo TCP), e
per questo motivo UDP predilige la velocità di trasmissione a discapito della
correttezza del messaggio.\\
Infatti, UDP vieen spesso utilizzato nell'anbito delle videochat o servizi
simili dove la correttezza/integrità del messaggio non è essenziale, ma avere la
velocità di trasmissione più alta possibile e la latenza più bassa possibile
sì.\\
UDP è inoltre utilizzato dal protocollo DNS, che si occupa di associare URL
facilmente memorizzabili ad indirizzi IP $>$ questo è, tra l'altro, proprio il
motivo per il quale il protocollo DNS è esposto ad attacchi di tipo DNS
spoofing.

\paragraph{pregi} è un protocollo rapido ed efficace e soprattutto molto utile
per applicazioni leggere e/o che necesitano di avere una bassa latenza.

\paragraph{difetti} non fornisce alcuna affidabilità sull'integrità del
messaggio in quanto non c'è alcun controllo sul riordinamento dei pacchetti o
sulla ritrasmissione di quelli persi.

\paragraph{ambiti d'uso} vedi sopra (comunicazioni broadcast, comunicazioni
multitask, applicazioni di rete che sono elastiche per quanto riguarda la
perdita di dati).

\section{ARP}
ARP è un protocollo che permette di "tradurre" indirizzi IP in indirizzi MAC, e
per questo motivo risulta fondamentale per le connesioni in entrata all'interno
della LAN (local area Network).\\
Siccome gli indirizzi MAC sono pre-assegnati ed unici per ogni host, avere una
gestione centralizzata non sarebbe efficiente perchè per ogni cambiamento della
rete il gestore centralizzato dovrebbe riaggiornare tutte le corrsipondenze
IP-MAC $>$ la cosa viene infatti gestita in maniera distribuita: quando uno
switch riceve un frame che nella sua ARP-table non è associato a nessun MAC
address, manda in broadcast una ARP request nella quale viene richiesto qual'è
l'host (assieme al suo relativo MAC address) che possiede tale indirizzo IP.
L'host in questione risponde poi con un ACK ed in piggy'backing manda il suo MAC
address. Siccome anche l'ACK stesso si trova in broadcast, anche gli altri host
all'interno della LAN riceveranno l'informazione e la salveranno.\\
Nel caso in cui un nuovo host dovesse connettersi all rete, anche quest'ultimo
manderà una ARP request richiedendo se qualcun altro possiede il suo stesso
indirizzo IP, così che eventuali errori (IP duplicato) possano essere rilevati
ed evitati fin da subito. Inoltre, questa procedura è utile anche per comunicare
agli altri dspositivi in rete di creare l'associazione IP-MAC.

\section{NAT}

Si trata di un protocollo che è stato introdotto a causa della carenza degli
indirizzi IP, la NAT box associa una porta (L4) ad un indirizzo IP locale e
rende in questo modo possibile la connessione di più dispositivi (all'interno
della stessa LAN) ad un unico indirizzo IP "globale". Infatti, NAT utilizza
determinati indirizzi IP riservati, quali 10.0.0.0/8, 17.18.0.0/10 e
192.168.0.0/16 per creare le reti colali $>$ si tratta di classi equivalenti dad
IP classful, ma ciò non è un problema in quanto sono solo locali (ed è pertanto
possibile utilizzare sempre gli stessi indirizzi, purché ciò venga fatto in LAN
diverse).\\
Il protocollo NAT è molto utile perché riduce il fabbisogno di indirizzi IP, ma
non rende la connessione connection oriented, introducendo una falla nel sistema
super robusto (ad attacchi esterni) per cui è stato creato il protocollo IP.
Infatti, nel caso in cui la NAT box non sia raggiungibile, la macchina rimane
solo una LAN.\\
Per le comunicazioni verso l'esterno, il corrispondente dell'ACK è il DHCP, il
quale tuttavia viene gestito questa volta "centralmente", essendo gli IP in
questo caso assegnati allo stesso modo (centralmente).

\paragraph{pregi} permette di raggruppare sotto un unico indirizzo IP assegnanto
dall'ISP molteplici macchine presenti in una sottorete locale.

\paragraph{difetti} viola il modello architetturale del protocollo IP, non
essendo questi più univoci (possibili conflitti, per esempio con connessioni
FTP).

\paragraph{ambiti d'uso} utilizzato all'interno del router.

\section{Flooding}

Il flooding è un modo per instradare i pacchetti IP (pur essendo utilizzato
anche in altri contesti) in cui viene prima copiato il pacchetto e
successivamente inviato a tutti router connessi alla rete (a parte quello da cui
proviene il pacchetto stesso).\\
Si tratta di una startegia che possiede delle caratteristiche peculiari, infatti
utilizzando il flooding si ha la certezza che se un pacchetto può arrivare a
destinazione, ci arriverà sempre per la strada più breve. Questo succede perché
tutte le strade in questione vengono tentate allo stesso momento. Ne deriva che,
se usato male, il flooding può portare facilmente a sovraccarichi di rete.\\
A causa della necessità di dover tenere sotto controllo il flooding (cioè
evitare che sommerga la rete) esiste il campo TTL, ossia time to leave $>$
normalmente nei pacchetti IP è settata a 255, ogni volta che un pacchetto
raggiunge un router il valore decresce di 1, qundo arriva ad essere 0 il
pacchetto viene in automatico eliminato. Inoltre, nel caso in cui un router
riceva un pacchetto uguale a quello che ha inviato, quest'ultimo verrà eliminato
$>$ per fare ciò, è logico pensare che sarebbe necessaria una grande quantità di
memoria dove immagazzinare i pacchetti ricevuti e poterli controllare, ma la
soluzione che è stata adottata è quella di utilizzare solo un unico numero n che
traccia la sequenza di pacchetti inoltrati: se n=5, pacchetti da 0 a 5 vengono
scartati.

\paragraph{pregi} è semplice da attuare ed assicura la ricezione del pacchetto
alla stazione desiderata.

\paragraph{difetti} spreco di banda, pacchetti duplicati (che sono difficili da
gestire) e rischio di cicli infiniti.

\paragraph{ambiti d'uso} utilizzato nello strato network. È molto utile come
metrica di confronto per altri algoritmi di routing (a causa della sua
caratteristica di riuscire a scegliere sempre il percorso più breve). Viene
utilizzato molto anche a livello militare (in situazioni del genere, a causa di
bombardamenti exx che potrebbero colpire un data center, l'avere pacchetti
duplicati all'interno della rete è paradossalmente una cosa positiva).

\section{Choke packet}

Il choke è un pacchetto che viene inviato ad un router che sta immettendo troppi
pacchetti in rete saturando quindi parte della infrastruttura. La funzione dello
choke packet è quella di risolvere il problema dimezzando i pacchetti che il
router che riceve il choke potrà immettere nella rete per un certo periodo di
tempo $>$ nel caso in cui più router contemporaneamente inviino choke packet
nello stesso momento allo stesso router, vi possono ovviamente essere problemi,
in quanto tale router potrebbe finire ad essere bloccato totalmente (nel caso in
cui 1 solo choke packet sarebbe stato sufficiente a risolvere il problema di
saturazione). Infatti, la soluzione che viene adottata è quella di permettere ad
ogni router di ricevere un solo choke alla volta, ignorando quindi "piogge di
choke" in arrivo. Un'altra problematica da considerare è che la rete venga
saturata completamente dai pacchetti inviati ad un router prima che il choke
packet giunga a destinazione $>$ soluzione: choke hop by hop: agiscono come un
normale choke, ma "strozzano" anche i router per i quali passano lungo il
percorso per arrivare al router "pericoloso".

\paragraph{pregi choke} permette di risolvere la congestione della linea.

\paragraph{difetti choke} lento a reagire, in quanto l'host produttore di
pacchetti congestionanti ci mette un po' a ricevere il choke $>$ con choke hop
by hop si ha un miglioramente.

\paragraph{pregi choke hop by hop} rispetto al choke normale, è molto più veloce
nella decongestione della rete.

\paragraph{difetti choke hop by hop} richiede un utilizzo più intensivo del
buffer di trasmissione fra mittente e destinatario.

\paragraph{ambiti d'uso} strato network, si tratta di un algoritmo di controllo
della congestione di rete.

\section{Token bucket}

Per evitare che la rete si saturi a causa di una "pioggia" momentanea di
pacchetti (che causerebbe un fisiologico crollo di prestazioni) possono essere
utilizzati i token bucket $>$ per ogni intervallo di tempo viene generato un
token: qunndo arriverà un pacchetto, questo verr immagazzinato in un buffer, e
potrà proseguire solo se vie è un token disponibile.\\
Nel caso in vui non vi fossero molti pacchetti in arrivo, i token verrano
accumulati fino ad un tetto prestabilito $>$ in questo modo, nei momenti di
maggior congestione della rete, più pacchetti potranno passare
contemporaneamente.\\
L'algoritmo token bucket infatti riprende l'idea del leaky bucket, ma aggiunge
il concetto di token. La differenza sta nel fatto che token bucket non scarta i
pacchetti quando il "secchio" è pieno. Per implementare il token bucket è
sufficiente l'utilizzo di una variabile che tenga conto del numero dei token, e
li diminuisca quando un paccehtto viene inviato.

\paragraph{pregi} rispetto al bucket di tipo leaky, questo algoritmo non scarta
alcun pacchetto e gestisce meglio eventuali burst improvvisi di pacchetti.

\paragraph{difetti} consentendo di trasmetter raffiche di dati insieme,
potrebbero crearsi problemi di sicurezza se la rete venissa compromessa e
finisse in mani sbagliate.

\paragraph{ambiti d'uso} strato network, viene utilizzato per gestire il
traffico in una rete dati. È finalizzato a regolare l'output di trasmissione.

\section{Distance vector routing}

Si tratta del primo algoritmo di routing usato in ARPANET e funziona nel
seguento modo: ogni router, quando si collega alla rete, chiede ai router vicini
le loro tabelle di routing ed enendovi insieme l'informazione riguardo il tempo
che attende per riceverle crea in questo modo la propria tabella di routing $>$
quest'ultima è costruita da un entry per router, il tempo necessario per cui gli
arrivi un pacchetto e quele sia la via migliore tra i router vicini. Per avere
un routing dinamico, che tiene conto del carico della rete e le modifiche alla
topologia, le tabelle di routing si aggiornano continuamente e dinamicamente: se
un collegamento migliora o peggiora di velocità le tabelle di routing vengono
subito aggiornate.\\
Tuttavia, nel caso in vui un nodo venga scollegato, nasce un problema noto con
il nome di count to infinity $>$ siccome i router si baseranno su tutte le altre
tabelle che ancora non sono state aggiornate, il percorso (per i router)
diventerà sempre più lungo, tendendo ad infinito un aggiornamento per volta.
D'altra parte, si tratta di un buon algoritmo per i suoi tempi, in quanto è
molto scalabile (aggiungere un router è un processo molto veloce).

\paragraph{pregi} ogni router è in grado di conoscere il percorso migliore per
arrivare a tutti i nodi adiacenti.

\paragraph{difetti} non tiene conto della banda della linea quantdo sceglie i
percorsi. Esiste la possibilità che si creino cicli infiniti. Impiega troppo
tempo a raggiungere la convergenza.

\paragraph{ambiti d'uso} strato network, ha la funzione di instradare i
pacchetti ai vari router della rete. si tratta di un algoritmo adattivo e quindi
dinamico.

\section{Link state routing}

Si tratta dell'algoritmo di routing che ha di fatto sostituito il distance
vector routing. Nel link state routing, ogni router possiede un identificativo
univoco, inoltre quando si collega alla rete manda ai router a lui collegati un
pacchetto HELLO, ed esi risponderanno con il proprio identificativo, oltre ad
altre informazioni. Dopo di ciò, tramite un apposito ECHO a cui i router
rispondono subito, il nuovo router acquisisce le informazioni riguardo la
velocità di ogni collegamento. Appena queste informazioni vengono raccolte viene
generato un paccehtto che le contiene, che viene inoltrato tramite flooding a
tutti i nodi della rete. Questo pacchetto ha una durata di vita oltre la quale
verrà considerato obsoleto e scartato.\\
Il fatto che questo pacchetto contentente tutte le informazioni venga trasmesso
tramite flooding rappresenta la principale differenza rispetto la distance
vector routing, e ne previene il problema del count to infinity. Infatti, ogni
volta che passa un certo intervallo di tempo oppure in presenza di eventi
speciali quali modifica, disconnessione o connessione del router viene
rieseguito l'intero processo. Un router scarterà uno di questi pacchetti se: ne
ha già ricevuto uno più recente; ha già inoltarto in precedenza il suddetto
file; è già scaduto il suo tempo di vita.\\
Una volta che un router riceve l'intera serie completa di pacchetti
identificativi di ogni router della rete, potra costruire il grado della rete
stessa $>$ sarà in grado di applicare l'algoritmo di Dijkstra e trovare i
percorsi più brevi per raggiungere ogni altro router.

\paragraph{pregi} si tratta di un lagoritmo molto scalabile e che molto
raramente genera cicli, è veloce a trovare il cammino più veloce ed è in grado
di gestire reti molto caotiche.

\paragraph{difetti} i router hanno bisogno di molta memoria e capacità di
calcolo.

\paragraph{ambiti d'uso} strato network, sostituito al distance vector routing.
Molto utilizzato al giorno d'oggi.

\section{QoS}

Con QoS si intende Quality of Service, ossia il livello di qualità raggiunto da
un determinato servizio di rete. I principali indicatori della qualità raggiunta
da un servizio di rete sono 4, anche se nella pratica ogno singola 
applicazione/software valuterà quali consederare più importanti nel contesto
dove opear. Gli indicatori sono:
\begin{itemize}
	\item affidabilità: una applicazione è affidabile quando nessun bit può
		venire trasmesso in maniera incorretta (CRC);

	\item banda: ogni protocollo ed applicazione differisce dalle altre per
		l'esigenza di banda, ossia la velocità di trasmissione dei dati;

	\item delay: che risulta per ovvi motivi importantissimo per applicazioni
		del tipo videoconferenza, live, trascurabile per altre come la
		messaggiistica;

	\item jitter: ossia il grado di variabilità del ritardo di trasmissione. Una
		applicazione con velocità di trasmissione mediamente costante ha un
		jitter basso, viceversa, una applicazione dove la velocità di
		trasmissione è variabile a causa di qualche condizione ha un jitter
		alot. Questo può portare ad una ricezione dei dati ad intervalli molto
		irregolari. Alcuni applicazioni, quali lo streaming video, oppure lo
		screen sharing, risultano inutilizzabili nel caso di jitter alto. Altre
		invece, come quelle di trasferimento file o quelle di posta elettronica,
		non sono molto soggette a problemi anche in caso di jitter alto.
\end{itemize}

\section{Numero di bit necessari per il riconoscimento degli errori di
trasmissione}

Per la gestione degli errori sono state sviluppate due strategie di base: la
prima si basa su una codifica a correzione d'errore; mentre la seconda è una
codifica a rivelazione d'errore. L aprima introduce una ridondanza tale da
riuscire a ricostruire il messaggio in caso di anomalie. La seconda introduce
ridondanza sufficiente a capire che quando si verifica un errore, dando la
possibilità di richiedere una ritrasmissione.\\
Un frame consiste in m bit di dati e r bit ridondanti per i controlli, $n=m+r$ è
la lunghezza totale del frame. Per trovare $d$ errori è necessaria una codifica
con distanza $d+1$. Per correggere $d$ errori è necessaria una codifica con
distanza $2d+1$.

\paragraph{pregi} rilevare semplicemente l'errore permette di diminuire la
quantità di bit dati. Tuttavia rilevazione e correzione permette un minor numero
di invii, e permette la ricostruzione autonoma del frame.

\paragraph{difetti} la semplice rilevazione non permette la ricostruizione del
dato, la rilevazione ecorrezione però necessita del doppio dei bit per poter
essere attuata.

\paragraph{ambiti d'uso} starto data-link. Nelle reti wireless conviene
utilizzare una codifica a correzione dell'errore, così da ricostruire il
messaggio in casi d'errore (se doveses solo rilevare e richiedere un altro
invio, il rischio della presenza di nuovi errori sarebbbe alta, quindi conviene
usare questa).

\section{Go back N}

Si tratta di un protocollo con cui può essere utilizzata una sliding window in
cui il mittente ha duna finestra di $n>1$ ed il ricevente uan finestra di 1
(così che) se un pacchetto viene perso, l'intera finestra a partire dall'errore
deve essere rispedita. Nel caso si prospetti di trovare pochi errori, può essere
considerato un buon protocollo, altrimenti risulta inefficiente in quanto
potrebbe far perdere troppa banda.\\
In generale, è un processo dove il mittente continua a mandare un numero di
frame specificato nella windows size, anche senza aver ricevuto alcun ack. Per
utilizzare Go back N sono quindi necessari un buffer grande n per ricevere, ed
n timer.

\paragraph{pregi} se gli errori non sono tanti, grazie all'invio a raffica di
pacchetti senza attenere l'ack corrispondente, risulta essere un protocollo
molto efficace.

\paragraph{ambiti d'uso} viene utilizzato in generale in tutti quei sistemi dove
la finestra di invio è più grande di quella ricevente (che è invece
particolarmente scarsa).

\section{DES}
DES è uno standard di criptaggio internazionale, creato da IBM su commissione
del governo americano, originariamente avrebbe dovuto avere una chiave di 128
bit e blocchi da 64 bit, ciò però non consentiva ai servizi di intelligence
americana di decriptare (abbastanza velocemente) i messaggi, e per questo motivo
si è deciso alla fine di ridurre la chiave a 56 bit.\\
Nello specifico, si tratta di un product cipher, cioè un cifrario che combina 2
o più trasformazioni, il quale sfrutta infatti 19 passaggi fra una P-box
(production box) ed una S-box (substitution box).\\
P-box: esegue permutazioni dei messaggi in stile cifrario a sostituzione.\\
S-box: sostituisce n bit con m bit: le associazioni fra le varie permutazioni di
n bit e gli m bit di uscita possono essere rappresentate semplicemente da una
tabella.\\
Entrambe le box sono molto efficienti ed è possibile reallizzarle facilmente in
$hw>$ grazie a 19 passaggi fra le due si ottiene un algoritmo sufficientemente
sicuro.\\

\section{Triple DES}

A due anni dalla nascita dell'algoritmo DES, a causa dell'avanzamento
esponenziale nelle capacità di calcolo dei computer, diventa anche per gli
utnetni comuni (e quindi non solo per i servizi di intelligence) troppo facile
decriptarlo $>$ viene per questo motivo creato un nuovo standard, che, tuttavia,
per ragioni sia pratiche che economiche doveva essere retrocompatibile. Si
tratta del Triple DES, che è basato direttamente sul DES originale. Infatti, si
aggiunge semplicemente una ulteriore chiave a 56 bit e viene eseuita poi la
codifica analogamente a quanto avviene nel DES, ma questa volta tutti i passaggi
vengono, di fatto, ripetuti 3 volte.\\
Per rimanere retrocompatibile, il Triple DES sfrutta le proprietà del DEs che
garantiscono a quest'ultimo di essere un metodo di cifratura a chiave simmetrica
$>$ ciò rende l'algoritmo di codifica interscambiabile, infatti il Triple DES
codifica in 3 fasi e decodifica specularmente.\\
Nello specifico, la retrocompatibilità è data da:\\
$CK1 > DK2 > CK1$ (decodifica)\\
$DK2 > CK1 > DK2$ (codifica)\\
Ponendoi $K1 = K2$, i primi due passaggi si annullano e di conseguenza ritorna
ad essere un semplice DES.

\paragraph{pregi} permettono una cifratura a blocchi.

\paragraph{difetti} a causa della dimensione ridotta della chiave DES non è 
molto sicuro, inoltre, sia con Triple DES che con DES lo stesso testo in chiaro 
produrrà sempre il medesimo testo cifrato.

\paragraph{ambiti d'uso} non è più utilizzato: è stato sostituito da AES.

\section{One Time Pad}

Si tratta di una modalità di criptaggio che, nel caso la chiave utilizzata si
asempre nuova, risulta al 100\% sicura. Consiste nel fare lo XOR fra una chiave
casuale lunga quanto il messaggio da criptare ed il messaggio stesso. La parte
più difficile del processo OTP è riuscire a trovare chiavi sempre diverse, in
quanto gli algoritmi al momento esistenti sono in grado di creare solamente
output pseudo-casuali. Questo è molto importante in quanto il OTP può essere
considerato veramente sicuro solamente se la chiave non viene mai ripetuta,
altrimenti facendo lo XOR fra due messaggi con la stessa chiave si otterrevve
proprio il messaggio in chiaro (dove poi tra l'altro si potrebbe applicare anche
la frequency analysis).\\
Per questi motivi, ai giorni nostri OTP viene raramente utilizzato nella sua
variante originale, viene spesso scelto di approssimarlo tramite la tecnica
dello stream cypher.

\paragraph{pregi} in generale OTP se eseguito a dovere, ovvero con chiavi sempre
nuove, non può mai venire compomesso matematicamente $>$ decifrare il messaggio
senza conoscere la chiave è impossibile.

\paragraph{difetti} la chiave non può essere memorizzata digitalmente (si
perderebbe la sicurezza), quindi mittente e destinatario devono entrambi avere
in principio una copia scritta della stessa. Ovviamente, una altro difetto è che
la quantità di dati che possono essere trasmessi equivale al numero di chiavi
che si posseggono.

\paragraph{ambiti d'uso} utilizzato poco, seppur esattamente sicuro, a causa
della sua scarsa praticabilità. In generale il suo utilizzo più comune è
nell'ambito dello spionaggio.

\section{Stream Cypher}

Lo stream cypher è un cossiddetto cifrario di flusso, ovvero una variante del
cifrario a blocchi. Si tratta di una tecnica crittografica che sfrutta un
vettore di iniziazione (IV, initialation vector) ed una chiave, le quali cifrate
insieme generano un keystream completamente indipendente dal resto in chiaro $>$
in seguito è possibile cifrare nuovametne il keystream con la chiave un numero
arbitrario di volte al fine di ottenere keystream sempre differenti, da mettere
infine a XOR con il testo.\\
In generale, lo stream cypher quindi simula il OTP, mettendo in fila gli IV
criptandoli più volte. Anceh in questo caso, quindi, l'IV non deve essere
riutilizzato più volte. Inoltre, se l'IV dovesse essere troppo corto,
diventerebbe possibile effettuare attachi ad analisi di frequenza per decifrare
il testo (per esempio WEP), rendendo la crittografia non troppo sicura.

\paragraph{pregi} garantisce una alta elasticità agli errori ed inoltre la
keystream è indipendente dai dati.

\paragraph{difetti} bisogna stare attenti a non riutilizzare né l'IV né i
keystream che ne derivano, altrimenti lo stream cypher è vulnerabile.

\paragraph{ambiti d'uso} semplificare il protocollo OTP.

\section{CIDR}
Inizialmente si era pensato di distribuire gli indirizzi IP in pacchetti di 3
dimensioni differenti:
\begin{itemize}
	\item 16 milioni ~ di IP (classe C);
	\item 65 mila IP (classe B);
	\item 256 IP (classe A).
\end{itemize}
Questo, teoricamente, sarebbe stata una scelta efficacie, tuttavia l apratica e
la società ne hanno, nel corso degli anni, dimostrato gli evidenti difetti.
Infatti, 256 IP sembrano pochi e quando una società o azienda doveva scegliere
quale pacchetto comprare, spesso e volentieri era propensa  acomprare il secondo
pacchetto (classe B), usandone in media solo 50 > a causa di un mero fattore
psicologico, un grandissimo spreco di indirizzi IP: presto questi ultimi vennero
a mandcare, e si fu constretti a correre ai rapari, grazie proprio al CIDR
(classless inter-domain routing).\\
Grazie ad esso gli indirizzi IP non vengono più forniti per classi fisse, ma in
blocchi di dimensioni variabili (purché sempre potenze di 2), utilizzando una
maschera sub-net mask, che permette di distinguere la parte dell'indirizzo di
rete dell'IP da quella per gli host $>$ questa maschera tuttavia complica il
compito dei router, i quali non possono più raggruppare gli indirizzi IP in base
alla classe, ma devono tenere conto di tutti i bit dell'indice (con conseguenti
tabelle degli indirizzi che diventano enormi, assieme alle ricerche sulle
stesse).\\
Questo problema è stato risoltao grazie alle aggregate entries $>$ i router
creano delle regole di routing che uniscono i vari indirizzi con i bit più
significativi. Per consentire un indirizzamento corretto ha la priorità la
regola più lunga.

\section{ICPM}
Si tratta di un sotto-protocollo del protocollo IP che ha il compito di gestire
eventi inaspettati, per esempio:
\begin{itemize}
	\item può mandare un echo;

	\item senala al mittente se in un pacchetto il tiem to leave è arrivato a 0;

	\item manda choke packets (funzione deprecata, gestita a livello trasporto);

	\item segnala una destinazione non raggiungibile.
\end{itemize}

Si tratta di un sotto-protocollo he viene in generale utilizzato per testare la
connessione di rete, più nello specifico, la sua velocità e i suoi percorsi.

\paragraph{pregi} protocollo molto utilizzato per tesstare le performance di una
connessione.

\paragraph{difetti} alcuni dei pacchetti sono ormai obsoleti perchè la
risoluzione del problema viene gestita meglio in altri livelli.

\paragraph{casi d'uso} strato network.

\section{IPV6}

Nonostante NAT e CIDR, gli indirizzi IPV4 restano pochi, perciò è stata creata
la versione 6 del protocollo IP ed è cominciata la sua diffusione.\\
IPV6 ha indirizzi di 128 bit, non è retrocompatibile con IPV4, ma lo è con i
suoi sotto-protocolli (UDP, TCP, DNS, ...).\\
IPV6 semplifica l'intestazione, infatti prevede solo 8 campi rispetto ai 13
della versione precedente. Ciò consente ai router di elaborare molto più
velocemente i pacchetti in arrivo.\\
Migliora il supporto per le opzioni, rendnedo campi che prima erano obbligatori,
opzionali. Un altro grande passo in avanti riguarda la sicurezza, oltre ad un
miglioramento generale del QoS.\\
Ormai, il grosso della rete è basato sul protocollo IPV4, ed essendo IPV6 non
direttamente retrocompatibile, il pasaggio ad esso sis ta rivelando lento ed
impegnativo. C'è sicuramente la necessità di mantenere il supporto ad IPV4
almeno un altro paio di decenni.

\section{IPSEC}

IPSEC è un insieme di regole e protocolli di telecomunicazioni per la creazione
di connessioni sicure su una rete. IPSEC aggiunge al protocollo IP la
crittografia e l'autenticazione allo scopo di renderlo più sicure e trasparente.
IPSEC è diviso in due parti principali:
\begin{itemize}
	\item descrive 2 nuovi header che possono essere aggiungi ai pacchetti, che
		portano la security identifier, controllo di integrità ed altre
		informazioni;

	\item chiamata ISAKMP (Internet Security Association and Key Management
		Protocol), che è un protocollo per la negoziazione delle chiavi.
\end{itemize}

Transport mode e tunnel mode sono due modalità di funzionamento di IPSEC.\\
Nel primo caso, il pacchetto IP originale viene inserito dopo l'header.
Nel secondo caso, invece, il pacchetto IP originale viene incapsulato nel corpo
di un nuovo pacchetto IP con un nuovo header (che diventa un Authentication
Header).

\paragraph{pregi} il framework può vivere anche se altri protocolli diventano
obsoleti o ne vengono aggiungi di nuovi.

\paragraph{difetti} nel caso della modalità tunnel, il pacchetto IP avrà un
header in più, aumentando quindi la dimensione del pacchetto nell'insieme.

\paragraph{casi d'uso} strato network.

\section{HMAC}

Facente parte dell'Authentication Header dell'IPSEC, si tratta di un campo di
lunghezza variabile che contiene la firma digitale del payload. Quando la
security association è stabilita, viene anche stabilito quale algoritmo di firma
verrà utilizzato.\\
Normalmente la crittografia  achiave pubblica non viene utilizzata in quanto i
pacchetti devono essere spediti rapidamente. Siccome la connessione security
association è basata sullo scambio di una chiave per la criptazione simmetrica,
la stessa viene utilizzata nella computazione della firma digitale.\\
Un semplice modo per farlo è quello di calcolare l'hash del pacchetto più la
chiave simmetrica. Uno schema di questo tipo è chiamato HMAC. Garantisce sia
l'integrità che l'autenticità del messaggio. È possibile utilizzare HMAC con
qualsiasi funzione di hash. Questo per rendere possibile una sostituzione della
funzione nel caso non fosse sufficientemente sicura.

\end{document}
