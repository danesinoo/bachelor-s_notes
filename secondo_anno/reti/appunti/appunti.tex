\documentclass{article}

\usepackage{listings}
\usepackage{xcolor}
\lstset { %
    language=C++,
    backgroundcolor=\color{black!5}, % set backgroundcolor
    basicstyle=\footnotesize,% basic font setting
}

\usepackage{amsfonts}
\usepackage{amsmath}
\usepackage{graphicx}
\usepackage{hyperref}
\usepackage{caption}
\usepackage{fancyhdr}
\usepackage{geometry}
	\geometry{height=24 cm}
	\geometry{left=2.5 cm}
	\geometry{right=2.5 cm}
	\geometry{top=2 cm}
	\geometry{headheight=1 cm}

\setcounter{secnumdepth}{2}


\title{\vspace{2cm}\textbf{Appunti di Programmazione\ a Oggetti}}
\author{\vspace{3mm}4 ottobre 2022}
\date{\vspace{3mm} \textbf{Rosso Carlo}}

\begin{document}

\begin{titlepage}
	\maketitle
	\thispagestyle{empty}
\end{titlepage}
\tableofcontents
\newpage

\section{Introduzione}
\paragraph{Computer network} un insieme di computer connessi, in un modo per cui
altrimenti sarebbero disconnessi o autonomi.

\paragraph{client-server model} coinvolge due processi: un processo è il client,
l'altro è il server. Tutti i client si connettono al medesimo server ed
interagicono con esso o mediante esso.
 
\paragraph{peer-to-peer} ogni utente mantiene un database locale e una lista
degli utenti. I dati sono trasferiti da un utente all'altro.

\paragraph{PAN} Personal Area Network sono le comunicazioni che avvengono tra
dispositivi appartenenti alla stessa persona (e.g. apple watch, airpods,
smartphone).

\paragraph{LAN} Local Area Network una rete privata che opera in singolo
edificio: per esempio il wifi.

\paragraph{MAN} Metropolitan Area Network una rete che si estende per un centro
abitato: per esempio i cavi della TV.

\paragraph{WAN} Wide Area Network una rete che si estende per un continente (tipo
le reti satellitari).

\paragraph{Protocol Layering} I protocolli sono dei regole sulle comunicazioni:
mediante i protocolli i computer sanno come interagire tra loro.
I layer servono per astrarre la tecnologia ed essere in grado di controllare le
comunicazione. In realtà esiste solo il livello hardware. 
Ciascun layer controlla le informazione inerenti al proprio layer e le 
passa al layer sottostante. Ogni layer comunica con un layer del proprio
livello mediante un protocollo. Per passare da un layer a quello superiore o
sottostante si utilizzano le interfacce (il layer sottostante da origine,
compone quello sovrastante).\\
Una serie di layer e protocolli si chiama architettura del network. 

\section{Il layer fisico}
Il livello fisico è quello mediante cui sono passati i dati. Ci sono tre
tipologlie di trasmissioni: 
\begin{itemize}
	\item con cavo (wired);
	\item senza cavo (wireless);
	\item satellitari.
\end{itemize}

\subsection{Wired}
Lo scopo del layer fisico è quello di trasportare o trasferire bit da un
computer ad un altro. Le reti di comunicazione più comuni sono formate da cavi
in rame o dalla fibra ottica.\\
Ciascun tipo di comunicazione ha le proprie caratteristiche: frequenza, 
larghezza di banda (bandwidth), ritardo (delay), costo, semplicità 
nell'installazione e manutenibilità. La larghezza di banda indica la velocità 
di trasmissione di bit di un medium, si misura in Hertz (Hz).

\subsubsection{Persistent Storage} 
Per portare i bit da un computer ad un altro possiamo salvarli in una memoria
esterna e trasportare fisicamente la memoria fino al secondo computer dove gli 
saranno resi accessibili.
Dal momento che la densità con cui sono memorizzati i bit in un dispositivo
fisico aumenta in modo esponenziale (nel tempo, scala bene) questo metodo rimane uno dei
più economici.\\
NB le trasmissioni da un computer ad un altro richiedono circa un
giorno: il tempo che impiega la memoria a raggiungere il secondo computer
partendo dal primo. D'altro canto con questo metodo è possibile spostare enormi
quantità di dati nell'arco di una sola giornata; la velocità media di queste
trasmissioni è di 70 Gbps per ciascuna scatola $60 \cdot 60 \cdot 60 cm^3$.\\
In questo caso la larghezza di banda è di circa 70 Gbps; questa tecnologia è
molto economica, ma ha anche un ampio delay, i dati ci mettono una giornata ad
arrivare dalla "richiesta".

\subsubsection{Twisted Pairs}
Twisted Pair è una connessione mediate due cavi in rame torti (torcere) tra di
loro, come il DNA, in questo modo i campi magnetici prodotti dai cavi si
annullano tra loro e l'informazione arriva più lontano. Un bit è rappresentato
dalla differenza di carica dei due cavi: se i due cavi trasportano la medesima
carica il segnale è 0 altrimenti è 1. I doppini intrecciati sono in grado di
percorrere chilometri senza bisogno di ripetitori. Se sono utilizzati  
più doppini intrecciati, allora sono protetti da uno scudo che diminuisce le
interferenze dall'esterno. Le twisted pairs trasmettono abbastanza dati:
qualche centinaio di Mbps su un tragitto di qualche chilometro. Sono semplici
da produrre e costano poco, per questo motivo sono molto diffusi a livello
globale.\\
Diverse LAN utilizzano questa tipologia di cavi.

\paragraph{full-duplex} twisted pairs che trasmettono i bit in
entrambe le direzioni nello stesso momento.

\paragraph{half-duplex} twisted pairs che trasmettono i bit in entrambe le
direzioni, ma in momenti diversi. 

\paragraph{simplex} twisted pairs che trasmettono i bit in una solo direzione.\\

Più i due cavi sono torti tra loro, minore sarà l'interferenza che producono, 
migliore sarà il segnale nelle lunghe distanze. Per aumentare la velocità 
delle comunicazioni è sufficiente aumentare lo spessore dei cavi. 
Un ulteriore distizione consiste nell'UTP: unshielded twisted pair, in 
questo caso i cavi non sono protetti.

\subsubsection{Coaxial Cable}
Il cavo coassiale è più schermato e ha una larghezza di banda maggiore rispetto
ai doppini intrecciati. Il cavo coassiale sta venendo sostituito con la fibra
ottica. Fu utilizzato per le comunicazioni telefoniche ed è ancora adoperato per
la televisione e per le comunicazioni a livello MAN.

\subsubsection{Power Lines} 
Le connessioni per alimentare gli elettrodomestici di casa possono essere
utilizzate per trasmettere dati. Poichè accendere e spegnere la luce causa
interferenze, queste connessioni sono utilizzate in ambito domestico, per cui la
perdita di dati nella connessione è poco rilevante. Permette di
trasmettere fino a 500 Mbps.

\subsubsection{Fiber Optic}
Le connessioni mediante la fibra ottica potrebbero arrivare a 50,000 Gbps, ma
non siamo in grado di convertire i segnali luminosi così trasmessi alla medesima
velocità. Ad oggi, il limite della larghezza di banda è di circa 100 Gbps. In
generale, è simpatico notare che è più costoso trasportare i bit piuttosto che
calcolarli, ma è più veloce trasportare i bit invece di calcolarli.\\
Un sistema di trasmissione ottico è formato da tre componenti: la sorgente, dove
viene prodotto il segnale luminoso; il media, il materiale attraverso cui passa
la luce (si tratta di qualcosa di molto trasparente); il detector, lo strumento
che assorbe la luce e la trasforma in segnale elettrico. Il media consiste in un
cavo in vetro molto sottile. Se il diametro del cavo è inferiore ai 10 microns
allora si dice che la fibra è single-mode; un media così sottile permette alla
luce di viaggiare solamente in linea retta: sono diminuite le dispersioni. 
Una fibra single-mode permette di percorrere maggiori distanze senza ripetitori: 
arriva fino a 100 Gbps per una distanza di 100Km senza ripetitore.\\
Alla fibra del tipo single-mode, si affianca quella multimode, in questo caso il
cavo può essere anche più spesso di 50 microns. Il cavo multimode è
attraversato da diversi raggi di luce nello stesso momento. La fibra multimode è
più economica, ma la larghezza di banda diminuisce con l'aumentare della
distanza.\\

Due fibre possono essere connesse in tre modi: 
\begin{itemize}
	\item gli estremi terminano in un cottore che è connesso ad un fiber socket,
		in questo caso è perso dal 10 al 20\% del segnale;
	\item possono essere accostate meccanicamente; ad un tecnico richiede 5
		minuti unire due fibre in questo modo e si perde il 10\% del segnale;
	\item i due estremi sono fusi, saldati, assieme. In questo caso si ottiene
		una fibra che è praticamente come una fibra unica, ma è un tipo di
		collegamento più complesso. 
\end{itemize}

Ci sono due tipi di sorgente di luce: LED (light emitting diodes) e laser
semiconduttori;

\begin{table}[ht]
	\label{sorgenti_luminose}
	\centering
	\begin{tabular}{|l|l|l|}
		\hline 
		item & LED & semiconductor laser \\
		\hline
		data rate & low & high \\
		fiber type & multi-mode & multimode o single-mode \\
		distanza & breve & lunga \\
		durabilità & lunga & breve \\
		sensibilità della temperatura & nessuna & sostanziale \\
		costo & economica & costosa \\
		\hline
	\end{tabular}
	\caption{differenze tra una sorgente led e una laser}
\end{table}

I raggi di luce sono assorbiti da un fotodiodo che trasforma la luce in segnali
elettrici. Il tempo di risposta del fotodiodo limita la velocità di banda a 100
Gbps.\\

\subsubsection{Differenze tra fibra ottica e cavo in rame}

\begin{table}[ht]
	\label{ottica_rame}
	\centering
	\begin{tabular}{|l|l|l|}
		\hline
		 item & fibra ottica & cavo in rame \\
		 \hline
		 data rate & fino a 100 Gbps & centinaia di Mbps \\
		 distanza & fino a 50 km & 5 km \\
		 interferenze & temperatura & tutte, meno temperatura \\
		 diametro & molto fino & molto grosso \\
		 peso & leggera (100 kg) & pesante (8000 kg) \\
		 costo di installazione & basso & elevato \\
		 direzione & unidirezionale o due frequenze & duplex \\
		 tap & impossibile & molto semplice \\
		 personale & pochi la sanno maneggiare & molti elettricisti \\
		 costo dell'interfaccia & elevato & economico \\
		 \hline
	\end{tabular}
	\caption{differenze tra fibra ottica e cavo in rame}
\end{table}

\subsection{Wireless}
Le telecomunicazioni wireless avvengono tutte con la medesima tecnologia: sono
sfruttate le onde elettromagnetiche per propagare le informazioni (proprio come
avviene per le comunicazioni wired). In questo caso le onde sono propagate in
tutte le deirezioni (in realtà dipende dalla frequenza delle onde utilizzata). 
La quantità di informazioni trasferibile dipende dalla
potenza che arriva e dalla larghezza di banda. Per mandare le informazioni sono
utilizzate le onde: più onde sommate assieme riescono a descrivere una funzione
che rappresenta i bit. In particolare, se la derivata della funzione descritta è
maggiore di 0 allora stiamo mandando un bit di valore 1, se la derivata della
funzione è minore di 0 allora stiamo mandando un bit di valore 0. Per capire il
segno della derivata è sufficiente memorizzare il valore dell'onda, misurarlo di
nuovo e confrontare i due valori. Avendo più frequenze è possibile modificare
la somma delle frequenze d'onda in modo più preciso: ogni volta che viene
aggiunta una frequenza in più è possibile raddoppiare la quantità di
informazioni che sono mandate.\\

Le connessioni wireless si distinguono per il range delle lunghezze d'onda, ma
anche per le singole frequenze utilizzate: produrre alcune onde
elettromagnetiche è più economico di altre. Per esempio produrre le onde
elettromagnetiche che trasmettono i canali della radio è piuttosto economico.
Comunicare mediante onde ultraviolette è ben più costoso (inoltre queste onde
sono nocive alle forme di vita). Minore è la frequenza d'onda maggiore è la
distanza che quell'onda riesce a percorrere, minore è il costo per produrla. Ci
sono dei pro e dei contro a questo: un'onda che non si ferma mai, che passa
attraverso la Terra, è intercettabile ovunque, per cui deve trasmettere
informazioni non segrete, altrimenti ha bisogno di essere criptata. Questo tipo 
di onda trasmette meno dati di un onda più costosa.\\

Il numero di oscillazioni al secondo di un'onda è chiamato frequenza e si misura
in Hertz (Hz). La distanza tra due massime (oppure tra due minime) è chiamata
lunghezza d'onda. I bit sono codificati modificando la portata (ampiezza), la
frequenza o la fase dell'onda. Gli ultravioletti e i raggi gamma sono in grado
di trasferire più informazioni, perchè la loro frequenza è maggiore; ma sono
difficili da produrre e da modulare, non passano attraverso gli oggetti (ogni
tanto anche il tempo atmosferico li blocca) e sono nocivi per gli animali (noi).

\subsubsection{Frequency hopping spread spectrum}
Questa tecnologia permette di diminuire le interferenze e rende le onde
elettromagnetiche più difficili da rintracciare: le frequenze per trasmettere le
informazioni cambiano centinaia di volte al secondo. viene utilizzato
soprattutto in ambito militare.

\subsubsection{direct sequence spread spectrum}
Sono utilizzate più frequenze d'onda che si possono sovrapporre a quelle di
altri. La connessione funziona anche se il dato è corrotto, perchè solo una
frazione viene corrotta. Il ricevitore del segnale avrà un "codice" per
decifrarlo. Genera interferenze con altre onde sulla medesima frequenza, ma non
risulta un problema per chi riceve le informazioni in questo modo grazie alla
cifratura.

\subsubsection{Ultra-WideBand}
Viene adottata moltissima banda, almeno $500MHz$, e sono trasmessi una serie di
impulti ad bassa intensità molto velocemente. Dal momento che il messaggio è
molto diluito tra le frequenze le interferenze hanno poca importanza. Non
trasmette tutto il tempo per cui non crea interferenze per altre tecnologie. 

\subsection{Analizziamo le frequenze}

\subsubsection{Radio Transmission}
Le frequenze radio (RF) sono semplici da generare, viaggiano per grandi distanze
e penetrano gli edifici; sono unidirezionali. A basse frequenze, il segnale
passa attraverso gli ostacoli, ma l'intensità diminuisce rapidamente. Ad altre
frequenze, il segnale viaggia dritto, ma rimbalza sugli ostacoli. Le
interferenze con gli altri utenti sono un problema. 

\subsubsection{Microwave Transmission}
Le onde viaggiano dritte, per cui il trasmettitore e l'antenna devono essere
meticolosamente allineati. Si possono allineare più ricevitori in fila. La
distanza a cui devono essere piazzati i ripetitori è il quadrato dell'altezza.
Il segnale trasmesso da una torre alta 100m viaggia 80 km. Le microonde non
passano attraverso gli ostacoli.\\
Aumentando la frequenze aumenta il data rate: oggi connessioni a 10GHz sono
comuni. Oltre i 4GHz le onde sono assorbite dall'acqua, umidità compresa.\\
Le microonde sono economiche: non serve chiedere il permesso per piazzare i
cavi, è sufficiente un ripetitore ogni 50 km.

\subsubsection{Infrared Transmission}
Le comunicazioni ad infrarossi sono molto utilizzate: sono economiche,
direzionali e semplici da produrre; ma non passano attraverso gli oggetti. Sono
utilizzate per le comunicazioni all'interno di una sola stanza (per esempio il
telecomando della TV). Non sono intercettabili: non si propagano al di fuori
della stanza.

\subsubsection{Light Transmission}
Sono molto semplici da utilizzare, ma hanno molte interferenze: il vento, la
temperatura, la pioggia o la nebbia.

\subsection{Codificare bit con le onde}
Grazie alle serie di Fourier siamo in grado di descrivere matematicamente le
onde. Grazie ad esse riusciamo a codificare i bit utilizzando le onde
elettromagnetiche. Purtroppo, a seconda della frequenza utilizzata un'onda
(elettroagnetica) si attenua in modo diverso.

\paragraph{Bandwidth} la larghezza del range di frequenze che sono trasmesse
senza che siano eccessivamente attenuate.

\paragraph{Baseband} i segnali che utilizzano le onde che partono dalla frequenza
0 e arrivano ad una frequenza massima. 

\paragraph{Passband} i segnali che utilizzano un range di onde che non partono
da 0 (sono traslate le baseband).

\paragraph{Data rate massimo} $2B\log_2(1+S/N)$ bits/s, dove $B$ sono le onde che non
sono filtrate e $S/N$ è il rapporto segnale/interferenza.

\subsection{Digital Modulation}
Le informazioni sono trasmesse modificando costantemente il voltaggio,
l'intensità della luce o l'intensità del suono. In genere un canale è utilizzato
nello stesso momento da più trasmettitori, questa tecnologia è chiamata
multiplexing (l'abbiamo già vista!). Il multiplexing è ottenuto in diversi modi:
trasmettendo segnali diversi in istanti diversi, utilizzando varie frequenze
oppure codificando l'informazione.

\subsubsection{Baseband transmission}
\paragraph{NRZ} Non-Return-to-Zero: il metodo più semplice per codificare i bit 
fa corrispondere 1 al voltaggio positivo e 0 a quello negativo; oppure 1 alla 
presenza di luce e 0 all'assenza.\\
In realtà non si utilizza una sola onda per trasmettere i bit: se si utilizzano
due onde si possono mandare due bit in una volta sola, in particolare ogni onda
rappresenta un bit. Per questo motivo, invece di usare il data rate è comodo
introdurre il symbol rate (anche chiamato baud rate).

\subsubsection{Clock Recovery} Con il metodo NRZ, il ricevitore ha bisogno di
scandire il tempo in modo efficacie: se una serie di 0 o di 1 si susseguono si
rischia di perdere informazioni. Sono bandati centinaia, talvolta migliaia, di
bit al secondo, i sistemi per misurare il tempo in modo così preciso sono
costosi, per questo in genere si ricorre a metodi di codifica differenti.

\paragraph{Manchester ecoding} Sono trasferiti la metà dei bit, ma il tempo è
scandito bene: un uno è codificato come segnale positivo e segnale nullo, mentre
lo 0 è codificato come segnale nullo e segnale positivo. In questo modo non si
sussegue più volte il medesimo segnale, ma i bit sono ben scanditi.

\paragraph{NRZI} Non-Return-to-Zero Inverted: 1 è codificato come transizine
dallo stato precedente ad uno diverso, se lo stato rimane invariato allora viene
codificato lo 0. Questa tecnologia viene utilizzata per connettere le porte USB.
In questo modo lunghe serie di 1 non causano alcun problema (lunghe serie di 0
sì).

\paragraph{4B/5B} Ad ogni sequenza di 4 bit da trasmettere è aggiunto un bit. Il
bit aggiuntivo è utilizzato per intervallare troppi 0 di seguito. In questo caso
c'è un 25\% di overhead (meglio del 100\% del Manchester encoding).

\subsubsection{Balanced signals}
Alcuni componenti che trasmettono i messaggi elettrici sfruttano la tecnologia
DC. Questo crea un problema: si spreca energia a mandare segnali non bilanciati.
Un segnale è bilanciato se ci sono lo stesso numero di segnali positivi e
negativi in brevi istanti di tempo.

\paragraph{Bipolar encoding} lo 0 è rappresentato dal segnale nullo, mentre l'1
è rappresentato prima da un segnale positivo e poi da un segnale negativo, e
così da capo. In questo modo il segnale sarà sempre bilanciato.

\paragraph{8B/10B} si dividono 8 bit in due gruppi una da 3 bit e uno da 5 bit.
I primi 3 bit sono codificati in 4 bit bilanciati rispetto ai 5 bit che sono
mappati su 6 bit. Dal momento che si aggiungono bit lo stesso codice può essere
mappato in modi diversi, la mappatura è scelta per bilanciare le cariche del
segnale mandato.

\subsubsection{Passband transmission}
Questa tecnologia è utilizzata principalmente nelle comunicazioni wireless: in
questo tipo di comunicazione è scomodo far partire la prima frequenza da 0 (le
antenne dovrebbero essere gigantesche). Per trasmettere le informazioni si
modula l'ampiezza (capacità), la frequenza o la fase del segnale.

\paragraph{ASK} Amplitude Shift Keying, l'ampiezza di un'onda indica il segnale:
ampiezza nulla codifica lo 0, ampiezza non nulla codifica 1. Più lughezze d'onda
sono sovrapposte per rappresentare più bit nello stesso momento.

\paragraph{FSK} Frequency Shift Keying, è cambiata la frequenza della portata
dell'onda. Le frequenze sono predeterminate per cui il client e il server hanno
prestabilito la codifica delle frequenze della portata.

\paragraph{Phase Shift Keying} l'onda è traslata di 0 o di 180 gradi sul suo
periodo. Per esempio traslare di 0 gradi corrisponde allo 0, mentre traslare di
180 gradi corrisponde all'1. In questo caso si tratta di Binary PSK. In realtà,
si trasla di 45, 135, 225 o 315 gradi per trasmettere un simbolo. In questo modo
ad ogni cambio di fase sono mandati due simboli contemporaneamente. Inoltre c'è
sempre una traslazione, per cui siamo più tranquilli riguardo ad eventuali
interferenze.\\

Si può cambiare solo la fase o la frequenza nello stesso momento, perchè le due
variabili sono legate tra loro. Di solito, l'amplitude e la fase sono modificate
nello stesso momento, in questo modo una sola onda è in grado di codificare otto
simboli in una volta sola. Per comodità del programmatore la fase e
l'amplitude sono codificate come cordinate sul piano complesso: la fase
rappresenta l'angolo e l'amplitude il modulo. Ora siamo in grado di
rappresentare le onde con due coordinate. Punti vicini tra loro codificano
sequenze di bit simili, per diminuire la quantità di errori di chi riceve.

\subsection{Multiplexing}
Multiplexing vuol dire mandare più messaggi nello stesso momento.
Ci sono diversi metodi per implementare il multiplexing:
\begin{itemize}
	\item tempo;
	\item frequenze;
	\item codifica.
\end{itemize}

\paragraph{FDM} Frequency division multiplexing: le frequenze sono divise in
slot, ciascun canale utilizza tutte le frequenze dello slot a lui assegnato.\\
Ciscun canale è diviso da una guard band per evitare interferenze. Veniva usato
per le linee telefoniche, oggi si preferisce dividere il tempo a disposizione
per mandare il segnale, piuttosto che la fascia di onde utilizzabili.

\paragraph{OFDM} Orthogonal Frequency Division Multiplexing: come nel caso
precedente le frequenze sono divise in canali. Ciascun canale è a sua volta
diviso in sottocanali che mandano dati in modo indipendente, in questo modo
anche se un sottocanale ha qualche interferenze il segnale arriva mediante un
sottocanale diverso.

\paragraph{TDM} Time division Multiplexing: ad ogni utente è assegnato un arco
di tempo nel quale può mandare informazioni, in quell'arco di tempo ha a
disposizione l'intera banda. Perchè questo sistema funzioni tutti gli utenti
devono essere sincronizzati.

\paragraph{STDM} Static time division multiplexing: aggiungendo l'aggettivo
statico, vuol dire che il tempo assegnato ha ciascun utente è deciso
staticamente: a run-time.

\paragraph{CDM} Code division multiplex (access): il segnale è diviso in una banda 
più larga e tutti gli utenti comunicano contemporaneamente. La stazione assegna 
ad ogni utente il vettore di una base (algebra). La base ha come dimensione il 
numero di utenti. Ciascun utente trasmette il proprio vettore per codificare un 
1 e la negazione del proprio vettore per codificare uno 0. Poichè tutti i vettori 
appartengono ad una base, i vettori sono linearmente indipendenti e conoscendo il vettore 
assegnato all'utente che vuoi ascoltare sei in grado di decodificare le onde. Per riuscire 
ad instaurare questo tipo di comunicazione c'è bisogno che i due utenti che
vogliono comunicare si mettano d'accordo prima.

\end{document}
