\documentclass[11pt]{article}
 \renewcommand{\familydefault}{\sfdefault}

\usepackage{listings}
\usepackage{xcolor}
\lstset { %
    language=C++,
    backgroundcolor=\color{black!5}, % set backgroundcolor
    basicstyle=\footnotesize,% basic font setting
}

\usepackage{amsfonts}
\usepackage{amsmath}
\usepackage{graphicx}
\usepackage{hyperref}
\usepackage{caption}
\usepackage{fancyhdr}
\usepackage{geometry}
	\geometry{height=24 cm}
	\geometry{left=2.5 cm}
	\geometry{right=2.5 cm}
	\geometry{top=2 cm}
	\geometry{headheight=1 cm}

\setcounter{secnumdepth}{2}
\linespread{1}
\renewcommand{\labelitemi}{-}


\begin{document}

\begin{center}
	\huge{\textbf{ECloth}}
\end{center}

\section{Abstract}
"ECloth Mountain" è un'azienda speccializzata nella vendita di abbigliamento 
tecnico da montagna, ha appena aperto il proprio sito di e-commerce e ha
commissionato il proprio database. \\
Dal sito di "ECloth Mountain" è possibile registrarsi e creare un account, per
salvare i propri dati e per poter effettuare acquisti. Inoltre è possibile
creare varie liste di prodotti, per poterli salvare e visualizzare in un secondo
momento. \\
L'utente per registrarsi deve inserire i propri dati personali, mentre
l'indirizzo e il metodo di pagamento non sono obbligatori. L'indirizzo e il
metodo di pagamento sono necessari al momento dell'acquisto, ma l'utente può 
decidere di non salvarli. L'utente attraverso il proprio account può rivedere 
gli acquisti effettuati e le liste di prodotti salvate.

\section{Analisi dei requisiti}

\subsection{Descrizione testuale}

Nella base di dati sono presenti i dati degli \textbf{utenti}, che si registrano
sul sito per effettuare gli acquisti. Di ogni utente sono noti: 

\begin{itemize}
	\item nome
	\item cognome
	\item email
	\item password
\end{itemize}

Ogni utente può memorizzare i dati di una \textbf{carta di credito}. Di ogni
carta di credito sono noti:

\begin{itemize}
	\item numero della carta
	\item intestatario della carta
	\item data di scadenza
	\item codice di sicurezza (CVV)
\end{itemize}

Ogni utente può memorizzare i dati di vari indirizzi di spedizione. Di ogni
\textbf{indirizzo} sono noti:

\begin{itemize}
	\item via
	\item numero civico
	\item città
	\item CAP
	\item stato
\end{itemize}

Ogni utente può avere diversi carrelli. Di ogni \textbf{carrello} sono noti:

\begin{itemize}
	\item nome
	\item costo totale
	\item data di creazione
	\item data di ultima modifica
	\item prodotti contenuti
\end{itemize}

Ogni utente può effettuare un \textbf{ordine}, effettuando il \textit{checkout}
del carrello desiderato. Di ogni ordine sono noti:

\begin{itemize}
	\item numero dell'ordine
	\item costo totale
	\item data di creazione
	\item data di ultima modifica
	\item prodotti contenuti
	\item provider del pagamento
	\item status del pagamento
	\item indirizzo di spedizione
\end{itemize}

I pagamenti sono gestiti mediante un provider esterno, che si occupa di
effettuare il pagamento e di notificare il sito di "ECloth Mountain" del
risultato dell'operazione. \\
Gli utenti possono acquistare dei prodotti, presenti sul sito. Di ogni
\textbf{prodotto} sono noti:

\begin{itemize}
	\item nome
	\item descrizione
	\item prezzo
	\item quantità
	\item tag
\end{itemize}

Non è possibile acquistare un prodotto se la quantità disponibile è minore di
quella richiesta. 

\subsection{Glossario dei termini}

\begin{table}[ht]
	\centering
	\begin{tabular}{|l|l|l|}
		\hline
		\textbf{Termine} & \textbf{Descrizione} & \textbf{Collegamenti} \\
		\hline
		Utente & Persona che si registra sul sito & Carta di credito,
		Lista, Indirizzo \\
		\hline
		Carta di credito & Carta di credito di un utente & Utente \\
		\hline
		Indirizzo & Indirizzo di spedizione di un utente & Utente,
		Ordine \\
		\hline
		Lista & Lista di prodotti, con relativo costo totale & Utente,
		Prodotto \\
		\hline
		Carrello & Lista con un nome & Entità figlia di Lista \\
		\hline
		Ordine & Lista con un identificativo e i dettagli di pagamento
		& Indirizzo, Entità figlia di Lista \\
		\hline
		Prodotto & Prodotto in vendita sul sito & Lista, Categoria \\
		\hline
		Categoria & Categoria di prodotti & Prodotto \\
		\hline
	\end{tabular}
	\caption{Glossario dei termini}
\end{table}

\subsection{Operazioni}

\begin{table}[ht]
	\centering
	\begin{tabular}{|l|l|l|}
		\hline
		\textbf{Operazione} & \textbf{Tipo} & \textbf{Frequenza} \\
		\hline
	\end{tabular}
	\caption{Operazioni}
\end{table}


\end{document}
