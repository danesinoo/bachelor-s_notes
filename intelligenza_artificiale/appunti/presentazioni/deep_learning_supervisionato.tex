\section{Deep Learning Supervisionato}

Per risolvere il problema della linearità, ovvero per riuscire a classificare
correttamente input che non sono linearmente separabili, ci è venuto in mente di
adottare funzioni non lineari.
Inoltre, mettendo più livello uno sopra l'altro, si ottiene una rete neurale profonda in
grado di approssimare funzioni complesse.\\
L'elaborazione gerarchica (profonda) è una caratteristica fondamentale della
computazione neurale nel cervello. La codifica dell'informazione diventa più
complessa ed astratta passando a livelli più elevati (profondi) della
gerarchia.\\
Per poter allenare una rete profonda sono necessari molti
pattern di addestramento (già etichettati) e un'enorme potenza di calcolo. Una
caratteristica interessante dei modelli che sfruttano il deep learning è che è
in grado di imparare da dati grezzi, anche rumorosi, senza bisogno di
pre-elaborazione.\\
Se aggiungiamo molti strati nascosti, il segnale di errore deve passare molti
livelli di backpropagation. Con una funzione del tipo sigmoide, la derivata
tende a 0 su una rete satura, ovvero quando il valore della sinapsi tende a 0 o
a 1. Questo porta a problemi di vanishing gradient, ovvero il gradiente diventa
troppo piccolo e la rete non riesce più ad apprendere.\\
Il problema del vanishing gradient viene arginato con alcune tecniche:
\begin{itemize}
	\item \textbf{inizializzazione dei pesi furba} (per esempio, nell'intorno dello 0,
		dove la derivata della sigmoide è massima);

	\item \textbf{learning rate adattivo}: piuttosto che costante;

	\item \textbf{regolarizzazione}: rete di grandi dimensioni insieme a 
		tecniche di regolarizzazione avanzata, come weight decay, dropout, etc.;

	\item \textbf{ottimizzatori del secondo ordine}: viene stimata la curvatura del
		gradiente e si aggiusta il learning rate di conseguenza;

	\item \textbf{funzione di attivazione ReLU}: al posto di sigmoide, che non satura il 
		gradiente.
\end{itemize}

I metodi sempre adottati per migliorare le prestazioni di una rete neurale sono:
l'adozione di dataset più grandi, l'utilizzo di hardware più potente e
l'adozione di architetture convoluzionali (per esempio, per immagini).

\section{Convolutional Neural Networks}

La CNN è una rete profonda che include almeno uno strato convoluzionale, in cui
i neuroni nascosti non sono interamente connessi con lo strato precedente, ma
hanno campi recettivi locali. 
Lo strato convoluzionale, di solito, è seguito da
uno strato di pooling per ridurre la dimensionalità ed enfatizzare le
caratteristiche più interessanti.
Nella parte più profonda della rete è inserito almeno uno strato nascosto
standard, interamente connesso con quello precedente.
L'ultimo strato nascosto è interamente connesso con lo strato di output.\\
Ogni neurone nascosto dello strato convoluzionale ha un campo recettivo locale,
che codifica una feature specifica, per questo motivo i neuroni in questo strato
sono anche chimati filtri (o kernel). Il numero di filtri definisce quante
feature sono rappresentate in ciascun livello (proprio perché ogni filtro
codifica una feature).
Ciascun filtro è applicato all'intera immagine attraverso un'operazione di
convoluzione. Fondamentalmente, i filtri sono applicati a una porzione piccola
dell'immagine, scorrendo su tutta l'immagine, è come se venisse passato l'input
poco per volta.\\

Iperparametri di una CNN:
\begin{itemize}
	\item \textbf{Numero di neuroni nascosti}: specifica quanti filtri usare in
		ciascun layer;
	
	\item \textbf{Dimensione del kernel}: definisce il campo recettivo del
		filtro, ovvero la dimensione del filtro stesso, quindi la dimensione
		dell'input che il filtro considera alla volta;

	\item \textbf{Stride}: la dimensione del passo con cui il filtro scorre
		sull'immagine;

	\item \textbf{Padding}: aggiunge zeri attorno ai bordi dell'immagine, in
		modo che il filtro possa essere applicato anche ai bordi;
\end{itemize}

\subsection{Pooling}

Lo strato di pooling viene inserito dopo uno strato convoluzionale per ridurre
la dimensione dell'immagine e quindi il numero di parametri.

\subsection{Adversarial Examples}

Possiamo indurre errori di classificazione in una rete neurale profonda facendo
modifiche ad hoc all'immagine di input. Le modifiche non sono distinguibili
all'occhio umano. Le modifiche sono studiate a partire dall'immagine di input in
modo tale da massimizzare la funzione di errore e tendenzialmente modelli
diversi sono ugualmente vulnerabili.
