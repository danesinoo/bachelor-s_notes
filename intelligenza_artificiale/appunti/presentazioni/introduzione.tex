\section{Introduzione}

\paragraph{Che cos'è l'intelligenza?}
La capacità di un agenti di affrontare e risolvere con successo situazioni e
problemi nuovi o sconosciuti.\\
I libri di testo classici definiscono l'IA come lo studio di agenti
intelligenti, che percepiscono il loro ambiente e producono azioni volte a
massimizzare la probabilità di successo nel raggiungere i loro scopi.

\paragraph{Intelligenza Artificiale stretta} si riferisce a qualsiasi
intelligenza artificiale in grado di eguagliare o superare un essere umano in un
compito strettamente definito e strutturato.

\paragraph{Intelligenza Artificiale generale} dovrebbe consentire alle macchine
di applicare conoscenze e abilità in diversi contesti anche di tipo nuovo. Si
tratta di un obiettivo non ancora realizzato e non è detto che sia possibile.\\

Lo scopo dell'IA può essere definito come quello di costruire "agenti
intelligenti". In particolare, l'IA studia come riprodurre in un computer 
processi mentali complessi. Questo conduce a due diverse prospettive:
\begin{itemize}
	\item construire dispositivi più intelligenti, che si avvicinano e superano
		l'intelligenza umana (prospettiva ingegneristica);

	\item costruire e testare ipotesi specifiche sui meccanismi all'interno 
		della scatola (cervello), per simulare e studiare il 
		comportamento umano anche quando questo non è ottimale (prospettiva 
		delle scienze congnitive).
\end{itemize}
