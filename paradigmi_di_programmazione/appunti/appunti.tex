\documentclass{article}
\usepackage{amsfonts}
\usepackage{amsmath}
\usepackage{graphicx}
\usepackage{hyperref}
\usepackage{caption}
\usepackage{fancyhdr}
\usepackage{geometry}
	\geometry{height=24 cm}
	\geometry{left=2.5 cm}
	\geometry{right=2.5 cm}
	\geometry{top=2 cm}
	\geometry{headheight=1 cm}

\setcounter{secnumdepth}{2}

\newtheorem{definition}{Def.}[section]

\title{\vspace{2cm}\textbf{Appunti di Paradigmi}}
\author{\vspace{3mm}4 ottobre 2022}
\date{\vspace{3mm} \textbf{Rosso Carlo}}

\begin{document}

\begin{titlepage}
	\maketitle
	\thispagestyle{empty}
\end{titlepage}
\tableofcontents
\newpage

\section{Introduzione}
\begin{definition}[Paradigma]
	Dal greco \textit{paradeigma}: modello, esempio. Modello di riferimento,
	termine di paragone. \\
	In informatica, un paradigma l'insieme delle tecniche e dei metodi
	considerati adeguati ad affrontare una classe determinata, anche se ampia,
	di problemi.
\end{definition} 

Esempi:
\begin{itemize}
	\item concorrenza: più linee di esecuzione contemporanee, asincrone, che
		condividono l'uso di un insieme di risorse in modo (possibilmente)
		coordinato;

	\item parallelismo: più linee di esecuzione contemporanee, sincrone, che
		eseguono in modo coordinato lo stesso calcolo su di un partizione dei
		dati di ingresso;

	\item in rete: collaborazione con altri sistemi attraverso la comunicazion
		esincrona su di una interfaccia di rete;
	
	\item distribuzione: un sistema è costituito da nodi indipendenti che,
		attraverso una rete non affidabile, devono coordinare l'esecuzione dello
		stesso lavoro, condividendo il consenso sullo stato complessivo del
		sistema;

	\item reattività: un sistema distribuito costruito sulle basi della
		comunicazione asincrona tramite messaggi da cui ottiene caratteristiche
		di flessibilità, resilienza, scalabilità e reattività;
\end{itemize}

Utilizziamo Java per implementare i paradigmi di programmazione.
Java è un linguaggio OO, a memoria gestita (GC), con controllo statico dei tipi,
basato su classi ad ereditarietà singola, con una sintassi simile a C e C++.
Java è compilato in un linguaggio intermedio (bytecode), il cui modello di
esecuzione è descritto nella specifica della JVM.
Il bytecode può essere eseguito in molti modi: interpretato, compilato durante
l'esecuzione (aka JIT), compilato prima dell'esecuzione (GraalVM).

\subsection{Java}
Ogni file deve chiamarsi come l'oggetto pubblico che contiene:
NomeClasse.java.\\
Il percorso delle directory deve corrispondere al package in cui l'oggetto si
trova.\\
Nei sorgenti, o nel CLASSPATH, devono esserci tutti i tipi nominati dai
sorgenti.\\
L'ordine di compilazione non è importante.\\
Il comando "java" avvia la JVM ed esegue il bytecode contenuto nel CLASSPATH.\\
Il formato .jar è il più comune formato di distribuzione del codice nella
piattaforma Java. Si tratta di un archivio compresso zip contenente alcuni file
specifici che descrivono il suo contenuto e come usarlo.\\
L'unità principale di organizzazione del codice è la Classe. Ogni oggetto fa
necessariamente riferimento alla definizione di un aClasse, che determina la
struttura del suo stato ed il codice che opera su tale stato. Il codice
eseguibile deve necessariamente essere incluso in una Classe: come metodo
dotato di un nome e richiamabile da altri oggetti, o come blocco anonimo
eseguito alla creazione di ciascuna istanza. Le classi formano l'insieme dei
Tipi che il compilatore riconosce in un programma Java. Per convensione i
package sono denominati con nomi di dominio, in ordine inverso.\\
Una classe non può usare un'altra classe qualsiasi: deve averne visibilità e
dichiarare l'intenzione di usarla. Una classe può usare una qualsiasi altra
classe all'interno dello stesso package senza indicazioni particolari. Un file
sorgente può contenere al più una classe pubblica, e deve chiamarsi come la
classe contenuta. "import", per importare una classe.
Una classe può contenere variabili, metodi, altre classi, blocchi di codice
anonimi.\\

Interfacce: una interfaccia dichiara le caratterisctiche di un tipo senza
fornire una sua implementazione. Le classi possono dichiarare di implementare
una interfaccia fornendo l'implementazione richiesta.
Una interfaccia può essere estesa solo da un'altra interfaccia. Una interfaccia
può avere visibilità pubblica o di package.

\section{Concorrenza}

\begin{definition}[programma concorrente]
	Teoria e tecniche per la gestione di più processi sulla stessa macchina che
	operano contemporaneamente condividendo le risorse disponibili.
\end{definition}

Questo richiede sempre maggiore gestione della concorrenza e dell'accesso
contemporaneo alle stesse risorse. Inoltre, alcune di queste tecniche si sono
rivelate problematiche dal punto di vista della sicurezza.
Per gestire più linee di esecuzione all'interno dello stesso processo è stato
ideato il concetto di thread. I thread condividono le risorse di uno stesso
processo, rendendo più economico il costo di passaggio da un ramo di esecuzione
all'altro.
La programmazione distribuita implica la comunicazione fra entità che non
possono avere stato condiviso.
La programmazione funzionale tratta preferibilmente dati immutabili, con qualche
concessione alla mutabilità per lo stretto necessario. Lo stato può essere
distinto o condiviso.
La programmazione concorrente si pone nel qudrante più difficile, dove lo stato
è mutabile e condiviso, e quindi l'accesso e l'intervento su di esso va
coordinato e gestito.

Problemi della concorrenza:
	\begin{itemize}
		\item non determinismo;
		
		\item starvation;

		\item race condition;

		\item deadlock;
	\end{itemize}

Tipologie di concorrenza:
	\begin{itemize}
		\item collaborativa: co-routine;
		
		\item pre-emptive: processi, threads;

		\item real-time: processi, threads;

		\item event driven (o async): future, events, treams;
	\end{itemize}

\paragraph{collaborativa}
I programmi devono esplicitamente cedere il controllo ad intervalli regolari.
Casi d'uso: embedded, very high perfomance.

\paragraph{pre-emptive}
Il sistema operativo è in grado di interrompere l'esecuzione di un programma
e sottrargli il controllo delle risorse per affidarle al programma seguente.

\paragraph{real-time}
Il sistema operativo garantisce prestazioni precise e prefissate nella
suddivisione delle risorse fra i programmi.

\paragraph{event driven/async}
I programmi dichiarano esplicitamente le operazioni che vanno eseguite e
lasciano all'ambiente di esecuzione la decisione di quando eseguirle e come
assegnare le risorse. Si tratta di un metodo poco comune a livello di sistema
operativo, ma molto popolare nell'organizzazione delle applicazioni.

\begin{definition}[Future]
	Un oggetto Future rappresenta un calcolo che prima o poi ritornerà un
	valore. Si può verificare se il calcolo è stato completato, e quindi
	ottenere il valore risultante, o controllare se sia ancora in corso.
\end{definition}

\begin{definition}[sezione critica]
	Si dice sezione critica la parte di codice in cui vengono acceduti i dati
	condivisi. Permettere a più thread di trovarsi contemporaneamente nella
	sezione critica porta ad errori (race condition).
\end{definition}

\textit{syncronized} è una parola chiave che applicata ad un blocco di
istruzioni impedisce che sia percorso contemporaneamente da più thread. Può
decorare due tipi di raggruppamenti di istruzioni: un blocco di istruzioni
semplice (\{\dots\}) o un metodo. In entrambi i casi è utilizzato un monitor
lock o intrinsic lock.\\
Si può anche controllare manualmente una sezione critica, con un lock o con un
semaforo. Un lock è un semaforo che può essere acquisito da un solo thread alla
volta. Un semaforo è un contatore che può essere incrementato o decrementato da
più thread contemporaneamente (più o meno).\\


\subsection{Thread-safety}

Se possibile è meglio evitare di condividere dati fra i thread; se necessario si
utilizzano alcuni strumenti specifici.

Problemi: 
\begin{itemize}
	\item deadlock

	\item race condition: il risultato non è deterministico
\end{itemize}

\subsubsection{Atomicità}

Per incrementare un contatore condiviso, si utilizza una delle classi del
package "java.util.concurrent.atomic":
\begin{itemize}
	\item Integer: AtomicInteger;

	\item Long: AtomicLong;

	\item Object: AtomicReference;

	\item IntegerArray: AtomicIntegerArray;

	\item \dots (altri array);
\end{itemize}

Queste classi garantiscono che la modifica del valore che contengono sia
"atomica" (quindi thread-safe). Inoltre, si assicurano di evitare deadlock.\\
L'AtomicReference è formata da un puntatore e da un boolean che funge da
semaforo.

\subsubsection{Volatile}

La parola chiave volatile viene utilizzata per dichiarare una variabile che
deve sempre essere letta "dalla memoria principale". Questa parola impone la
regola "Happens-Before Order": la scrittura su un campo volatile avviene prima
di ogni altra lettura su quel campo. La garanzia fornita da volatile è utile
alla correttezza del programma solo se nessun thread scrive nella variabile
volatile un valore dipendente dal valore che ha appena letto dalla stessa
variabile.

\subsubsection{Concurrent Data Structures}

Nel package "java.util.concurrent" sono presenti le collezioni ottimizzate per
la concorrenza. Queste collezioni danno garanzia di atomicità ed ordinamento
delle operazioni.\\
Alcune implementazioni offrono metodi come reduce, search o foreach, che
suddividono automaticamente il lavoro fra i thread.\\

BlockingQueue è una coda che rende possibile scegliere la semantica
dell'operazione di accodamento e prelievo. Permette quindi maggiore
flessibilità nella gestione dell'accesso ai dati su più thread.

\subsubsection{ThreadLocal}

Una variabile "ThreadLocal$<$T$>$" esiste in una copia differente ed indipendente
per ciascun Thread che attravera la sua dichiarazione. Ovvero, ogni volta che
tale variabile è richiesta in un thread diverso, suddetto thread la copia e 
crea una nuova istanza. Assomigliano alle variabili globali.

\subsection{Parallel Streams}
Lo Stream rappresenta l'iterazione su di un insieme di cardinalità non nota,
potenzialmente infinita. Si tratta di un modello in cui l'esecuzione sequenziale
e quella parallela non richiedono modifiche di codice. Questo è possibile perchè
sono permesse operazioni con proprietà più restrittive.\\
Uno Stream richiede la definizione dell'algoritmo di calcolo che avverrà sopra i
suoi elementi, indipendentemente dal loro recupero, che è fissato.

Ci sono alcune flag, per indicare le proprietà dell'operazione:
\begin{itemize}
	\item concurrent;

	\item distinct: garantisce che non ci siano duplicati secondo equals();

	\item immutable;

	\item nonnull;

	\item ordered: l'ordine di elaborazione è lo stesso dell'ordine di
		origine;

	\item sized: garantisce che il numero di elementi sia noto;

	\item sorted: mantiene gli elementi ordinati secondo il loro ordinamento
		naturale o dato da un Comparator;

	\item subsized;
\end{itemize}

\subsubsection{SplitIterator}
Per esprimere una classe parallelizzabile servono dei metodi che consentano alla
sorgente di esplicitare la suddivisione del lavoro.\\
Con "tryAdvance()" lo stream fornisce un elemento al thread che ci lavora.

\subsubsection{Collector}

Nell'operazione di riduzione, l'accumulazione del risultato avviene creando
nuovi valori (altrimenti si usa forEach). Questo non è sempre conveniente.
Lo Stream, di default, decide autonomamente quanto parallelismo usare attraverso
il ForkJoinPool.

\subsubsection{Approfondimento}
La descrizione degli stream fino ad ora proposta è incompleta: manca un
protocollo esplicito per gestire la terminazione dello stream; e gli errori sono
gestiti come eccezioni. Per questo, sono sviluppati i Reactive Streams. Le basi
del modello sono i seguenti concetti:
\begin{itemize}
	\item Observable: emette i dati, è l'equivalente di uno stream;

	\item Scheduler: esegue i task;

	\item Subscriber: riceve i dati, è l'equivalente di un Consumer;

	\item Subject: può consumare un Observable per produrre un altro Observable,
		permette di introdurre modifiche sostanziali nel flusso di dati;
\end{itemize}

L'idea del modello è un'implementazione dell'Observer Pattern.
In questo modo si ottiene una semantica più ricca, maggiore regolarità nella
composizione e indipendenza dal modello di esecuzione.

\section{Programmazione Distribuita}

La programmazione distribuita sono la teoria e le tecniche per la gestione di
più processi su macchine diverse che operano in modo coordinato allo svolgimento
di un unico compito. Un insieme di macchine che esegue un algoritmo distribuito
è detto sistema distribuito. Caratteristiche:
\begin{itemize}
	\item affidabilità: anche se alcuni nodi sono fermi o in errore, il
		sistema continua a funzionare;

	\item scalabilità: il sistema può essere esteso con l'aggiunta di nuovi
		nodi, se quelli già presenti non fossero sufficienti;

	\item diffusione: il sistema può essere distribuito su un'area geografica
		ampia;

	\item concorrenza: i vari nodi di esecuzione operano contemporaneamente;

	\item totale asincronia: l'ordine temporale delle operazioni non è
		strettamente condiviso; se fosse necessario sarebbe imposto con
		opportuni mezzi (meglio evitare);

	\item fallimenti imperscrutabili: un guasto è indistinguibile da un ritardo.
		L'unica risorsa in comune tra i nodi è il collegamento di rete, per cui
		le principali astrazioni disponibili riguardano l'invio di un messaggio
		e l'attesa della ricezione di un messaggio (la sincronizzazione rallenta
		parecchio);
\end{itemize}

\subsection{Serializzazione}
Un passo fondamentale è la serializzazione, ovvero il metodo con cui un oggetto
viene predisposto per la trasmissione in un messaggio. La serializzazione prende
un oggetto e lo traduce in un messaggio, utilizzando il metodo "marshall()".
La deserializzazione è il processo inverso ed usa il metodo "unmarshall()".\\
Tendenzialmente sono utilizzate connessioni HTTP, per cui si usa XML o JSON, in
modo che i messaggi siano leggibili anche da un uomo.\\

\subsection{Comunicazioni}
Le astrazioni di vase che abbiamo a disposizione per la comunicazione tra più
JVM corrispondono direttamente alle caratteristiche del protocollo TCP/IP:
socket e datagram.\\

\subsection{Socket}
Un Socket è un'astrazione per la comunicazione bidirezionale punto-punto fra due
sistemi. Il Socket è composto da un client e un server. Un Socket fornisce un
InputStream e un OutputStream per ricevere e trasmettere dati nel collegamento.
Regole:
\begin{itemize}
	\item sono thread-safe, ma un solo thread può scrivere o leggere per volta;

	\item i buffer sono limitati, quindi si rischia di perdere dati;

	\item lettura e scrittura possono bloccare il thread (deadlock);

	\item le connessioni possono avere caratteristiche particolari (urgent data,
		\dots);

	\item una volta terminato, il socket va chiuso esplicitamente;

	\item le stream sono continue, per cui è necessario un carattere di
		separazione;
\end{itemize}

\subsection{Datagram}
Non c'è garanzia di ricezione o ordinamento in arrivo. La dimensione massima è
di 64Kb. Per inviare o ricevere abbiamo una sola classe, senza distinzione di
operatività. Con i Datagram la dimensione del messaggio è nota, inoltre è
possibile inviare messaggi a più indirizzi contemporaneamente. Contro (rispetto
ai Socket): non c'è garanzia nè segnale di consegna; i messaggi sono inviati in
una sola direzione; messaggi lunghi subiscono una forte penalità di
affidabilità.

\subsection{URL}
Permette connessioni del tipo HTTP a basso livello.

\subsection{HTTP client}
Si tratta di un'evoluzione della classe URL. Utilizza il build pattern. Una
volta costruita una richiesta, la si esegue per ottenere la risposta.

\subsection{Channel}
Un Channel è un'astrazione per la comunicazione bidirezionale punto-punto fra
due sistemi. Per comunicazioni tra più di due sistemi, si consiglia di
utilizzare i channel, che permettono una gestione più sintetica e leggibile.

\subsection{Punti Deboli}
\begin{itemize}
	\item the network is reliable;

	\item latency is zero;

	\item bandwidth is infinite;

	\item the network is secure;

	\item topology doesn't change;

	\item there is one administrator;

	\item transport cost is zero;

	\item the network is homogeneous;
\end{itemize}

Oltre a queste tecnologie a basso livello, sono disponibili i framework,
talvolta opensource. Per cui si utilizza del codice già scritto, per risparmiare
tempo. I framework sono librerie in cui alcuni casi d'uso sono già implementati.
Per scegliere il framework più adatto bisogna leggere la documentazione per
capire quali proprietà sono garantite e quali sono messe in secondo piano. Usare
un framework significa accettare le sue proprietà e i suoi limiti. Inoltre, può
succedere che un framework sia aggiornato, nel qual caso può risultare
necessario un refactoring.\\

\subsection{CRDT (Conflict-free Replicated Data Type)}
Un CRDT è un tipo di dato che può essere replicato in modo asincrono su più
nodi, fornendo la garanzia che esiste un modo di riconciliare tutte le possibili
modifiche risolvendo ogni possibile conflitto. Ecco alcuni esempi di queste
strutture dati:
\begin{itemize}
	\item grow-only counter;

	\item positive-negative counter;

	\item grow-only set;

	\item 2-phase set: set di elementi aggiunti e set di elementi rimossi;

	\item last-writer-wins set;
\end{itemize}

\subsection{Back-pressure}
Con back-pressure si intende la resistenza che il componente successivo può
opporre ai dati provenienti dal componente precedente della catena di
elaborazione. Questa strategia permette ad ogni nodo della catena di dichiarare
quanti dati è in grado di elaborare.

\subsection{Reactive Streams}

\begin{itemize}
	\item Publisher: fornisce un numero arbitrario di elementi di tipo T;

	\item Subscriber: consuma gli elementi di tipo T e controlla il flusso degli
		elementi in arrivo;

	\item Subscription: rappresenta la relazione tra un Publisher e un
		Subscriber;

	\item Processor: estende Publisher e Subscriber, permettendo di trasformare
		ed elaborare gli elementi;
\end{itemize}





\end{document}
