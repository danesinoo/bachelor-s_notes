\documentclass{article}

\begin{document}

\begin{table}[ht]
	\centering
	\begin{tabular}{|l|l|}
		\hline
		  & \\
		Microprogrammazione & Cablato \\
		  &  \\
		\hline
		  & \\
	versatile & medesimi calcoli\\
  complessa & poco complessa\\
  microcircuiti & circuito unico\\
  lenta & veloce\\
  costosa & economica\\
  & \\
  \hline
	\end{tabular}
	\caption{Microprogrammazione e Embedded System}
\end{table}


\paragraph{Ciclo di fetch/execute completo:}
\begin{enumerate}
    \item instruction fetch: si estrae l'istruzione dalla memoria;
		\item instruction operation decoding: si comprende l'operazione da eseguire;
    \item operand address calculation: si calcola l'indirizzo degli operandi;
    \item operand fetch: si recuperano gli operandi;
    \item data operation: l'ALU esegue le operazioni richieste;
    \item operand address calculation: si calcola l'indirizzo del risultato;
    \item operand store: si salva il risultato;
    \item interrupt check: si contralla se la CU ha inoltrato interrupt;
    \item interrupt: si recuperano le informazioni dell'interrupt;
    \item instruction address calculation: si calcola l'indirizzo dell'istruzione;
\end{enumerate}

\paragraph{Descrivere il bus di sistema}
\begin{itemize}
	\item[--] serve per trasportare i dati;
	\item[--] livello dati, indirizzi e controllo;
	\item[--] linee, ampiezza dei bit;
	\item[--] arbitro del bus;
\end{itemize}

\paragraph{Mapping dei blocchi}
\begin{itemize}
 \item associazione diretta:
 	\begin{itemize}
		\item la linea del blocco è univoca: indirizzo del blocco modulo n. righe della cache;
		\item veloce;
		\item swap molto frequente;
	\end{itemize}

 \item associzione completa:
 	\begin{itemize}
		\item il blocco è posizionato nella prima riga disponibile;
		\item miss lento ad essere identificato;
		\item efficienza della cache massima;
	\end{itemize}

 \item associazione a gruppi:
 	\begin{itemize}
		\item il set del blocco è univoco: indirizzo del blocco modulo n. set nella cache;
		\item swap non troppo frequente;
		\item capacità della memoria sufficientemente ottimizzata;
	\end{itemize}
\end{itemize}

\paragraph{Write through e write back}
\begin{itemize}
	\item write through:
	 \begin{itemize}
		 \item le momorie sono aggiornate contemporaneamente ogni volta che un dato è modificato;
		 \item collo di bottiglia tra le momorie;
		 \item la memoria è sempre aggiornata;
	 \end{itemize}

	\item write back:
	 \begin{itemize}
		 \item la memoria superiore è aggiornata quando c'è uno swap nella memoria inferiore;
		 \item dirty bit;
		 \item la memoria non è aggiornata, problema nei multicore;
	 \end{itemize}
 \end{itemize}

\paragraph{CD-ROM}
\begin{itemize}
	\item[--] memoria ottica;
	\item[--] una traccia a spirale;
	\item[--] settori che contengono la medesima quantità di dati;
	\item[--] velocità lineare costante;
	\item[--] prima del blocco sequnza 00 FFx10 00, tempo lettura, codice correzione finale;
\end{itemize}

\paragraph{Memorie a semiconduttori}
\begin{itemize}
 \item[--] random access;
 \item[--] ROM e RAM;
 \item[--] DRAM e SRAM;
 \item[--] ROM per il BIOS;
 \item[--] EPROM erasebla;
 \item[--] EEPROM lente, modificabili con l'elettrcità;
 \item[--] flash come le EEPROM ma veloci;
\end{itemize}

\paragraph{DRAM}
\begin{itemize}
	\item[--] volatili, scrittura e lettura;
	\item[--] analogiche;
	\item[--] il dato è salvato in un condensatore;
	\item[--] refresh;
	\item[--] lettura: il condensatore è scaricato, il segnale passa attraverso un amplificatore, refresh;
	\item[--] scrittura: si accede attraverso la linea indirizzo, nel condensatore si piazza la carica;
  \item[--] economiche;
	\item[--] sono utilizzate per la RAM;
\end{itemize}

\paragraph{Miss e come diminuirli}
\begin{itemize}
	\item[--] miss di primo accesso: irrimediabile;
	\item[--] miss per capacità: maggiore dimensione dei blocchi;
	\item[--] miss per conflitto: aumentare il numero di vie, separare cache dati e cache istruzioni;
\end{itemize}

\paragraph{I Raid}
\begin{itemize}
	\item[--] 0: stripping;
	\item[--] 1: mirroring di ciascun disco;
	\item[--] 2: stripping stretto, hamming dei dischi necessari aggiungendo dischi;
	\item[--] 3: stripping stretto, parità;
	\item[--] 4: stripping largo, parità, collo di bottiglia;
	\item[--] 5: stripping largo, parità con round robin;
	\item[--] 6: stripping largo, parità con round robin, algoritmo di backup aggiuntivo;
\end{itemize}

\paragraph{Dischi Rigidi}
\begin{itemize}
	\item[--] memorie magnetiche;
	\item[--] tracce e sezioni e gap tra le divisioni;
	\item[--] CAV o MZR;
	\item[--] economici, capienti, possono essere messi in raid;
	\item[--] estraibili o no;
	\item[--] cilindri o no;
	\item[--] doppia faccia o no;
\end{itemize}

\paragraph{Gestione I/O programmata}
\begin{itemize}
	\item[--] la CPU si occupa di tutta la gestione delle periferiche;
	\item[--] controllo, test, lettura, scrittura;
	\item[--] la CPU richiede periodicamente lo stato della periferica;
	\item[--] la CPU si occupa di trasferire i dati;
\end{itemize}

\paragraph{Gestione I/O interrupt driven}
\begin{itemize}
	\item[--] la CPU si occupa di tutta la gestione delle periferiche;
	\item[--] la CPU riceve gli interrupt;
	\item[--] Controllo, test, lettura, scrittura;
\end{itemize}

\paragraph{DMA}
\begin{itemize}
	\item[--] la CPU manda alla DMA tutte le comunicazioni per i dispositivi di I/O;
	\item[--] la DMA comunica con le periferiche;
	\item[--] la DMA inoltra eventuali interrupt;
	\item[--] la DMA comunica parole o blocchi;
\end{itemize}

\paragraph{Codice Hamming}
\begin{itemize}
	\item[--] nessun errore, si può correggere l'errore, non è possibile correggere l'errore;
	\item[--] identifica 2 bit sbagliati, ne corregge 1;
	\item[--] il codice di correzione è salvato con i dati;
	\item[--] in lettura è controllato il dato estratto;
\end{itemize}


\end{document}
