\documentclass{article}

\usepackage{listings}
\usepackage{xcolor}
\lstset { %
    language=C++,
    backgroundcolor=\color{black!5}, % set backgroundcolor
    basicstyle=\footnotesize,% basic font setting
}

\usepackage{amsfonts}
\usepackage{amsmath}
\usepackage{graphicx}
\usepackage{hyperref}
\usepackage{caption}
\usepackage{fancyhdr}
\usepackage{geometry}
	\geometry{height=24 cm}
	\geometry{left=2.5 cm}
	\geometry{right=2.5 cm}
	\geometry{top=2 cm}
	\geometry{headheight=1 cm}

\setcounter{secnumdepth}{2}


\title{\vspace{2cm}\textbf{Appunti di Programmazione\ a Oggetti}}
\author{\vspace{3mm}4 ottobre 2022}
\date{\vspace{3mm} \textbf{Rosso Carlo}}

\begin{document}

\begin{titlepage}
	\maketitle
	\thispagestyle{empty}
\end{titlepage}
\tableofcontents
\newpage

\section{Introduzione}

Un linguaggio di programmazione si dice ad oggetti se gode di tre proprietà:

\begin{itemize}
    \item incapsulamento;
    \item ereditarietà;
    \item polimorfismo.
\end{itemize}

L'incapsulamento è l'inaccessibilità dell'implementazione di una classe o di una
funzione. Questa proprità permette di rendere il codice più leggibile e più
mantenibile. Le classi per esempio sono implementazioni del tipo di dato
astratto. Con l'incapsulamento la definizione di classe viene nascosta. In
questo modo si opera con un sistema a scatola nera: non sai come i dati sono
rappresentati, pensi solo a come utilizzare la tecnologia, senza curarti del
perchè funzioni in questo modo. NB Nel corso di programmazione ad oggetti
impariamo come nascondere l'implementazione, in modo che altri utilizzino
l'implementazione in modo chiaro senza conoscerla.\\

Il c++ offre una programmazione generica: proprio come in rust, ci sono classi e
funzioni generiche; inoltre offre la gestione degli errori (non abbiamo ancora
chiarito bene).

\subsection{Operatori} 

Il c++ fornisce molti operatori, circa 50. Un operatore è una fuzione che viene
chiamata attraverso dei simboli o attraverso la ripetizione di simboli.

Per esempio, l'operatore "<<", oppure l'operatore "::", scoping.

\subsection{namespace}

Una condizione frequente nella stesura di programmi lunghi riguarda
l'inquinamento dello spazio del nome. Il nome delle variabili deve essere
significativo e preferibilmente in inglese.

%le struct sono sequenze di field
Il namespace è l'equivalente del modulo su rust. Permette di raggruppare
(recintare) le keyword in modo che siano accessibili solo attraverso l'operatore
di scoping, oppure attraverso le keyword "using namespace NomeSpazio".

\begin{lstlisting}
namespace SpazioUno {
    namespace f {
        g()
    }
    ...
}
\end{lstlisting}

Il namespace viene dichiarato come nell'esempio.\\
Nel momento in cui vari namespace sono contenuti uno dentro l'altro e noi
desideriamo accedere solo ad un namespace particolare è utile ricorrere alla
tecnica dell'alias:

\begin{lstlisting} 
namespace Uno = SpazioUno;
namespace ...
\end{lstlisting}

Questa tecnica si utilizza nel file in cui utilizziamo le classi o le funzioni
contenute in un determinato namespace.\\
Se intendiamo utilizzare tutte le funzioni in un namespace allora possiamo
evitare di indicare ogni volta il namespace a cui desideriamo accedere:

\begin{lstlisting}
// using namespace SpazioUno; // deprecato
// using namespace SpazioUno::f; // per includere l'intero namespace f
using namespace SpazioUno::f::g; // forma preferibile
\end{lstlisting}

Tra le due righe di codice riportate, la prima è deprecata, perchè si rischia di
inquinare lo spazio dei nomi attuale.
Inoltre, se si utilizzano solo alcune funzioni in un namespace è preferibile
incorporare solo quelle funzioni, per leggibilità e per evitare di inquinare lo
spazio dei nomi.

\subsection{ADT}

L'abstract data type è alla base dell'incapsulamento: viene nascosta
l'implementazione ed è visibile solo l'interfaccia, ovvero i metodi pubblici e
le operazioni proprie.

% esiste la libreria string, che permette di lavorare con le stringhe, ovvero
% con gli array di caratteri. La definizione di un oggetto di tipo string è 
% "typedef basic_string<char> string;" // in questo caso viene utizzato un
% template.

\subsection{Classe}
Il c++ fornisce l'astrazione classe. L'astrazione classe è l'astrazione di un
tipo, si tratta dell'implementazione dell'ADT che viene fornita in c++.
Prendiamo per esempio una classe che rappresenta il tempo, se scegliessimo di
divere l'ora in ore, minuti, secondi, ci ritroveremmo ad utilizzare 12 byte: un
byte per ciascuno di questi campi. Un sistema più conveniente consiste nel
contare i secondi passati dalla mezzanotte, in questo modo manteniamo tutte le
informazioni necessarie per conoscere l'ora, ma occupiamo solo un byte.\\
Implementiamo la classe orario:

\begin{lstlisting}
\label{classe_orario}
classe orario {
    private:
        int sec; // il campo che indica i secondi dalla mezzanotte

    public:
        orario();
        orario(int, int);
        orario(int, int, int);
        int Ore();
        int Minuti();
        int Secondi();
};
\end{lstlisting}
Una classe definisce un proprio namespace!\\
Il this è l'equivalente di self in rust. Il this è un handel all'oggetto che
viene modificato in un metodo. Poniamo il caso in cui un metodo ritorni
un'oggetto della classe, una soluzione è la seguente:
\begin{lstlisting}
classe A {
    public:
        A f() {
            return this;
        }
};
\end{lstlisting}
Il this serve per applicare una funzione all'oggetto di invocazione.

\subsubsection{In line}
Le definizioni delle funzioni possono essere fatte inline:
\begin{lstlisting}
classe orario {
    ...
    Ore() {
        return sec / (60 * 60);
    }
    ...
};
\end{lstlisting}

Se la definizione è fatta inline allora quando è chiamato il metodo (una
funzione relativa ad una classe), il corpo della funzione è inserito al posto
della funzione. Come lato negativo questo comporta un codice molto più pesante e
lungo (da evitare).

\subsubsection{Costruttore}
Un costruttore è una funzione che per nome ha il nome della classe di cui è il
costruttore. Il costruttore senza parametri è chiamato default contructor;
se non si specifica il costruttore di default allora viene generato un
costruttore standard che per generare un nuovo oggetto chiama i costruttori di
default di ciascun campo della classe.
Come vediamo nell'esempio \ref{classe_orario}, facciamo un overloading del
costruttore in modo da avere costruttori diversi in base ai dati che abbiamo
disponibili, senza dover inquinare lo spazio dei nomi.

\subsection{information hiding}
Information hiding consiste nell'evitare dipendente del codice per qualcosa che
potrebbe cambiare. In questo modo l'implementazione diventa indipendente
dall'utilizzo di una classe, per cui è più semplice cambiare una classe senza
dover riscrivere il codice che utilizza quella classe.\\
L'information hiding è permesso dai specificatori di accesso: public e private.
Tutti i metodi e i campi presenti in private sono accedibili solo da metodi
della classe. Mentre qualunque campo o metodo in public è accedibile.\\
In genere la documentazione descrive il funzionamento dell'interfaccia pubblica,
mentre l'interfaccia privata viene tralasciata per evitare dipendenze dal codice
che potrebbero essere modificato.

\subsection{The rule of three}
Se definiamo uno dei seguenti, allora dovremmo definirli tutti e tre:
\begin{itemize}
    \item distruttore;
    \item costruttore di copia;
    \item operatore di assegnazione di copia;
\end{itemize}
Questa regola è anche chiamata la regola del pollice (rule of thumb).

\subsection{Array}
Gli array hanno dimensione statica: il valore deve essere conosciuto a
compilazione.\\

Nella costruzione di un array i valori sono assegnati mediante costruttori di
copia:
\begin{lstlisting}
class C {
    public:
        int i;
        C(int x=3): i(x) {}
        ~C() {std::cout << i << "~C ";}
};

int main() {
    C a[4] = {C(1), C(), C(8)};
    // distruzione oggetti temporaneri, stampa: 8~C 3~C 1~C
    std::cout << a[0].i << a[1].i << a[2].i << a[3].i << std::endl;
    //stampa: 1383
}
//a all'uscita stampa 3~C 8~C 3~C 1~C
\end{lstlisting}

Con il file "*.h" l'informazione da nascondere (come sono implementate le
classi) viene mostrato: è un rischio perchè ti espone ad attacchi esterni.
Soluzione:
\begin{lstlisting}
// file "C_handle.h"
class C_handle {
    public:
        // parte pubblica
    private:
        class C_privata;    // dichiarazione incompleta
        C_privata* punt;    // solo punt. e ref. per dich. incompleta
};
\end{lstlisting}
\begin{lstlisting}
// file "C_handle.cpp"
class C_handle::C_privata {
    ...
};
\end{lstlisting}

\subsection{Friend function}
Una funzione amica (si utilizza la keyword "friend") può eccedere ai campi
privati e protetti di una classe diversa. Rende qualche funzione dipendente
dalla classe (rischia di diventare un problema di dipendenza).
    


\end{document}
