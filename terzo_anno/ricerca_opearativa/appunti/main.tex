\documentclass[12pt]{article}
\newcommand{\template}{../../../template}
\usepackage{\template/packages}

\newcommand{\titolo}{Appunti di Ricerca Operativa}
\newcommand{\autore}{Rosso Carlo}
\newcommand{\data}{A.A. 2023/2024}
\newcommand{\corso}{Ricerca Operativa}

\input{\template/copertina}

\begin{document}
\copertina
\tableofcontents
\newpage

\section{Introduzione}

\subsection{Modelli di programmazione lineare}

Un modello di programmazione lineare matematico è composto dai seguenti
elementi:
\begin{itemize}
	\item \textbf{Insiemi}: gli insiemi in cui sono contenuti gli altri elementi
	      del sistema;

	\item \textbf{Parametri}: i dati del problema;

	\item \textbf{Variabili decisionali}: le incognite del problema;

	\item \textbf{Vincoli}: le relazioni tra le variabili decisionali;

	\item \textbf{Funzione obiettivo}: la funzione che si vuole ottimizzare, ovvero
	      massimizzare o minimizzare.
\end{itemize}

Notiamo che i modelli di programmazione lineare sono dei particolari modelli di
programmazione matematica, in cui:
\begin{itemize}
	\item la funzione obiettivo è un'espressione lineare delle variabili
	      decisionali;

	\item i vincoli sono determinati da un insieme di equazioni o disequazioni
	      lineari.
\end{itemize}

\subsection{Costruzione di un modello di programmazione lineare}

Di seguito riportiamo i passi per la costruzione di un modello di
programmazione:

\begin{enumerate}
	\item \textbf{Definire le variabili decisionali}: ovvero le incognite del
	      problema. Per ogni variabile decisionale è necessario definire il suo
	      dominio, ovvero l'insieme dei valori che può assumere.

	\item \textbf{Formulare la funzione obiettivo}: ovvero la funzione che si
	      vuole ottimizzare, ovvero massimizzare o minimizzare. La funzione
	      obiettivo è un'espressione lineare delle variabili decisionali.

	\item \textbf{Formulare i vincoli}: ovvero le relazioni tra le variabili
	      decisionali. I vincoli sono determinati da un insieme di equazioni o
	      disequazioni lineari.
\end{enumerate}

\end{document}
