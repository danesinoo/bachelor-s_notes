\documentclass[12pt]{article}
\newcommand{\template}{../../../template}
\usepackage{\template/packages}

\newcommand{\titolo}{Appunti di Ricerca Operativa}
\newcommand{\autore}{Rosso Carlo}
\newcommand{\data}{A.A. 2023/2024}
\newcommand{\corso}{Ricerca Operativa}

\newcommand{\copertina}{
	\begin{titlepage}
		\begin{center}
			\vspace*{3cm}
			\LARGE{\textbf{\titolo}}\\
			\vspace{0.33cm}
			\Large{\textbf{\data}}\\
			\vspace{0.33cm}
			\Large{\autore}\\
		\end{center}
		\thispagestyle{empty}
	\end{titlepage}
}


\begin{document}
\copertina
\tableofcontents
\newpage

\section{Introduzione}

\subsection{Modelli di programmazione lineare}

Un modello di programmazione lineare matematico è composto dai seguenti
elementi:
\begin{itemize}
	\item \textbf{Insiemi}: gli insiemi in cui sono contenuti gli altri elementi
	      del sistema;

	\item \textbf{Parametri}: i dati del problema;

	\item \textbf{Variabili decisionali}: le incognite del problema;

	\item \textbf{Vincoli}: le relazioni tra le variabili decisionali;

	\item \textbf{Funzione obiettivo}: la funzione che si vuole ottimizzare, ovvero
	      massimizzare o minimizzare.
\end{itemize}

Notiamo che i modelli di programmazione lineare sono dei particolari modelli di
programmazione matematica, in cui:
\begin{itemize}
	\item la funzione obiettivo è un'espressione lineare delle variabili
	      decisionali;

	\item i vincoli sono determinati da un insieme di equazioni o disequazioni
	      lineari.
\end{itemize}

\subsection{Costruzione di un modello di programmazione lineare}

Di seguito riportiamo i passi per la costruzione di un modello di
programmazione:

\begin{enumerate}
	\item \textbf{Definire le variabili decisionali}: ovvero le incognite del
	      problema. Per ogni variabile decisionale è necessario definire il suo
	      dominio, ovvero l'insieme dei valori che può assumere.

	\item \textbf{Formulare la funzione obiettivo}: ovvero la funzione che si
	      vuole ottimizzare, ovvero massimizzare o minimizzare. La funzione
	      obiettivo è un'espressione lineare delle variabili decisionali.

	\item \textbf{Formulare i vincoli}: ovvero le relazioni tra le variabili
	      decisionali. I vincoli sono determinati da un insieme di equazioni o
	      disequazioni lineari.
\end{enumerate}

\subsection{min-max/max-min}

Scriviamo $\min\max \{e_1, e_2, \dots, e_n\}$ nel seguente modo:
\begin{align*}
	\min{y}    & \\
	y \leq e_1 & \\
	y \leq e_2 & \\
	\vdots     & \\
	y \leq e_n &
\end{align*}

Si lascia al lettore come scrivere $\max\min \{e_1, e_2, \dots, e_n\}$.

\subsection{Vincoli logici}

\begin{table}[H]
	\centering
	\begin{tabular}{l|c|c|c}
		  & \textbf{Formulazione non lineare} & \textbf{Formulazione lineare} & \textbf{Dominio}                           \\
		\hline
		1 & if $y_1 > 0$ then $y_2 = 1$       & $x \leq My$                   & $y_1 \in \mathbb{R | N}, y_2 \in \{0, 1\}$ \\
		2 & $y_1 = 1$ or $y_2 = 1$            & $y_1 + y_2 \geq 1$            & $y_1, y_2 \in \{0, 1\}$                    \\
		3 & $y_1 = 1$ and $y_2 = 1$           & $y_1 + y_2 = 2$               & $y_1, y_2 \in \{0, 1\}$                    \\
		4 & $y_1 = 1$ only if $y_2 = 1$       & $y_1 \leq y_2$                & $y_1, y_2 \in \{0, 1\}$                    \\
		5 & if $y_1 = 1$ then $y_2 = 1$       & idem                          & idem                                       \\
		6 & $y_1 = 1$ only if $y_2 = 0$       & $y_1 \leq 1 - y_2$            & $y_1, y_2 \in \{0, 1\}$                    \\
		7 & $y_1 = 1$ xor $y_2 = 1$           & $y_1 + y_2 = 1$               & $y_1, y_2 \in \{0, 1\}$                    \\
	\end{tabular}
\end{table}

Di seguito riporto la spiegazione di alcuni vincoli logici:
\begin{table}[H]
	\centering
	\begin{tabular}{l|c|c|c|c}
		  & $y_1 \geq 1$ & $y_1 = 0$          & $y_2 = 1$              & $y_2 = 0$          \\
		\hline
		1 & $y_2 = 1$    & $y_2 \in \{0, 1\}$ & $x \in \mathbb{R | N}$ & $x = 0$            \\
		4 & $y_2 = 1$    & $y_2 \in \{0, 1\}$ & $y_1 \in \{0, 1\}$     & $y_1 = 0$          \\
		6 & $y_2 = 0$    & $y_2 \in \{0, 1\}$ & $y_1 = 0$              & $y_1 \in \{0, 1\}$ \\
	\end{tabular}
\end{table}

\subsection{Problemi di programmazione lineare}

Un problema di ottimizzazione lineare è definito dalla minimizzazione o
massimizzazione di una funzione obiettivo soggetta a vincoli lineari. In termini
matematici scriviamo:
\begin{align*}
	\min \ (\text{oppure} \max) \quad f( & x)                                   \\
	\text{s.t. } g_i(x)                  & = b_i, \quad i = 1, \dots, k         \\
	g_i(x)                               & \leq b_i, \quad i = k + 1, \dots, k' \\
	g_i(x)                               & \geq b_i, \quad i = k' + 1, \dots, m \\
	x                    \in             & \ \mathbb{R}^n
\end{align*}

dove
\begin{itemize}
	\item $\begin{aligned}[t]
			      \mathbf{x} & = \begin{bmatrix}
				                     x_1 \\
				                     x_2 \\
				                     x_3
			                     \end{bmatrix}
		      \end{aligned}$
	      sono le variabili decisionali;
	\item $f, g_i$ sono funzioni lineari del tipo $\mathbb{R}^n \rightarrow
		      \mathbb{R}$;
	\item $b_i \in \mathbb{R}$;
\end{itemize}

Si noti che si possono scomporre $f$ e $g_i$ per rappresentarle come prodotto
tra un vettore di coefficienti e un vettore di variabili decisionali. Indichiamo
con $c$ il vettore di coefficienti della funzione obiettivo e con $a_i$ il
vettore di coefficienti del vincolo $i$-esimo. Allora $f = c^T x$ e $g_i = a_i^T
	x$.

\subsection{Soluzioni di un problema}

Una soluzione si dice ammissibile se il vettore $x$ soddisfa tutti i vincoli.
L'insieme di tutte le soluzioni ammissibili si dice
\textbf{insieme ammissibile}.
L'insieme ammissibile può essere:
\begin{itemize}
	\item \textbf{vuoto}: ovvero non esistono soluzioni;
	\item \textbf{limitato}: ovvero esiste una soluzione ottima;
	\item \textbf{illimitato}: ovvero i vincoli non limitano la soluzione.
\end{itemize}

Risolvere un problema di programmazione lineare significa riconoscere uno dei
tre casi precedenti, nel caso limitato trovare la soluzione ottima ed il
corrispondente valore della funzione obiettivo.

\subsection{Problemi di programmazione lineare in forma standard}

Un problema di programmazione lineare in forma standard è rispetta le seguenti
condizioni:
\begin{itemize}
	\item la funzione obiettivo è di minimo e senza costanti additive o
	      moltiplicative;
	\item tutte le variabili sono positive o nulle;
	\item i vincoli sono tutti di uguaglianza.
	\item il vettore dei termini noti $b$ è non negativo, ovvero $b_i \geq 0
		      \forall i \in {1, m}$.
\end{itemize}

\subsection{Trovare una soluzione ammissibile}

Un problema PL può essere scritto come $\min{c^T x: Ax=b, x \geq 0}$, dove $A$
è la matrice dei coefficienti dei vincoli, $b$ è il vettore dei termini noti e
$x$ è il vettore delle variabili decisionali. Si noti che il PL considerato si
trova in forma standart ($\min$, vincoli di uguaglianza e $x \geq 0$).\\
Ad ogni base della matrice $A$ corrisponde una soluzione ammissibile. Infatti,
sia $B$ una base di $A$ allora $x_B = B^{-1} b$ è una soluzione ammissibile. Se
utilizziamo la geometria, allora ci immaginiamo che $A$ rappresenti una poliedro
dell'iperspazio $\mathbb{R}^n$ e $x$ rappresenta un vertice del poliedro.\\
Notiamo che se il PL ammette una soluzione ottima, allora esiste una base
ottima. Ovvero la soluzione ottima è un vertice del poliedro.
Notiamo inoltre che dato un problema PL, e la rispettiva matrice che lo
rappresenta $[A|b]$, il numero di soluzioni ammissibili di base è limitato
superiormente da $\binom{n}{m}$, ovvero $\frac{n!}{m!(n-m)!}$, dove $m$ sono le
colonne di $A$ che possono formare una base, scelte tra le $n$ colonne totali.

\subsection{Metodo del simplesso}

Il metodo del simplesso, ad alto livello, funziona nel seguente modo:
\begin{itemize}
	\item si ha una base ammissibile $B$;
	\item si calcola il costo ridotto di ogni variabile non base $x_j$;
	\item se tutti i costi ridotti sono non negativi, allora la soluzione
	      corrente è ottima;
	\item altrimenti, si sceglie una variabile $x_j$ con costo ridotto
	      negativo e si calcola il rapporto tra il termine noto e l'elemento
	      corrispondente della colonna $j$-esima della matrice $B^{-1}$;
	\item si sceglie la variabile $x_k$ che minimizza il rapporto precedente;
	\item si effettua un pivotaggio per rendere $x_k$ una variabile base e
	      $x_j$ una variabile non base;
	\item si ripete il procedimento.
\end{itemize}

Tornando alla rappresentazione geometrica, il metodo del simplesso parte da un
vertice del poliedro e si sposta verso vertici adiacenti che migliorano il
valore della funzione obiettivo. Il metodo termina quando non esistono vertici
adiacenti che migliorano il valore della funzione obiettivo. Si noti che ci sono
dei casi in cui il metodo del simplesso non termina, oppure termina ma non
trova la soluzione ottima. Si pensi ad un poliedro con una faccia ammissibile e
sconnessa da un'altra faccia ammissibile con soluzione ottima. Se la prima base
corrisponde ad un vertice della prima faccia ammissibile, allora il metodo del
simplesso non trova la soluzione ottima e bisogna passare per qualche vertice
che non migliora il valore della funzione obiettivo, per poi arrivare ad un
vertice della seconda faccia ammissibile. Da questo punto, il metodo del
simplesso trova la soluzione ottima.\\

Se una variabile fuori base $x_j$ ha costo ridotto negativo, allora esiste un
base adiacente, tale che migliora la soluzione corrente. In particolare, se
$x_j$ non ha bisogno di rispettare alcun vincolo, allora il problema è
illimitato e non esiste una soluzione ottima.\\

\subsection{Il tableau del simplesso}

Il tableau del simplesso è una rappresentazione tabellare del problema di
programmazione lineare. Di seguito riportiamo un esempio di tableau del
simplesso:

\[
	\begin{array}{|ccccc|c|c|}
		\hline
		\multicolumn{5}{|c|}{c^T}                 & -1     & 0             \\
		\hline
		                                          &        &   &  &  & 0 & \\
		\multicolumn{5}{|c|}{\quad A \quad \quad} & \vdots & b             \\
		                                          &        &   &  &  & 0 & \\
		\hline
	\end{array}
\]

A questo punto si applicano le operazioni di pivotaggio per trovare una
soluzione di base ammissibile. Per indenderci si applicano le operazioni sulle
righe (si possono anche scambiare le colonne). Fino ad arrivare alla seguente
forma:

\[
	\begin{array}{|ccc|ccc|c|c|}
		\hline
		0 & \cdots & 0 & \bar{c}_{F_1}  & \cdots & \bar{c}_{F_{n-m}}  & -1     & -\bar{z}_B \\
		\hline
		  &        &   & \bar{a}_{1F_1} & \cdots & \bar{a}_{1F_{n-m}} & 0      &            \\
		  & I      &   & \vdots         & \ddots & \vdots             & \vdots & \bar{b}    \\
		  &        &   & \bar{a}_{mF_1} & \cdots & \bar{a}_{mF_{n-m}} & 0      &            \\
		\hline
	\end{array}
\]

Il tableau in questa forma è estremamente comodo perché contiene tutte le
informazioni necessarie per applicare il metodo del simplesso a portata di mano.
Leggiamo il tableau come segue:
\begin{itemize}
	\item $\bar{b}$ è il vettore che rappresenza la soluzione del problema
	      rispetto alla base corrente;

	\item $\bar{z}_B$ è il valore della funzione obiettivo rispetto alla
	      soluzione corrente, ovvero ponendo ciascuna variabile fuori base
	      uguale a $0$;

	\item $\bar{c}_{F_j}$ è il costo ridotto della variabile $x_{F_j}$ e la
	      colonna sottostante è il vettore $a_{F_j}$, che ci permette di capire
	      quale variabile esce dalla base;
\end{itemize}

Si noti che la penultima colonna è piuttosto inutile e può essere omessa.

\subsection{Cambio di base e operazioni di pivot}

Rispetto alla matrice qui sopra riportata, notiamo un paio di cose:
\begin{itemize}
	\item se $\bar{c} \geq 0$, allora la soluzione corrente è ottima;
	\item se $\exists \, \bar{c}_{F_j} < 0$ e la colonna sottostante ha
	      coefficienti minori o uguali a $0$, allora il problema è illimitato;
	\item risultao molto semplice calcolare il tableau di una base adiacente:
	      \begin{enumerate}
		      \item si sceglie una variabile $x_{F_j}$ con costo ridotto
		            negativo;

		      \item si calcolano i rapporti $\frac{\bar{b}_i}{\bar{a}_{iF_j}}$,
		            per ogni riga $i$ tale che $\bar{a}_{iF_j} > 0$;

		      \item si sceglie la riga $k$ che minimizza il rapporto;

		      \item si effettua un pivotaggio per rendere $x_{F_j}$ una variabile
		            base e $x_{B_k}$ una variabile non base.
	      \end{enumerate}
\end{itemize}

\section{Duale}

\subsection{Cambiamo punto di vista}

Un problema di programmazione lineare consiste nella determinazione del valore
ottimo $z^*$ di una funzione obiettivo $c^Tx$ su di un poliedro $P$. Perché
cambia il punto di vista? In effetti, possiamo vedere come il poliedro $P$, come
il dominio della funzione obiettivo $c^Tx$, ovvero l'insieme di punti in cui si
può cercare il valore ottimo. In questo caso, la funzione obiettivo riordina i
punti del poliedro $P$ secondo un certo criterio e ci assegna un valore in
$\mathbb{R}$. Questo vuol dire che la funzione obiettivo trasforma un poliedro
in una retta. La retta, può essere limitata o illimitata, oppure nulla. Il
problema di programmazione lineare cerca il valore massimo o minimo della retta
e quindi il limite superiore o inferiore dell'insieme di arrivo della funzione
obiettivo.

\subsection{Problema duale}

Sia $PL$ il problema di programmazione lineare:
\begin{align*}
	z^* = \min \quad c^T x &        \\
	s.t. \quad Ax          & = b    \\
	x                      & \geq 0
\end{align*}
Allora il problema duale $PD$ è:
\begin{align*}
	w^* = \max \quad u^T b &                  \\
	s.t. \quad u^T A       & \leq c^T         \\
	u                      & \in \mathbb{R}^m
\end{align*}

Dove $u^T = c^T B^{-1}$, dove $B$ è una base ammissibile del problema $PL_1$.\\

\subsection{Trasformazione al problema duale}

\begin{table}[H]
	\centering
	\begin{tabular}{cc|c}
		                                               & \textbf{Problema primale} ($\min{c^Tx}$) & \textbf{Problema duale} ($\max {u^Tb}$) \\
		\hline
		\multirow{3}{*}{Vincoli $\rightarrow$ Dominio} & $a^T_i x \geq b_i$                       & $u_i \geq 0$                            \\
		                                               & $a^T_i x \leq b_i$                       & $u_i \leq 0$                            \\
		                                               & $a^T_i x = b_i$                          & $u_i$ libero                            \\
		\hline
		\multirow{3}{*}{Dominio $\rightarrow$ Vincoli} & $x_j \geq 0$                             & $u^TA_j \leq c_j$                       \\
		                                               & $x_j \leq 0$                             & $u^TA_j \geq c_j$                       \\
		                                               & $x_j$ libero                             & $u^TA_j = c_j$                          \\
	\end{tabular}
\end{table}

Ovviamente, $PL$ ha una soluzione ottima se e solo se $PD$ ha una soluzione
ottima e inoltre $z^* = w^*$. Inoltre, se $PL$ è illimitato, allora $PD$ è
inammissibile. Se $DL$ è illimitato, allora $PL$ è inammissibile. Nota bene,
esistono casi in cui $PL$ e $PD$ sono entrambi inammissibili. E.g. PL:
\begin{align*}
	\min \quad & x_1                \\
	s.t. \quad & x_1 + x_2 & \geq 1 \\
	           & -x_1 -x_2 &        \\
	x_1, x_2 \text{libere}
\end{align*}

PD:
\begin{align*}
	\max \quad & u_1 + u_2       \\
	s.t. \quad & u_1 - u_2 & = 1 \\
	           & u_1 - u_2 & = 0 \\
	u_1, u_2 \geq 0
\end{align*}





\end{document}
