\documentclass[12pt]{article}
\newcommand{\template}{../../../template}
\usepackage{\template/packages}

\newcommand{\titolo}{Appunti di Tecnologie Web}
\newcommand{\autore}{Rosso Carlo}
\newcommand{\data}{A.A. 2023/2024}
\newcommand{\corso}{Tecnologie Web}

\input{\template/copertina}

\begin{document}
\copertina
\tableofcontents
\newpage

\section{css}
Procedimento a cascata:
\begin{enumerate}
	\item impostazioni personali dell'utente
	\item dichiarazioni defnite "!important"
	\item impostazioni di stile inline definite dall'autore della pagina
	\item fogli di stile \textit{embedded} definiti dall'autore
	\item fogli di stile esterni definiti dall'autore
	\item impostazioni di stile predefinite dal browser (le impostazioni di
	      default cambiano da browser a browser)
\end{enumerate}

Selettore di attributo:
\begin{lstlisting}{css}
	element_name[attribute=valore] { 
	/// base
	/* regole */
	}

	element_name[attribute] { 
	/// attributo presente
	/* regole */
	}

	element_name[attribute="valore"] { 
	/// attributo uguale alla stringa "valore"
	/* regole */
	}

	element_name[attributo~="valore"] { 
	/// attributo contenente la parola "valore"
	/* regole */
	}
	
\end{lstlisting}

\begin{definition}[ereditarietà]
	\begin{enumerate}
		\item flag \lstinline{attributo=valore !important}
		\item num id
		\item num attributi
		\item num tag html
	\end{enumerate}
\end{definition}

\subsection{pseudoclassi}
\begin{table}[H]
	\centering
	\begin{tabular}{|l|l|}
		\hline
		pseudoclasse & descrizione                     \\
		\hline
		:link        & link non visitato               \\
		:visited     & link visitato                   \\
		:active      & link attivo                     \\
		:hover       & mouse sopra                     \\
		:focus       & elemento attivo (tab)           \\
		:first       & prima pag per media paginati    \\
		:left        & pagine di sinistra              \\
		:right       & pagine di destra                \\
		:first-child & prima occorrenza di un elemento \\
		:lang        & seleziona una lingua            \\
		\hline
	\end{tabular}
	\caption{pseudoclassi}
\end{table}

Esempio:
\begin{lstlisting}
a:link:hover {
	font-size: 2em;
}

// generale
selettore::pseudoclasse {
	/* regole */
}
\end{lstlisting}

\subsection{pseudo-elementi}
\begin{table}[H]
	\centering
	\begin{tabular}{|l|l|}
		\hline
		pseudo-elemento & descrizione                \\
		\hline
		:first-letter   & prima lettera di un blocco \\
		:first-line     & prima riga di un blocco    \\
		:before         & testo da aggiungere prima  \\
		:after          & o dopo un elemento         \\
		\hline
	\end{tabular}
	\caption{pseudo-elementi}
\end{table}

\subsection{unità di misura}
\begin{table}[H]
	\centering
	\begin{tabular}{|l|l|l|}
		\hline
		unità & descrizione                                   & esempio                 \\
		\hline
		em    & altezza media del font, relativo              & h1 \{ margin:0.5em \}   \\
		px    & utilizzato, assoluto                          & p \{ font-size:12px \}  \\
		in    & numero pixel nello schermo, assoluto          & p \{ font-size:0.5in \} \\
		cm    & inch, pollici (1 in = 2,54 cm), assoluto      & p \{ font-size:1cm \}   \\
		mm    & millimetri, assoluto                          & p \{ font-size:1mm \}   \\
		pt    & punti (1 pt = 1/72 in), assoluto              & p \{ font-size:12pt \}  \\
		pc    & pica (1 pc = 12 pt), assoluto                 & p \{ font-size:1pc \}   \\
		\%    & percentuale, relativo all'elemento principale & p \{ font-size:120\% \} \\
		\hline
	\end{tabular}
	\caption{unità di misura}
\end{table}

\subsection{colori}
Espressi in formato rgb esadecimale funzionano su tutti i browser.
Esempio:
\begin{lstlisting}
p {
	/// color: #RRGGBB
	color: #ff0000;
}
\end{lstlisting}

\subsection{url}
Esempio:
\begin{lstlisting}
p {
	/// url: url(protocollo://dominio/percorso)
	background-image:url(percorso/immagine.jpg);
	background-repeat:repeat;
}
\end{lstlisting}

\subsection{stile del testo}
\begin{itemize}
	\item dimensione: \lstinline{font-size: value;}
	\item interlinea: \lstinline{line-heigh}
	\item sovvrapposizione: \lstinline{z-index}
	\item corsivo: \lstinline{font-style}
	\item grassetto: \lstinline{font-weight // bold, normal, bolder, lighter}
	\item variante maiuscoletto: \lstinline{font-variant // small-caps}
	\item maiuscolo o minuscolo: \lstinline{text-transform // uppercase, lowercase, capitalize}
	\item decorazione: \lstinline{text-decoration // underline, overline, line-through}
	\item colori: \lstinline{color, background-color}

	\item distanza tra le lettere: \lstinline{letter-spacing}
	\item distanza tra le parole: \lstinline{word-spacing}
	\item indentazione: \lstinline{text-indent}
	\item allineamento orizzontale: \lstinline{text-align // left, right, center, justify}
	\item allineamento verticale: \lstinline{vertical-aligh}
\end{itemize}

\subsection{stile dello sfondo}
\begin{itemize}
	\item \lstinline{background-attachment}: stabilisce se l'immagine segue il
	      contenuto nello scroll oppure no
	\item \lstinline{background-repeat}
	\item \lstinline{background-position}
	\item \lstinline{background-image: url(/* path */)}
\end{itemize}

\section{Domande e risposte}

\subsection{HTML}

\subsubsection{Le tabelle devono sempre essere evitate?}

\textbf{Falso}, le tabelle non devono essere usate per presentare e organizzare
i contenuti. Le tabelle sono pesanti per il browser da renderizzare. Questo a
parte, se bisogna visualizzare dei dati in una struttura tabellare allora è bene
usare le tabelle, rendendole accessibili, ovvero usando
\lstinline{thead, tbody, tfoot, th, td}, è consigliato usare anche
\lstinline{aria-describedby, scope, lang, abbr} per permettere agli
\textit{screen reader} una lettura più agevole della tabella per gli utenti che
li usano.

\subsubsection{Le tabelle sono un ostacolo alla facile comprensione per le
	perosne con disabilità, quindi andrebbero evitate dove possibile.}

\textbf{Vero}, le tabelle non devono essere usati per presentare e organizzare i
contenuti. Inoltre sono complesse da renderizzare per i browser. Sarebbe meglio
evitarle, a meno che i dati da visualizzare non si prestino alla forma
tabellare. In ogni caso, se si intende adottarle è importante strutturarle
rigorosamente in modo accessibile.

\subsubsection{Elencare gli accorgimenti che si devono adottare per rendere una
	tabella accessibile.}

Di seguito sono riportati i passi da seguire per rendere accessibile una
tabella:

\begin{enumerate}
	\item Riportare una breve descrizione della tabella. Il tag della
	      descrizione deve avere un id, che viene associato alla tabella mediante
	      l'attributo \lstinline{aria-describedby};

	\item Utilizzare i tag \lstinline{thead, tbody, tfoot} per dividere
	      semanticamente le informazioni. Infatti, per esempio, html rende sempre
	      disponibile il \lstinline{tfoot};

	\item Utilizzare l'attributo \lstinline{lang} per segnalare agli
	      \textit{screen reader} le parole che differiscono dalla lingua di
	      default della pagina;

	\item Utilizzare \lstinline{th} per le celle che descrivono i dati in una
	      riga o una colonna. Oltre a \lstinline{th}, bisogna utilizzare anche
	      \lstinline{scope}, per indicare se quell'informazione vale per la riga
	      o per la colonna. Si consiglia anche l'adozione di \lstinline{abbr}
	      per consigliare ad uno screen reader l'adozione di un'abbreviazione
	      piuttosto che il titoletto della riga o della colonna completo;

	\item Utilizzare il tag \lstinline{caption} per scrivere la didascalia
	      della tabella.
\end{enumerate}

Inoltre si consiglia di alternare i colori delle righe o delle colonne per le
tabelle molto lunghe o molto larghe, prenstando attenzione al contrasto dei
colori.

\subsubsection{Esempio di tabella accessibile}

\begin{htmlcode}
	<table aria-describedby="table_occupazione_palestra">
	<caption>Occupazione della palestra</caption>

	<thead>
	<tr>
	<th></th>
	<th scope="col" abbr="9 - 10">9.00-10.00</th>
	<th scope="col" abbr="10 - 11">10.00-11.00</th>
	<th scope="col" abbr="11 - 12">11.00-12.00</th>
	<th scope="col" abbr="12 - 15">12.00-15.00</th>
	<th scope="col" abbr-"15 - 16">15.00-16.00</th>
	</tr>
	<tbody>
	<tr>
	<th scope="row" abbr="lun">Lunedi</th>
	<td colspan="3">Hip Hop</td>
	<td></td>
	<td>Pilates</td>
	</tr>
	<tr>
	<th scope="row" abbr="mar">Martedi</th>
	<td colspan="2">Karate</td>
	<td>Hip Hop</td>
	<td></td>
	<td>Pilates</td>
	</tr>
	<tr>
	<th scope="row" abbr="mer">Mercoledi</th>
	<td></td>
	<td></td>
	<td>Hip Hop</td>
	<td rowspan="3">Ginnastica Artistica</td>
	<td></td>
	</tr>
	<tr>
	<th scope="row" abbr="gio">Giovedi</th>
	<td colspan="2">Karate</td>
	<td></td>
	<td></td>
	</tr>
	<tr>
	<th scope="row" abbr="ven">Venerdi</th>
	<td></td>
	<td></td>
	<td>Hip Hop</td>
	<td>Pilates</td>
	</tr>
	<tbody>
	<table>

	<span id="table_occupazione_palestra">Si tratta di una tabella che illustra
	quali corsi occupano la palestra durante la settimana dal lunedi al venerdi,
	dalle 9 di mattina fino alle 16 di sera.<span>
\end{htmlcode}

\subsection{CSS}

\subsubsection{Differenze fondamentali tra id e class}
\begin{itemize}
	\item id univoco per tag, class raggruppa i tag;
	\item un tag può avere al massimo un solo id, mentre può avere tante classi;
	\item l'id si può riferire nell'url come "link\_della\_pagina\#id\_elemento";
	\item in css il selettore dell'id è "\#", mentre per le classi è ".";
	\item l'id può iniziare solo con "\_" oppure con una lettera;
	\item si può usare \lstinline{.selectById(id)} per selezionare un elemento
	      tramite id in js;
	\item ha valenza massima nel calcolo della specificità, dopo
	      \lstinline{!important}.
\end{itemize}

\subsubsection{Ordine di applicazione degli stili in CSS}

\begin{enumerate}
	\item stili di default del browser;
	\item fogli di stile esterni (.css) definiti dall'autore del sito;
	\item fogli di stile embedded all'interno del documento;
	\item stile definito inline dall'autore, all'interno dei tag;
	\item se presente "!important";
	\item stile dell'utente.
\end{enumerate}

\subsubsection{Di che colore è il testo all'interno del tag?}

Per rispondere al quesito si procede calcolando la tupla di specificità per
ciascuna regola definita in css: (!important (se presente), id, attributi o
classi, tag e pseudoclassi). Le tuple sono riordinate dalla maggiore alla minore
e gli stili con specificità maggiore sovrascrivono gli stili con specificità
minore. In caso di tuple con la medesima specificità, viene applicata l'ultima.

\subsection{Web Design}

\subsubsection{Le convenzioni interne non devono essere rotte, per non
	disorientare l'utente}

\textbf{Vero}, le convenzioni interne non devono mai essere rotte, perché
aiutano l'utente a famialiarizzare con il sito e a prendere confidenza con esso.

\subsubsection{La struttura organizzativa deve essere poco ampio e molto
	profonda}

\textbf{Falso}. L'ampiezza massima consigliata è di 7 voci, comunque si
sconsiglia un'ampiezza superiore alle 10 voci. Mentre la profondità massima
consigliata è di 5 voci e fino a 7 voci nel caso si arrivi ad una specificità
elevata. Detto questo, la profondità corrisponde al numero di click che l'utente
è tenuto a fare per passare dalla homepage alla pagina. Se la struttura
organizzativa del sito è poco profonda, vuol dire che l'utente ha modo di
raggiungere la pagina che cerca con meno click e quindi più velocemente.
Inoltre, risulta più semplice manutenere e gestire nel tempo una gerarchi ampia
e poco profonda. Bisogna ricordare però che un numero di voci troppo elevato
rischia di causare il "sovraccarico cognitivo", perché l'utente ha bisogno di
memorizzare troppe informazioni nell'arco di poco tempo e rischia di
disorientare l'utente.

\subsubsection{Una buona regola è segnalare i link che portano a pagine in
	lingua diversa da quella del sito utilizzando le bandiere nazionali.}

\textbf{Falso}, perché la bandiera è collegata alla nazione e non alla lingua,
infatti ci sono stati in cui si parlano più lingue e ci sono stati diversi in
cui si parla la medesima lingua. Piuttosto, si consiglia di segnalare tali link
mendiante l'uso di ISO che segnalano la lingua.

\subsubsection{Il numero massimo di voci in un menù è 6.}
\textbf{Falso}, il numero di voci consigliato di un menù è 7, ma si
possono aggiungere ulteriori 3 voci. In ogni caso, bisogna tenere a mente il
rischio di causare il "sovraccarico cognitivo" all'utente, ovvero la condizione
in cui sono fornite troppe informazioni all'utente, e questo lo porta a
confondersi o a perdersi nel sito.

\subsubsection{Una struttura organizzativa gerarchica deve essere ampia e poco
	profonda.}

\textbf{Vero}, infatti tale struttura organizzativa diminiusce il numero di
click che l'utente è tenuto ad eseguire per raggiungere la pagina che cerca.
Inoltre risulta più semplice gestire un sito con questo tipo di struttura
organizzativa.

\subsubsection{Il layout a schede può andare incontro a problemi di
	manutenibilità nel tempo.}

\textbf{Vero}:
\begin{itemize}
	\item Con l'aumento delle schede diventa difficile per gli utente trovare i
	      contenuti che stanno cercando;
	\item Quando qualche contenuto viene aggiunto, rimosso o modificato, è
	      necessario aggiornare il layout a schede per riflettere i cambiamenti.
	      Ciò spesso comporta l'aggiornamento delle etichette e del contenuto
	      delle singole schede;
	\item Se non sono progettate tenendo conto dell'accessibilità, causano
	      problemi agli utenti con disabilità visive oppure potrebbero non
	      funzionare bene in schermi piccoli. Per questo motivo può diventare
	      molto complesso ristrutturare il layout del sito per renderlo più
	      accessibile.
\end{itemize}

Si ricorda inoltre che le regole dell'accessibilità sono continuamente in
aggiormento, per questo motivo è necessario ristrtutturare spesso un sito.

\subsubsection{La divisione tra contenuto e presentazione diminuisce il peso
	totale di un sito web}

\textbf{Vero}, infatti dividendo il contenuto dalla presentazione permette di
ottimizzare l'html dal punto di vista della leggibilità e della chiarezza e il
css dal punto di vista del peso del file con strumenti \textit{ad hoc}.
Si consiglia di dividere sempre la struttura, dalla presentazione e dal
comportamento, in modo tale da poter ottimizzare singoli moduli di ciascuna
categoria con strumenti opportuni, tipo minimal per js.

\subsubsection{La divisione tra contentuto e layout influenza il posizionamento
	di una pagina web nelle pagine di risposta dei motori di ricerca}

\textbf{Vero}, infatti i motori di ricerca raccolgono in modo automatico il
contenuto della pagina in modo tale da indicizzare le pagine. Tenere le pagine
leggere aiuta molto sul posizionamento del sito.

\subsubsection{Dare una definizione di linguaggio markup, descriverne le
	principali caratteristiche e fornire alcuni esempi di linguaggi di markup
	conosciuti (evidenziando in modo opportuno le differenze)}

Un linguaggio markup è un linguaggio formato da un insieme di regole che danno
la possibilità di esplicitare la struttura o la presentazione di un dato
strutturato. Attraverso l'uso dei tag è possibile assegnare una valenza
semantica o rappresentativa di un'informazione. I linguaggi markup si dividono
in due categorie:
\begin{itemize}
	\item \textbf{di struttura}, per esempio html, in cui il contenuto della
	      pagina è racchiuso nei tag che associano significato semantico;

	\item \textbf{di presentazione}, per esempio css, tendenzialmente un foglio
	      di stile viene collegato ad una o più pagine html. Il foglio di stile
	      permette di specificare le regole con cui presentare ciascun tag.
	      Infatti le regole di presentazione sono associate ad un selettore che
	      può essere più o meno specifico, in modo tale da permettere un controllo
	      della presentazione delle informazioni sia molto fine che granulare.
\end{itemize}

Si consiglia di dividere i due linguaggi i file differenti, perché questo
migliora l'indice che i motori di ricerca assoceranno alla pagina. Ma
soprattutto, dal punto di vista della manutenibilità è buona norma tenere la
logica di struttura e quella di presentazione ben divise, in modo tale da
rendere la logica di struttura e quella di presentazione indipendenti e più
facilmente modificabili.


\subsubsection{Dare una breve descrizione del Document Object Model (DOM). In
	particolare si indichi come questo viene rappresentato e quali sono le sue
	funzioni principali.}

Il DOM è uno standard W3C che definisce un modello per l'accesso dinamico al
contenuto di un documento, consentendo la visualizzazione e l'aggiornamento di
contenuto, presentazione e struttura.
Il DOM di una pagina è rappresentato da un albero in cui ciascun nodo
rappresenta un tag che ne descrive le proprietà associate. La root è il DOM
stesso, i suoi figli sono l'insieme dei tag che contiene direttamente. Ciascun
nodo contiene dei figli se e solo se il tag che gli corrisponde contiene altri
tag. Il DOM viene utilizzato in particolare per controllare e gestire il
comportamento di qualche elemento di una pagina web. Il DOM fornisce
un'interfaccia neutrale per gestire la presentazione di una pagina web. In
questo modo possono essere definiti diversi linguaggi di programmazione per
rendere un sito dinamico. Il linguaggio maggiormente adottato a questo scopo è
javascript.

\subsubsection{Descrivere la differenza tra linguaggi di tipo server side e
	client side}

Non c'è una vera differenza tra i due linguaggi, infatti molto spesso un
linguaggio può essere usato sia per gestire il lato server che per il lato
client. Consideriamo per esempio javascript, il linguaggio più usato per gestire
il lato client, esiste node.js, un engine che permette di eseguire javascript
per adottarlo lato server. Tendenzialmente entrambi i linguaggi sono turing
completi. Detto questo ci sono alcune differenze rispetto allo scopo di un
programma che viene eseguito lato server o lato client.
Un programma lato server deve gestire le richieste che arrivano dai client del
servizio. Tendenzialmente è collegato ad un database per garantire la
persistenza dei dati e si occupa di mantenere i dati coerenti nella base di
dati. Alcuni linguaggi lato server sono golang, rust o python.
Un linguaggio lato client, invece, ha due funzioni: la prima è quella di rendere
la pagina dinamica, in modo tale da permettere un'interazione più agevole per
l'utente che utilizza il servizio; la seconda funzione è quella di comunicare al
al programma che è eseguito sul server le azioni dell'utente. Un linguaggio lato
client, si occupa fondamentalmente di visualizzare le informazioni che gli sono
date dal server ed esegue le richieste al server, mostrando all'utente l'esito
di ciascuna. I linguaggi lato client sono eseguiti dal browser. Alcuni linguaggi
lato client sono javascript e web asm.




\section{Riassunto}

\subsection{HTML}

\subsubsection{Tabelle}

\begin{itemize}
	\item si utilizzano solo sei i dati hanno struttura tabellare;
	\item NOn per il layout;
	\item applicando con rigore le regole per renderle accessbili:
	      \begin{enumerate}
		      \item \lstinline{aria-describedby};
		      \item \lstinline{caption};
		      \item dividere la tabella in \lstinline{thead, tbody, tfoot};
		      \item le celle \lstinline{th} devono avere anche
		            \lstinline{scope, abbr};
		      \item le celle in lingua diversa devono essere segnalate;
		            l'attributo \lstinline{lang};
		      \item \lstinline{colspan, rowspan}.
	      \end{enumerate}
\end{itemize}


\end{document}
