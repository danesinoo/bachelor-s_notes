\documentclass[11pt]{article}
 \renewcommand{\familydefault}{\sfdefault}

\usepackage{listings}
\usepackage{xcolor}
\lstset{
    language=c++, % Set the language to SQL
    basicstyle=\ttfamily\color{black}, % Regular font style
    commentstyle=\color{green!60!black}, % Set the color for comments
    keywordstyle=\bfseries\color{blue}, % Bold and color for keywords
    stringstyle=\color{orange}, % Set the color for strings
}

\newtheorem{definition}{Def.}[section]

\usepackage{listings}
\usepackage{tabularx}
\usepackage{changepage}
\usepackage{amsfonts}
\usepackage{float}
\usepackage{amsmath}
\usepackage{graphicx}
\usepackage{hyperref}
\usepackage{caption}
\usepackage{fancyhdr}
\usepackage{geometry}
	\geometry{height=24 cm}
	\geometry{left=2.5 cm}
	\geometry{right=2.5 cm}
	\geometry{top=2 cm}
	\geometry{headheight=1 cm}

\setcounter{secnumdepth}{2}
\renewcommand{\labelitemi}{-}

\title{\vspace{2cm}\textbf{Appunti di Tecnologie Web}}
\author{\vspace{3mm}4 ottobre 2022}
\date{\vspace{3mm} \textbf{Rosso Carlo}}

\begin{document}

\begin{titlepage}
	\maketitle
	\thispagestyle{empty}
\end{titlepage}
\tableofcontents
\newpage

\section{css}
Procedimento a cascata:
\begin{enumerate}
	\item impostazioni personali dell'utente
	\item dichiarazioni defnite "!important"
	\item impostazioni di stile inline definite dall'autore della pagina
	\item fogli di stile \textit{embedded} definiti dall'autore
	\item fogli di stile esterni definiti dall'autore
	\item impostazioni di stile predefinite dal browser (le impostazioni di 
	default cambiano da browser a browser)
\end{enumerate}

Selettore di attributo:
\begin{lstlisting}{css}
	element_name[attribute=valore] { 
	/// base
	/* regole */
	}

	element_name[attribute] { 
	/// attributo presente
	/* regole */
	}

	element_name[attribute="valore"] { 
	/// attributo uguale alla stringa "valore"
	/* regole */
	}

	element_name[attributo~="valore"] { 
	/// attributo contenente la parola "valore"
	/* regole */
	}
	
\end{lstlisting}

\begin{definition}[ereditarietà]
	\begin{enumerate}
		\item flag \lstinline{attributo=valore !important}
		\item num id
		\item num attributi
		\item num tag html
	\end{enumerate}
\end{definition}

\subsection{pseudoclassi}
\begin{table}[H]
	\centering
	\begin{tabular}{|l|l|}
		\hline
		pseudoclasse & descrizione \\
		\hline
		:link & link non visitato \\
		:visited & link visitato \\
		:active & link attivo \\
		:hover & mouse sopra \\
		:focus & elemento attivo (tab) \\
		:first & prima pag per media paginati \\
		:left & pagine di sinistra \\
		:right & pagine di destra \\
		:first-child & prima occorrenza di un elemento \\
		:lang & seleziona una lingua \\
		\hline
	\end{tabular}
	\caption{pseudoclassi}
\end{table}

Esempio:
\begin{lstlisting}
a:link:hover {
	font-size: 2em;
}

// generale
selettore::pseudoclasse {
	/* regole */
}
\end{lstlisting}

\subsection{pseudo-elementi}
\begin{table}[H]
	\centering
	\begin{tabular}{|l|l|}
		\hline
		pseudo-elemento & descrizione \\
		\hline
		:first-letter & prima lettera di un blocco \\
		:first-line & prima riga di un blocco \\
		:before & testo da aggiungere prima \\
		:after & o dopo un elemento \\
		\hline
	\end{tabular}
	\caption{pseudo-elementi}
\end{table}

\subsection{unità di misura}
\begin{table}[H]
	\centering
	\begin{tabular}{|l|l|l|}
		\hline
		unità & descrizione & esempio \\
		\hline
		em & altezza media del font, relativo & h1 \{ margin:0.5em \} \\
		px & utilizzato, assoluto & p \{ font-size:12px \} \\
		in & numero pixel nello schermo, assoluto & p \{ font-size:0.5in \} \\
		cm & inch, pollici (1 in = 2,54 cm), assoluto & p \{ font-size:1cm \} \\
		mm & millimetri, assoluto & p \{ font-size:1mm \} \\
		pt & punti (1 pt = 1/72 in), assoluto & p \{ font-size:12pt \} \\
		pc & pica (1 pc = 12 pt), assoluto & p \{ font-size:1pc \} \\
		\% & percentuale, relativo all'elemento principale & p \{ font-size:120\% \} \\
		\hline
	\end{tabular}
	\caption{unità di misura}
\end{table}

\subsection{colori}
Espressi in formato rgb esadecimale funzionano su tutti i browser.
Esempio:
\begin{lstlisting}
p {
	/// color: #RRGGBB
	color: #ff0000;
}
\end{lstlisting}

\subsection{url}
Esempio:
\begin{lstlisting}
p {
	/// url: url(protocollo://dominio/percorso)
	background-image:url(percorso/immagine.jpg);
	background-repeat:repeat;
}
\end{lstlisting}

\subsection{stile del testo}
\begin{itemize}
	\item dimensione: \lstinline{font-size: value;}
	\item interlinea: \lstinline{line-heigh}
	\item sovvrapposizione: \lstinline{z-index}
	\item corsivo: \lstinline{font-style}
	\item grassetto: \lstinline{font-weight // bold, normal, bolder, lighter}
	\item variante maiuscoletto: \lstinline{font-variant // small-caps}
	\item maiuscolo o minuscolo: \lstinline{text-transform // uppercase, lowercase, capitalize}
	\item decorazione: \lstinline{text-decoration // underline, overline, line-through}
	\item colori: \lstinline{color, background-color}
	
	\item distanza tra le lettere: \lstinline{letter-spacing}
	\item distanza tra le parole: \lstinline{word-spacing}
	\item indentazione: \lstinline{text-indent}
	\item allineamento orizzontale: \lstinline{text-align // left, right, center, justify}
	\item allineamento verticale: \lstinline{vertical-aligh}
\end{itemize}

\subsection{stile dello sfondo}
\begin{itemize}
	\item \lstinline{background-attachment}: stabilisce se l'immagine segue il
		contenuto nello scroll oppure no
	\item \lstinline{background-repeat}
	\item \lstinline{background-position}
	\item \lstinline{background-image: url(/* path */)}
\end{itemize}

\end{document}
