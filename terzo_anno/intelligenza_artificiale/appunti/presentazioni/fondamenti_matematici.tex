\section{Fondamenti matematici}

La scienza cognitiva 
trae vantaggio dall'utilizzo della modellazione computazionale perché 
è uno strumento preciso e sistematico. 
In particolare, un avanzamento più rapido della conoscenza scientifica è
solitamente supportato dall'adozione di un linguaggio formale (matematico), dove
i concetti sono definiti nel modo più esplicito e preciso possibile.\\
L'idea fondamentale alla base degli algoritmi di apprendimento automatico
(machine learning) è di fare in modo che l'algoritmo stesso possa modificare la
propria struttura con l'esperienza. Per esempio, se l'output dell'algoritmo
dipende da alcune variabili memorizzate, possiamo programmare l'algoritmo
affinché modifichi autonomamente tali variabili a seconda degli input che gli
vengono mostrati. In altre parole, il comportamento dell'algoritmo dipende dal
valore di alcuni suoi stati interni, e l'algoritmo ha la capacità di modificare
in modo autonomo tali stati in base al tipo di esperienza che riceve.\\
Nel caso di una rete neurale artificiale, il sistema modifica la forza delle
connessioni sinaptiche in base alle informazioni rilevate in un particolare
insieme di esempi di addestramento (training set). 
