\section{Modelli computazionali}

\subsection{Modelli connessionisti localistici}

I modelli connessionisti (quindi basati su una rete neurale) localistici hanno
le seguenti caratteristiche:
\begin{itemize}
	\item non apprendono: le connessioni e i pesi sono fissati a priori;

	\item ogni nodo corrisopnde a una rappresentazione con un ruolo specifico;

	\item i livelli di rappresentazione sono decisi a priori;

	\item sono utilizzati per simulare i dati comportamentali.
\end{itemize}

\subsection{Modelli connessionisti PDP (Parallel Distributed Processing)}

I modelli connessionisti PDP hanno le seguenti caratteristiche:
\begin{itemize}
	\item sono utilizzati per simulare l'apprendimento di un'abilità;

	\item utilizzano algoritmi di apprendimento per reti neurali, in particolare
		si basano sulla backpropagation;

	\item le informazioni sono distribuite su più nodi, il che li rende più
		plausibili da un punto di vista biologico;

	\item livelli emergenti di rappresentazione.
\end{itemize}

Problemi dei modelli PDP fino al 2010:
\begin{itemize}
	\item architettura feed-forward;

	\item architettura superficiale (non più di uno strato nascosto, perché non
		avevamo computer abbastanza forti);

	\item apprendimento solo supervisionato, che non è biologicamente 
		plausibile.

	\item input non realistico e su piccola scala.
\end{itemize}

Queste caratteristiche non valgono per i modelli basati sul deep learning
generativo. Caratteristiche dell'appendimento generativo (non supervisionato):
\begin{itemize}
	\item apprendimento per osservazione: l'obiettivo è di costruire un modello
		interno dell'informazione sensoriale, quindi compattare l'informazione
		sensoriale;

	\item l'apprendimento avviene su dati grezzi;

	\item approccio probabilistico: sono trovati i parametri del modello che
		meglio descrivono le osservazioni, per massima verosimiglianza.

	\item elbaorazione gerarchica: ci sono molti strati di neuroni organizzati
		gerarchicamente;

	\item elaborazione ricorrente: ci sono sia connessioni feed-forward che
		feedback, ovvero ricorrenti.
\end{itemize}

In particolare ci sono i modelli DBN (Deep Belief Networks) che si ottengono
ponendo diversi RBM (Restricted Boltzmann Machine) una sopra l'altra.
