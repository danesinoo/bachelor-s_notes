\section{Dalla regola di Hebb ai modelli generativi}

\subsection{Reti di Hopfield}

Le reti di Hopfield possono essere utilizzate per memorizzare e recuperare
pattern (e niente altro).
La memorizzazione, come al solito, avviene modificando i pesi delle connessioni,
in particolare, applicando la regola di Hebb. Il recupero avviene in modo
dinamico, aggiornando iterativamente lo stato dei neuroni fino a che non si
raggiunge uno stato di equilibrio (attrattore).
Infatti, si definisce una funzione di energia che indica in quale modo
aggiornare lo stato dei neuroni. Ai minimi locali corrisponde uno stato di
equilibrio, per cui durante il recupero, la funzione di energia aggiorna lo
stato dei neuroni per diminuire l'energia del sistema e quindi per raggiungere 
uno stato di equilibrio.
Lo scopo dell'apprendimento è quello di modificare i pesi in modo tale che la
funzione di energia abbia minimi locali corrispondenti ai pattern memorizzati.
Quando si presenta un pattern parziale (in realtà per ogni input), la rete
assegna si assesta nella configurazione corrispondente al pattern
memorizzato più simile.
\paragraph{Architettura}: la rete è ricorrente, ovvero tutte le connessioni sono
bidirezionali con topologia completamente connessa, ovvero tutti i neuroni sono
collegati tra loro.
Infine, tutti i neuroni sono visibili; cioè lo strato di input coincide con lo
strato di output e non ci sono altri strati.

\paragraph{Funzione di energia}: la funzione di energia è definita come:
\begin{equation*}
	E = -\frac{1}{2} \sum_{i,j} w_{ij} x_i x_j
\end{equation*}

Gli attrattori della rete corrispondono ai punti di minimo locale della funzione
di energia, e rappresentano i pattern memorizzati.\\
Ci sono due metodi per minimizzare la funzione di energia:
\begin{itemize}
	\item \textbf{Recupero di un pattern}: sono aggiornate le attivazioni dei
		neuroni una alla volta, in modo sequenziale, fino a che non si raggiunge
		uno stato di equilibrio (è aggiornato $x_i$ o $x_j$);

	\item \textbf{Apprendimento}: sono aggiornati i pesi delle connessioni per
		creare minimi locali in corrispondenza dei pattern memorizzati (è 
		aggiornato $w_{ij}$).
\end{itemize}

\subsubsection{Recupero dei pattern}
I neuroni possono avere solo due stati possibili: -1 e 1 e la loro attivazione è
calcolata con la regola:
\begin{equation*}
	x_i = \begin{cases}
		1 & \text{se } \sum_{j} w_{ij} x_j > \theta_i\\
		-1 & \text{altrimenti}
	\end{cases}
\end{equation*}

\subsubsection{Apprendimento}

Come sono modificati i pesi delle connessioni per memorizzare i pattern? Si
utilizza la regola di Hebb, che in questo caso è:

\begin{equation*}
	\Delta w_{ij} = \eta x_i y_j
\end{equation*}

Ovvero, la variazione del peso tra due neuroni è proporzionale al prodotto dei 
segnali in input e in output. Interprentando la formula, il peso della
connessione tra i due neuroni aumenta se i due neuroni hanno lo stesso segno (+
per +, - per -), e diminuisce se i due neuroni hanno segno opposto; infine
notiamo che $x_i, y_j \in \{-1, 1\}$, per cui anche $x_i y_j \in \{-1, 1\}$.\\
L'apprendimento scava il bacino degli attrattori, in modo che
la funzione di energia abbia minimi locali corrispondenti ai pattern memorizzati.
Allo stesso modo è aumentata l'energia corrispondente a configurazioni
improbabili (non presenti nei pattern).\\
Questo metodo può imparare solo un numero limitato di pattern in base al numero
di neuroni della rete. La dinamica stocastica può aiutare a memorizzare più
pattern, per cui la funzione di attivazione diventa:

\begin{equation*}
	P(x_i) = \frac{1}{1 + e^{-\frac{1}{T} \sum_{j} w_{ij} x_j}}
\end{equation*}

Fondamentalmente, al posto di usare la funzione a scalino si usa una sigmoide
che da un valore di probabilità tra 0 e 1 per l'attivazione del neurone, infatti
$T$ è la temperatura e regola la pendenza della sigmoide e quindi a una
temperatura elevata, la sigmoide è piatta e quindi la probabilità di attivazione
è 0.5; mentre a una temperatura bassa, la sigmoide è molto ripida e quindi la
probabilità di attivazione è 0 o 1 (diventa la funzione a gradino vista prima).\\
L'idea è quella di diminuire la temperatura durante il recupero, in modo che la
rete si assesti in uno stato di equilibrio evitando attratori spuri (fasulli).
Questa tecnica è chiamata \textbf{simulated annealing}.

\subsection{Macchine di Boltzmann}

Le macchine di Boltzmann sono una variante stocastica delle reti di Hopfield,
che sfruttano unità nascoste per estrarre correlazioni di ordine superiore dei
dati (astrazioni maggiori sui dati).
Hanno una capacità di memorizzazione superiore, infatti i neuroni nascosti sono
usati per comprimere l'informazione.
Le macchine di Boltzmann computazionalmente sono molto costose.\\
Fondamentalmente, oltre alle unità visibili, sono presenti unità nascoste
totalmente connesse tra loro e con le unità visibili.

\subsection{Macchine di Boltzmann Ristrette}
 
 In questo caso la complessità computazionale viene enormemente ridotta. Ora ci
 sono due strati di neuroni: uno visibile e uno nascosto. Gli strati sono
 completamente connessi tra loro, ma non ci sono connessioni tra neuroni dello
 stesso strato.\\
 La funzione di energia dipende dalle attivazioni dei neuroni visibili e 
 nascosti e dai pesi delle connessioni.

\subsubsection{Divergenza contrastiva}

Lo scopo del modello è quello di minimizzare la discrepanza fra la distribuzione
empirica dei dati (pattern nel training set) e la distribuzione generata dal
modello (pattern generati dalla rete).\\
Intuitivamente vorremmo che il modello generi pattern simili a quelli
dell'allenamento:
\begin{equation*}
	\Delta W = \alpha(v^{+}h^{+} - v^{-}h^{-})
\end{equation*}

dove $v^{+}$ e $h^{+}$ sono le attivazioni dei neuroni visibili al tempo $0$,
mentre $v^{-}$ e $h^{-}$ sono le attivazioni dei neuroni visibili al tempo
$1$. Infatti:
\begin{enumerate}
	\item si passa un pattern in input, per cui $v(0) = v^{+}$ vale quanto
		l'input; 

	\item Si campionano le attivazioni dei neuroni nascosti $h(0)$, ovvero viene
	fatto passare il segnale dallo strato di input allo strato nascosto;

	\item Si campionano le attivazioni dei neuroni visibili $v(1)$, ovvero viene
	fatto passare il segnale dallo strato nascosto allo strato di input;

	\item Si campionano le attivazioni dei neuroni nascosti $h(1)$, ovvero viene
	fatto passare il segnale dallo strato di input allo strato nascosto.

	\item Si aggiornano i pesi delle connessioni con la formula sopra.

	\item Si ripete il processo per un numero di iterazioni fissato.
\end{enumerate}

Dove nei punti 1 e 2 si ha la fase positiva, mentre nei punti 3 e 4 si ha la
fase negativa.\\
Si nota la somiglianza con gli autoencoder: l'econder è rappresentato dalla fase
attiva, mentre il decoder è rappresentato dalla fase negativa.\\
Sembra che i primi stadi di processamento visivo possano essere appresi in modo
non supervisionato (non è detto che si utilizzino le macchine di Boltzmann).\\
Si possono combinare varie RBM per apprendere modelli interni più complessi: lo
strato nascosto di una RBM è usato come input per la RBM successiva.

\subsection{Generative Adversarial Network (GAN)}

Si utilizza un training set per allenare un discriminatore, ovvero una rete che
deve distinguere i pattern reali da quelli generati da un generatore. Il
generatore è una rete che cerca di generare pattern simili a quelli del training
set. \\
Un po' casualmente ha citato che si può prendere un transformer in grado di
comprendere il testo e mappare la sua rappresentazione dello spazio latente con
un GAN per generare immagini a partire da testo, DALL-E è un esempio di questo
processo. Questo processo è chiamato fine-tuning.


