\section{Apprendimento e memoria nelle reti neurali}

Il machine learning consiste nei metodi e negli algoritmi che permettono ad un
software di apprendere dall'esperienza. Gli algoritmi di apprendimento
automatico apprendono dai dati. La qualità dell'apprendiemnto tipicamente
migliora all'aumentare del numero di esempi disponibili per l'addestramento. Le
reti neurali artificiali rappresentano una classe importante di algoritmi di
apprendimento, ma non sono gli unici metodi di machine learning.\\
Il deep learning si riferisce alle reti neurali artificiali (con almeno due
strati nascosti) e rappresenta lo stato dell'arte.\\
Il machine learning si divide in tre categorie a seconda dei dati forniti al
modello:
\begin{itemize}
\item \textbf{Apprendimento supervisionato}: al modello sono presentati dei dati
in input e i risultati attesi, lo scopo è di imparare a produrre l'output
corretto dato un nuovo input;

\item \textbf{Apprendimento non supervisionato}: al modello sono presentati solo
dati in input, non gli viene detto che cosa cercare. Lo scopo è di costruire una
rappresentazione dell'input scoprendo le proprietà più importanti e informative
(ci saranno esempi di modelli di questo tipo);

\item \textbf{Apprendimento per rinforzo}: il modello produce delle azioni a
partire da un input. A seconda delle azioni che produce, il modello riceve dei
rinforzi o delle punizioni (come carota e bastone). Lo scopo del modello è
mangiare più carote possibili e evitare i bastoni nel lungo periodo. Formalmente
il modello massimizza i rinforzi a lungo termine.
\end{itemize}

\subsection{Apprendimento}

L'apprendimento in una rete neurale consiste nel trovare l'insieme di pesi delle
connessioni che permette alla rete di produrre la risposta appropriata ad un
certo input.\\
La forma più semplice di apprendimento è la regola di Hebb: \textit{se due
neuroni collegati tra loro sono contemporaneamente attivi, l'efficacia sinaptica
della connessione viene aumentata}. L'apprendimento hebbiano è biologicamente
plausibile.\\
I contro della regola di Hebb sono:
\begin{itemize}
\item la regola di hebb può solo aumentare il peso delle connessioni;
\item i valori dei pesi (le connessioni) possono cresce indefinitamente
(all'infinito);
\end{itemize}

I valori dei pesi sinaptici iniziali sono impostati in modo casuale a valori
piccoli. Durante l'allenamento sono presentati più volte gli stessi pattern di
addestramento. L'apprendimento consiste nella modifica dei pesi, ovvero il
calcolo di $\Delta w_{ij}$, rispetto a $w_{ij}$, dove $w_{ij}$ è il peso della
connessione tra il neurone $i$ e il neurone $j$.\\
Se l'aggiornamento dei pesi avviene dopo ogni pattern si chiama on-line
learning.  Se l'aggiornamento dei pesi avviene dopo ogni epoca si chiama batch
learning.\\
per non stravolgere le conoscenze precedentemente apprese, viene utilizzata solo
una frazione della modifica sinpatica calcolata. Il fattore molteplicativo
$\eta$ è chiamato tasso di apprendimento.
\begin{equation*}
	w^t_{ij} = w^{t-1}_{ij} + \eta \Delta w^t_{ij}
\end{equation*}

Dove $t$ è l'epoca corrente, $w^t_{ij}$ è il peso della connessione tra il
neurone $i$ e il neurone $j$ all'epoca $t$, $w^{t-1}_{ij}$ è il peso della
connessione tra il neurone $i$ e il neurone $j$ all'epoca $t-1$, $\eta$ è il
tasso di apprendimento e $\Delta w^t_{ij}$ è la modifica del peso della
connessione tra il neurone $i$ e il neurone $j$ all'epoca $t$.\\

\subsection{Memoria}

La memoria in una rete neurale consiste nella capacità di memorizzare la
corrispondenza tra un input e un output. La memoria di un modello è data
dall'insieme dei pesi delle connessioni.\\
L'attivazione dei neuroni è un fenomeno temporaneo ed è specifico per lo stimolo
presentato e si esaurisce con la sua scomparsa. Fondamentalmente, da uno stimolo in
input corrisponde un'esito in output. Questo vuol dire che c'è
un'associazione tra tutti gli stimoli possibili in input e ciascun esito in
output. Queste associazioni sono memorizzate nei pesi delle connessioni.\\
Invece, se l'attivazione non cessa bruscamente, può influenzare l'elaborazione
dello stimolo successivo (reti ricorrenti). Alcuni compiti cognitivi richiedono
di ricordare per breve tempo informazioni che non sono più presenti utilizzanod
la memoria di lavoro o a breve termine.

\paragraph{Esempio del fallimento della regola di Hebb} Se una rete neurale
viene sottoposta al compito di apprendimento associativo AB-AC si ottiene un
effetto di interferenza ancora maggiore che negli esseri umani. Ovvero viene
rinforzata la connessione AB e la connessione AC, dunque ogni volta che si
riceve A in input, la risposta è arbitraria tra B e C, perché entrambe le
connessioni AB e AC sono state rafforzate.\\
Due fattori determinano l'interferenza in una rete neurale:
il grado di sovrapposizione delle rappresentazioni, ovvero quanto sono simili B
e C e la correlazione tra i pattern di input; e il tasso di apprendimento
elevato. Biologicamente, il cervello umano ha sviluppato due sistemi di
apprendimento separati e complementari:
\begin{itemize}
\item \textbf{Ippocampo}: specializzato nell'apprendimento rapido e non
soggetto a interferenze, con rappresentazioni sparse;
\item \textbf{Neocorteccia}: apprende lentamente e integra gradualmente le
esperienze estrendo le conoscenze generali sul mondo. In questo caso la
rappresentazione è distribuita (sono coinvolti più neuroni e c'è maggiore
precisione).
\end{itemize}
