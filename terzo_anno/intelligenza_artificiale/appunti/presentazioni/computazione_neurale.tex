\section{Computazione Neurale}
La computazione neurale consiste in un nuovo paradigma di elaborazione
dell'informazione, ovvero viene contrapposto alla computazione digitale.
La computazione neurale viene eseguita da reti di neuroni biologici o
artificiali.\\
Le reti biologiche sono formate da neuroni che si influenzano attraverso le
connessioni (sinapsi) che li collegano:
\begin{itemize}
	\item ogni neurone rileva un certo insieme di condizioni e segnala ciò che ha
	rilevato attraverso la sua frequenza di scarica;
	\item i neuroni possono ricevere segnali da altri neuroni e formare strati di
	rilevatori più complessi, il cui segnale indica la presenza di qualche struttura
	più elaborata (letteralmente come somma di segnali più semplici);

	\item le interazioni tra neuroni sono adattive e si modificano attraverso
	l'apprendimento.
\end{itemize}

Ciascun neurone è sintonizzato su uno stimolo preferito, specifico. Per cui se
nell'input c'è lo stimolo ricercato dal neurone il neurone emette un segnale
molto forte, altrimenti emette un segnale debole. Il campo recettivo del neurone
è quella porzione di spazio sensoriale che attiva la risposta del neurone.\\
Dunque, la codifica dell'informazione diventa più complessa ed astratta passando
a livelli più profondi della gerarchia. Infatti ci negli
stratti più vicini all'input sono codificate caratteristiche molto semplici,
queste sono raggruppate negli strati successivi in maniera sempre più complessa
fino ad arrivare a caratteristiche molto complesse.\\
Le reti neurali artificiali sono sistemi di elaborazione in grado di apprendere
compiti complessi e ispirati al funzionamento dei sistemi biologici.\\
Una rete neurale artificiale ha un'architettura ad elaborazione distribuita e
parallela; questo vuol dire che "sullo stesso livello" (layer letteralmente)
possono essere presenti più neuroni che lavorano in parallelo. Per esempio, data
un'immagine, essa può contenere un quadrato ed una stella, e ci possiamo
immaginare che ci siano due neuroni sullo stesso livello che si attivano nello
stesso momento e che rilevano la presenza di un quadrato e di una stella. In
generale, possiamo assumere che neuroni sullo stesso livello codifichino
(segnalino la presenza di) caratteristiche ugualmente complesse. Le reti neurali
sono caratterizzate da semplici unità di elaborazione (eseguono solo l'addizione
e la moltiplicazione), con un'elevata interconnessione tra le unità. I messaggi
che viaggiano tra le unità sono semplici numeri reali (i numeri che studiamo
alle superiori, quelli con la virgola per intenderci); infine, le reti neurali
possono evolvere nel tempo (ovvero possono imparare) e lo fanno modificando i
pesi delle connessioni tra le unità.\\
Nell'ambito delle scienze cognitive, le reti neurali artificiali sono utilizzate
con lo scopo di "ricostruire" le abilità cognitive e il comportamento umano.
Infatti per spiegare come funziona una funzione cognitiva, viene costruita una
simulazione adeguata, utilizzando un substrato di elaborazione che somigli
quanto più possibile al substrato biologico.\\

La rete neurale si compone di:
\begin{itemize}
	\item neuroni: sono dei rilevatori, per cui la loro attivazione segnla che hanno
	rilevato qualche cosa;

	\item reti neurali: connettono, coordinano, amplificano e selezionano
	pattern di attivazioni su neuroni;

	\item apprendimento: organizza le reti per eseguire compiti e per sviluppare
	modelli interni dell'ambiente. In particolare, viene attuato modificando la
	rete naurale (in base ai pattern in input).
\end{itemize}

\subsection{Neurone}

Un neurone formale (unità di elaborazione o nodo) è un modello matematico che
cerca di catturare gli aspetti fondamentali del funzionamento neuronale:
\begin{itemize}
\item I neuroni hanno connessioni in ingresso e in uscita con altri neuroni;
\item Ogni connessione ha un peso che indica la forza della relazione e può
avere valore positivo (eccitatorio) o negativo (inibitorio);
\item L'input di un neurone da un altro neurone si calcola come moltiplicazione
del segnale di output del neurone in ingresso per il peso della connessione;
\item L'input totale di un neurone è la somma di tutti gli input calcolati al 
passo precedente;
\item Lo stato di attivazione finale viene calcolato come funzione di
attivazione o di output del neurone (ovvero si prende la somma precedentemente
definita e la si passa ad una funzione che restituisce un valore, che è il
valore di attivazione del neurone);
\item L'output del neurone (il suo valore di attivazione) viene passato a tutti
i neuroni a cui è connesso (in uscita e pesando il segnale con la connessione,
ovvero moltiplicazione al punto 3).
\end{itemize}

\subsection{Reti di neuroni}

L'architettura della rete identifica:
\begin{itemize}
\item l'organizzazione in gruppo o strati dei neuroni;
	\begin{itemize}
		\item le unità che ricevono input dirrettamente dall'ambiente formano lo strato
		di input;
		\item le unità che producono l'output finale della rete formano lo strato di
		output;
		\item gli strati intermedi sono detti strati nascosti.
	\end{itemize}
\item il modo in cuii neuroni sono collegati tra di loro;
	\begin{itemize}
		\item reti feed-forward: ci sono solo connessioni unidirezionali da unità di
		input a unità nascoste a unità di output;

		\item reti ricorrenti: ci sono connessioni bidirezionali, per cui
		l'attivazione di uno strato può attivare uno strato precedente;

		\item reti interamente ricorrenti: come le reti ricorrenti, in più ci
		sono connessioni anche tra unità dello stesso strato.
	\end{itemize}
\end{itemize}

\subsection{Differenze tra reti neurali biologiche e artificiali}

Le reti neurali artificiali catturano i principi base di funzionamento delle
reti neurali biologiche, ma non sono copie fedeli (perché non siamo in grado di
costruire una copia fedele).\\
Cosa manca al neurone artificiale:
\begin{itemize}
\item organizzazione spaziale dei contatti sinaptici: le connessioni tra neuroni
biologici sono molto più complesse di quelle tra neuroni artificiali; inoltre,
nel cervello ci sono gruppi di neuroni che lavorano insieme e poi ci sono dei
neruoni che collegano questi gruppi, ma non abbiamo ancora compreso questo
meccanismo;
\item differenziazione tra neuroni eccitatori e inibitori: i neuroni artificiali
sono tutti uguali, mentre i neuroni biologici possono essere eccitatori o
inibitori;
\item tipi diversi di sinapsi: nelle reti artificiali abbiamo solo connessioni
con peso positivo o negativo, mentre nel cervello ci sono sinapsi con diversi
meccanismi di trasmissione;
\item ?: qualcos'altro che non sono in grado di spiegare;
\end{itemize}

Cosa manca alla rete artificiale:
\begin{itemize}
\item struttura laminare e organizzazioen colonnare: le reti neurali artificiali
sono bidimensionali, mentre il cervello è tridimensionale;
\item organizzazione in mappe topografiche: la rete non si divide in gruppi per
rappresentare informazioni diverse o capacità diverse;
\item differenza di scala: le reti neurali artificiali sono molto più piccole
rispetto al cervello;
\end{itemize}

Infine, le differenze maggiori sono quelle di cui ancora non ci rendiamo conto,
meglio simuliamo il cervello e più ci rendiamo conto di quante cose manchino per
rappresentarlo fedelmente, questo elenco non è esaustivo.
