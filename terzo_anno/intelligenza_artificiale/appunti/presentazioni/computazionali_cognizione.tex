\section{Modelli computazionali della cognizione}

Le scienze cognitive studiano il funzionamento della mente, cioè descrivono,
spiegano e predicono il comportamento umano.\\
Per fare ciò, risulta utile introdurre il formalismo matematico e dei modelli
quantitativi, per simulare le teorie e soprattutto per testarle.
I modelli quantitativi possono essere di due tipi:
\begin{itemize}
	\item \textbf{descrittivi}: riassumono i dati in forma matematica;

	\item \textbf{computazionali}: fanno assunzioni sui processi o sulle
	rappresentazioni della mente. I parametri e le carratteristiche del modello
	hanno interpretazioni psicologiche.
\end{itemize}

La modellazione computazionale, l'uso dei computer per simulare e studiare
sistemi complessi utilizzando la matematica e l'informatica, è lo strumento
fondamentale per costruire ipotesi su come funzioni la mente.

\paragraph{Scopo} di un modello computazionale:
\begin{itemize}
	\item rimpiazzare oppure migliorare modelli verbali;

	\item sistematizzare i dati sperimentali;

	\item testare le teorie, per trovare la migliore e continuare a migliorarla;

	\item simulare il comportamento di un sistema complesso e generare
		predizioni.
\end{itemize}

\paragraph{Simulazione} è il processo di costruzione di un modello computazionale
di un sistema reale e l'esecuzione di questo modello per studiare il sistema
reale. Per capire se il modello è più o meno allineato al funzionamento della
mente, si possono fare confronti con i dati sperimentali. Di seguito viene
riportato qualche esempio dei dati che sono confrontati rispetto al modello per
stimarne la plausibilità:
\begin{itemize}
	\item accuratezza e tempi di reazione in compiti sperimentali;

	\item come una funzione cognitiva venga danneggiata in seguito a una lesione
		celebrale;

	\item la traiettoria di acquisizione di una specifica abilità cognitiva
		(sviluppo cognitivo tipico e atipico).
\end{itemize}

\subsection{Teoria into modello}

Una teoria non può essere implementata in una simulazione al computer se non è
pienamente specificata. Formulare una teoria sui processi cognitivi attraverso
un modello computazionale rivela gli aspetti che rendono la teoria incompleta o
sotto-specifica. Infine, molte assunzioni potrebbero essere implicite, renderle
esplicite è importante per capire i successi e i fallimenti di un modello.\\
La discrepanza tra i dati reali e i dati simulati ci rivelano in che modo la
teoria di partenza è sbagliata; ogni tanto aiuta persino a capire dove la teoria
sia sbagliata e quindi rende possibile correggerla.

\subsection{Valutazione dei modelli computazionali}

Il grado di accuratezza con cui un modello predice un set di dati empirici sia a
livello qualitativo che quantitativo è chiamata adeguatezza descrittiva,
accuratezza predittiva o fit del modello.\\
L'adeguatezza descrittiva di un modello può essere valutata in due modi:

\begin{itemize}
	\item \textbf{qualitativamente}: valuta se le risposte del modello sono 
		modulate dagli stessi fattori che influenzano le risposte umane;

	\item \textbf{quantitativamente}: confrontando i dati simulati con i dati
		empirici, per vedere quanto il modello riesca a predirre i dati reali,
		ovvero quanto il modello riesce a predirre le risposte biologiche (per
		esempio, umane).
\end{itemize}

I due punti appena citati sono usati come principali indicatori per valutare
l'adeguatezza predittiva di un modello. In realtà ci sono un altro paio di punti
che possono essere usati per valutare un modello, d'altra parte, questi sono
considerati secondari:

\begin{itemize}
	\item \textbf{Generalità}: un modello dovrebbe essere in grado di rendere
		conto di più fenomeni e simularli attraverso variazioni del set di
		stimoli e del tipo di compito o di risposta;

	\item \textbf{Semplicità}: quando due modelli risultano essere ugualmente
		accurati, allora il modello più semplice è preferibile. La semplicità ha
		la forma di un'architettura più semplice, di un numero minore di
		parametri o di un numero minore di assunzioni;

	\item \textbf{Falsificazione}: se due modelli risultano indistinguibili,
		allora si prova a falsificare un modello, ovvero si cerca di trovare un
		compito o un set di dati che uno dei due modelli non riesce a 
		simulare. 
\end{itemize}

\subsection{Modelli connessionisti}

I modelli connessionisti si dividono in due categorie a seconda della teoria che
vuole essere simulata:
\begin{itemize}
	\item \textbf{localistici}: sono modelli della prestazione e non
	dell'apprendimento. Ogni nodo della rete ha un ruolo predefinito e codifica
	localmente le informazioni specifiche. Il modello è costruito a priori.
	L'enfasi è posta sulla simulazione dei dati comportamentali;

	\item \textbf{PDP (Parallel Distributed Processing)}: sono modelli 
		dell'apprendimento. Ogni nodo della rete codifica
		informazioni globali e locali. Alcuni livelli di rappresione possono
		emergere spontaneamente. L'enfasi è posta sulla simulazione dei dati
		comportamentali e sulle dinamiche di apprendimento.
\end{itemize}
