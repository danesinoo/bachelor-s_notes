\section{Model}

Sentiment classification takes a natural language text as input and outputs a
sentiment score $\in \{0, 1, 2, 3, 4\}$. Here we give a brief description of the
pretrained BERT model and then we describe our model architecture.

\subsection{BERT}

BERT is an embedding layer designed to train deep bidirectional representation
from unlabeled text by jointly conditioning on both left and right context in
all layers. It is pretrained from a large unsupervised text corpus using the
following objectives:

\begin{itemize}
	\item \textit{Masked word prediction}: 15\% of the words in the input
	      sequence are masked out, the entire sequence is fed to a deep bidirectional
	      Transformer encoder, and then the model learns to predict the masked words.

	\item \textit{Next sentence prediction}: to learn the relationship between
	      sentences, BERT takes two sentences $A$ and $B$ as inputs and learns to
	      classify whether $B$ actually follows $A$ or is it just a random sentence.
\end{itemize}

Unlike traditional sequential or recurrent models, the attention architecture
processes the whole input sequence at once, enabling all input tokens to be
processed in parallel. Pretrained BERT model can be fine-tuned with just one
additional layer to obtain state-of-the-art results in a wide range of NLP
tasks.

\subsection{Preprocessing}

BERT requires its input token sequence to have a certain format. So we perform
the follwoing preprocessing steps on the review text before we feed them into
our model:
\begin{enumerate}
	\item \textit{Canonicalization}: first, we remove all the digits,
	      punctuation sumbols and acent marks, and convert everything to lowercase;

	\item \textit{Tokenization}: we then tokenize the text using the word-piece
	      tokenizer. It breaks the words down to their prefix, root and suffix to
	      handle unseen words better;

	\item \textit{Sepcial toekn addition}: finally we add the \texttt{[CLS]} and
	      \texttt{[SEP]} tokens at the appropriate positions.
\end{enumerate}

\subsection{Architecture}

We build a simple architecture with just a dropout regularization and a softmax
classified layers on top of pretrained BERT layer to demonstrate that BERT can
produce gread resulsta even without any sophisticated task-specific
architecture. There are four main stages:

\begin{enumerate}
	\item preprocessing: described above;

	\item sequence embedding: from BERT;

	\item regularization: dropout layer with a dropout rate of 0.1 to prevent
	      overfitting;

	\item softmax classification: it will output the probabilities of the input
	      text belonging to each of the class labels such that the sum of the
	      probabilities is 1. The softmax layer is just a fully connected neural
	      network with the softmax activation function. The softmax function $\sigma:
		      \mathbb{R}^K \rightarrow \mathbb{R}^K$ is given by:
	      \begin{equation}
		      \sigma(x)_i = \frac{e^{z_i}}{\sum_{j=1}^K e^{z_j}} for i = 1, \ldots, K
	      \end{equation}

	      where $z = (z_1, \ldots, z_K) \in \mathbb{R}^K$ is the intemediate output of
	      the softmax layer (also called logits). The output node wiht the
	      highest probability is then chosen as the predicted label for the
	      input.
\end{enumerate}
