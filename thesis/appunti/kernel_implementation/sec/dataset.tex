\section{Dataset}

\begin{itemize}

\item È stata ottenuta una versione modificata del treebank, in cui le golden
    label sono state sostituite con i rispettivi POS-tag, utilizzando lo 
    Stanford Parser. In particolare, il parser è stato utilizzato tramite il 
    container Docker descritto nella repository \cite{docker-parser}.
\item I due dataset risultanti sono stati convertiti nel formato svmlight.
\item È stato generato un terzo dataset unendo gli alberi dei due dataset 
    precedenti.
\item Poiché il software fornito consente di effettuare sia regressione che 
    classificazione binaria, i dataset sono stati ulteriormente elaborati.
\item È stato creato un dataset contenente solo esempi positivi o negativi, 
    unendo le classi "negativo" e "un po' negativo", e le classi "positivo" e 
    "un po' positivo". La classe "neutro" è stata eliminata.
\item Sono stati creati cinque dataset distinti per la classificazione 
    multiclasse; ciascun dataset è stato progettato per addestrare un modello 
    specializzato su una singola classe, ponendo tutte le classi tranne una a -1
    e la classe target a 1.
\item Le classi dei modelli sono state normalizzate trattando il valore come un 
    float. In particolare, nel dataset di partenza le classi variavano da 0 a 4, 
    quindi è stata applicata l'operazione $(\text{class} - 2) / 2$, ottenendo 
    valori compresi tra -1 e 1. In questo modo è ottenuto il treebank per
    allenare il modello di regressione.
\item Sono stati generati dataset contenenti tutti i sottoalberi (togliendo le
    foglie) dei dataset con etichette di sentimenti sviluppati fino a questo
    punto.
\end{itemize}

Il codice utilizzato per generare i dataset è disponibile nella repository 
\cite{repo}. Non sono stato in grado di generare i sottoalberi per i dataset
con etichette di sintassi, poiché gli alberi di sintassi differiscono per poco
da quelli di sentimenti, per cui diventa complesso assegnare il target corretto
ai sottoalberi. Per esempio, un albero di sintassi divide la parola
"fully-written" in tre foglie "fully", "-", "written", mentre nell'albero di
sentimenti è una sola foglia con etichetta "fully-written". Questa non è l'unica
differenza che ho notato, ma anche una differenza così piccola rende complesso
da abbinare i sottoalberi alla golden label corretta.
