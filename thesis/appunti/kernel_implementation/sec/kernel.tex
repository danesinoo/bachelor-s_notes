\section{Modello}

Per l'allenamento dei modelli, è stato sviluppato un programma che utilizza i 
dataset forniti. L'allenamento complessivo ha avuto una durata media di circa 
1 ore per ciascun modello.\\
Il kernel method utilizzato è stato esclusivamente il subset tree 
kernel. Abbimao provato ad utilizzare anche il partial tree kernel,
l'allenamento è andato a buon fine, tuttavia non è stato possibile predire i
risultati sul dataset di test, perché il programma che esegue la classificazione
termina con un segmentation fault. Risalendo alla causa del problema, abbiamo
scoperto che il problema è dovuto alla normalizzazione dell'albero di test,
durante la lettura del pattern.\\

Abbiamo provato a modificare la funzione di attivazione, in particolare abbiamo
sperimentato:
\begin{itemize}
    \item \textbf{linear}: l'allenamento non funziona, perché il programma va in
        segmentation fault;

    \item \textbf{polynomial}: l'allenamento non funziona, perché viene allenato
        un modello vuoto;

    \item \textbf{radial basis function}: l'allenamento non funziona, perché
        viene allenato un modello vuoto;

    \item \textbf{sigmoid}: l'allenamento non funziona, perché viene allenato un
        modello vuoto.

    \item \textbf{reranking}: l'allenamento funziona e il modello è in grado di
        predire i risultati sul dataset di test;

    \item \textbf{forest combination}: l'allenamento funziona e il modello è in
        grado di predire i risultati sul dataset di test.
\end{itemize}

Ancora, per quanto riguarda il reranking, come funzione di attivazione, abbiamo
modificato i seguenti iperparametri:
\begin{itemize}
    \item decay factor;
    \item normalizzazione;
    \item metodo di somma degli alberi.
\end{itemize}

In particolare, non è stato possibile ottenere l'acurattezza dei modelli sui
nodi degli alberi che hanno decay factor diverso da 0. Non siamo stati in grado
di capire il motivo di questo comportamento.\\
Per quanto riguarda la combinazione di foreste, abbiamo sperimentato con gli
stessi iperparametri del reranking, ottenendo risultati simili.\\
Si noti che è possibile usare reranking come funzione di attivazione solo sul
dataset che combina la sintassi e il sentiment, mentre la combinazione di
foreste può essere utilizzata su tutti i dataset.
