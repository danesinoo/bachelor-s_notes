\section{Conclusioni}

\subsection{Sentiment vs Syntax vs Subtree}

I modelli focalizzati sulla sintassi hanno mostrato prestazioni inferiori 
rispetto agli altri due approcci. Questa differenza è particolarmente evidente 
nell'analisi su tutti i nodi dell'albero. Tale risultato è prevedibile dato 
che il treebank di test utilizza etichette di riferimento non incluse nella 
formazione dei modelli, rendendo il confronto meno rilevante. \\
Nonostante ciò, anche limitando l'analisi alla radice dell'albero, i modelli 
basati sulla sintassi hanno registrato performance inferiori. Tale tendenza non 
si verifica nella classificazione dei pattern basata sulla regressione. \\
Al contrario, i modelli che integrano l'analisi del sentiment con quella
sintattica tendono a superare quelli basati esclusivamente sul sentiment, anche
se i risultati sono simili quando si focalizzano sulla predizione del target
alla radice dell'albero.\\
In aggiunta, i modelli che operano sui sottoalberi hanno dimostrato un'efficacia
notevolmente superiore nella classificazione dei pattern su ogni nodo, pur
mantenendo risultati comparabili a quelli dei modelli basati sul sentiment per
quanto riguarda la radice.

\subsection{Kernel Methods vs RNN}

Le tecniche basate su Kernel Methods, quando applicate ai sottoalberi, hanno
evidenziato performance superiori rispetto alle RNN con 50 unità nascoste. Le
RNN, però, si distinguono nettamente nell'identificazione dei pattern su ciascun
nodo dell'albero rispetto agli altri modelli.\\
Si osserva, inoltre, che nei risultati relativi alla radice dell'albero, i
modelli che combinano il sentiment con i Kernel Methods superano
significativamente le RNN (0.12), evidenziando un divario maggiore di quello 
notato quando le RNN confrontano i loro risultati con quelli dei Kernel Methods
allenati sulla sintassi (0.3).
