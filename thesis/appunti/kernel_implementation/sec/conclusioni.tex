\section{Conclusioni}

\subsection{Sentiment vs Syntax vs Subtree}

I modelli allenati sulla sintassi hanno mostrato prestazioni inferiori 
rispetto agli altri due approcci. Questa differenza è particolarmente evidente 
nell'analisi su tutti i nodi dell'albero. Tale risultato è prevedibile dato 
che il treebank di test utilizza etichette di riferimento non incluse
nel treebank di allenamento dei modelli, rendendo il confronto meno rilevante.\\
Nonostante ciò, anche limitando l'analisi alla radice dell'albero, i modelli 
basati sulla sintassi hanno registrato performance inferiori. Tale tendenza non 
si verifica nella regressione. \\
Al contrario, i modelli che integrano l'analisi del sentiment con quella
sintattica tendono a superare quelli basati esclusivamente sul sentiment, anche
se i risultati sono simili quando si focalizzano sulla predizione del target
alla radice dell'albero.\\
In aggiunta, i modelli allenati sui sottoalberi hanno dimostrato un'efficacia
notevolmente superiore nella classificazione dei pattern su ogni nodo, pur
mantenendo risultati comparabili a quelli dei modelli basati sul sentiment per
quanto riguarda la radice.

\subsection{Kernel Methods vs RNN}

Le tecniche basate su Kernel Methods, quando applicate ai sottoalberi, hanno
evidenziato performance superiori rispetto alle RNN. \\
Invece, le RNN si distinguono nettamente nell'identificazione dei pattern su 
ciascun nodo dell'albero rispetto modelli rimanenti (modello allenato su 
sentiment \& co.).\\
Si osserva, inoltre, che nei risultati relativi alla radice dell'albero, il
modello che combina il sentiment con i Kernel Methods supera
significativamente le RNN (0.12), evidenziando un divario maggiore di quello 
notato tra le RNN e il Kernel Method allenato sulla sintassi (0.3).\\
In ogni caso, è importante notare che il modello RNN ha dimensione fissa
rispetto al numero di unità nascoste scelte per la rete, d'altro canto i kernel
method aumentano di dimensione con il numero di pattern nel treebank di
allenamento, per questo motivo su un dataset più grande i kernel method
potrebbero risultare poco efficienti o addirittura non utilizzabili.

\subsection{Iperparametri}

Putroppo non è stato possibile eseguire un'analisi accurata degli iperparametri
perché il software trovato fallisce nel calcolo delle performance dei modelli
oppure non è in grado di classificare i pattern come richiesto.\\
I risultati mostrano che l'iperparametro 'W', ovvero forest sum, non è
rilevante, infatti qualunque valore assuma non influisce sulle performance del
modello.\\
La normalizzazione degli alberi peggiora le performance dei modelli migliori,
tuttavia ogni tanto può portare ad una maggiore accuratezza delle predizioni. In
ogni caso, un modello che normalizza l'albero richiede un tempo di allenamento
in media 2 volte superiore rispetto allo stesso modello non normalizzato.\\
Infine, il decay factor migliore risulta essere di 0.4 per i modelli subset
tree, subset tree-bow e partial tree, mentre il modello subtree con decay factor
0.3 ha performance migliori.\\
Infine si nota che i modelli subset tree, subset tree-bow e partial tree
ottengono performance comparabili: più del 54\% di accuratezza nella
classificazione dei pattern sulla radice dell'albero. Mentre il modello con
kernel subtree raggiunge un'accuratezza poco superiore al 41\%, simile all'RNN
con 50 unità nascoste.
