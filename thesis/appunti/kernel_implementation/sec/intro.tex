\section{Introduzione}

La sperimentazione sui Kernel Methods è stata condotta utilizzando il software 
fornito e descritto nella pagina citata in \cite{kernel}. A causa di alcune 
incertezze riguardo le modalità operative, ho consultato il professore per 
chiarimenti. Durante l'incontro, abbiamo deciso di intraprendere le seguenti 
sperimentazioni:

\begin{itemize}
    \item Sperimentazione su diversi dataset:
    \begin{itemize}
        \item Un dataset per la regressione.
        \item Un dataset per la classificazione binaria (positivi e negativi).
        \item Cinque dataset, ciascuno rappresentante una classe distinta.
        \item Dataset sui sottoalberi.
    \end{itemize}
\item Sperimentazione su diversi kernel:
    \begin{itemize}
        \item Moschitti et. al. (2007): non sono sicuro sia il Partial Tree
            Kernel;

        \item VISHTANAM and SMOLA 2002: Subtree Kernel;

        \item COLLINS and DUFFY 2002: Subset Tree Kernel;

        \item ZHANG, 2003: Subset Tree Bow Kernel;

        \item MOSCHITTI, 2006: Partial Tree Kernel.

        \item Something ibrid.
    \end{itemize}
\end{itemize}

In particolare dei kernel sperimentati, per quanto riguarda gli ultimi due non 
sono stato in grado di effettuare delle predizioni perché il programma di 
classificazione fallisce in segmentation fault. 
Non sono stato in grado di sperimentare la funzione
di kernel lineare per nessun modello, perché il programma di allenamento
fallisce in segmentation fault. Infine, per tutti i modelli con funzione di
kernel polinomiale, radiale o sigmoidale, il programma di allenamento ritorna un
modello vuoto, ovvero un modello senza alcun support vector. Da questo
deriva che le predizioni sono casuali perché il modello non impara. Dunque,
l'unica funzione di kernel effettivamente sperimentata risulta essere quella di
re-ranking oppure di combinazione di foreste.
