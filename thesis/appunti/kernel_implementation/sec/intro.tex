\section{Introduzione}

La sperimentazione sui Kernel Methods è stata condotta utilizzando il software 
fornito e descritto nella pagina citata in \cite{kernel}. A causa di alcune 
incertezze riguardo le modalità operative, ho consultato il professore per 
chiarimenti. Durante l'incontro, abbiamo deciso di intraprendere le seguenti 
sperimentazioni:

\begin{itemize}
    \item Sperimentazione su diversi dataset:
    \begin{itemize}
        \item Un dataset per la regressione.
        \item Un dataset per la classificazione binaria (positivi e negativi).
        \item Cinque dataset, ciascuno rappresentante una classe distinta.
        \item Dataset sui sottoalberi.
    \end{itemize}
    \item Sperimentazione esclusiva su Partial Tree Kernels, scelti per le loro 
    buone prestazioni e tempi di allenamento ridotti.
\end{itemize}

In realtà non è stato possibile sperimentare sul Partial Tree Kernel, perché il
programma di classificazione va in segmentation fault. Per questo motivo,
ho sperimentato sul Subset Tree Kernel,
