\section{Summary from the paper}

Further progress towards understanding compositionality in tasks such as
sentiment detection requires riched supervised training and evaluation resources
and more powerful models of computation. Particularly, in this note, we will
discuss about the former. The authors introduce a Sentiment Treebank, which
includes fine grained sentiment labels for 215,154 phrases in the parse trees of
11,855 sentences. The \textit{Sentiment Penn Treebank} is the first corpus with
fully labeled parse trees tha allows for a complete analysis of the
compositional effects of sentiment in language.\\ 
The corpus is based on the dataset introduced by Pang and Lee (2005) and
consists of 11,855 sentences extracted from movie reviews. It was parsed with
the Stanford parser and includes a total of 215,154 unique phrases from those
parse trees, each annotated by 3 human judges.

\subsection{Origin}

It is considered the corpus of movie review excerpts from the
\texttt{rottentomatoes.com} website originally collected and published by Pang
and Lee (2005).
The original dataset includes 10.662 sentences, half of which were considered
positive and the other half negative.\\
The Stanford Parser (...) in used to parses all 10.662 sentences. In
approximately 1.100 cases it splits the snippet into multiple sentences. It was
used Amazon Mechanical Turk to label the resulting 215.154 phrases.\\
Starting at length 20, the majority are full sentences. One of the findings from
labeling sentences based on \textit{reader's perception} is that many of them
could be considered neutral.
