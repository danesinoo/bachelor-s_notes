\section{Related Work}

\subsection{Transformer}

The Transformer model revolutionized NLP by leveraging self-attention mechanisms
to process sequences of data. Here's a breakdown of its core components:

\begin{itemize}
    \item \textbf{Input Representation}: It is used an embedding layer that
        converts words or subwords into dense vectors;

    \item \textbf{Positional Encoding}: Since the Transformer doens't inherently
        capture the order of tokens, positional encodings are added to the input
        embeddings to provide information about the position of each token in 
        the sequence;

    \item \textbf{Multi-head Self-Attention}: 

        \begin{itemize}

        \item \textbf{Self-Attention Mechanism}:
        Self-attention allows each token
        to attend to every other token in the sequence, enabling the model to
        capture dependencies regardless of distance. This is done by computing
        attention scores between pairs of tokens and then using these scores to
        weight the importance of each token’s contribution to the representation
        of the others;

        \item \textbf{Multi-head Mechanism}:
        Instead of using a single attention mechanism, the Transformer uses
        multiple attention mechanisms (or heads) in parallel.

        \item \textbf{Formulas}:
        Given a sentence with $n$ input tokens, the Transformer model
        iteratively computes at layer $t$ the $d$-dimensional representations of
        each input token $H^t \in \mathbb{R}^{n \times d}$, where $H^0$
        represents the initial token embeddings. The core of a Transformer step
        is Multi-head Self-Attention (MSA), which can be formulated as follows:

        \begin{align*}
            \text{MSA}(H) = \left[ \text{head}_1, \dots, \text{head}_h \right]
            W^O,\\
            \text{head}_i = \text{softmax}\left( \frac{Q_i K_i^T}{\sqrt{d}} 
            \right) V_i,\\
            Q_i = H W_i^Q, \quad K_i = H W_i^K, \quad V_i = H W_i^V
        \end{align*}

        Where $h$ is the number of heads, and $W_i^Q, W_i^K, W_i^V, W^O$ are
        learnable parameters;
    \end{itemize}

    \item \textbf{Feed-Forward Networks}:
        Transformer then computes $H^{t+1}$ from $H^t$ as follows:
        \begin{align}
            Z^t = \text{norm} \left( {H^t + \text{MSA}(H^t)} \right),\\
            H^{t+1} = \text{norm} \left( {Z^t + \text{FFN}(Z^t)} \right)
        \end{align}
        where $\text{norm}{\cdot}$ is layer normalization and $\text{FFN}$ 
        stands for Position-wise Feed-Forward Networks. Note that each step $t$ 
        has its own parameters;

    \item \textbf{Layer Normalization and Residual Connections}:
        Each sub-layer (self-attention and feed-forward) if followed by layer
        normalization and residual connections to stabilize training and help
        the model learn more effectively;

    \item \textbf{Stacking Layers}:
        The Transformer model stacks multiple identical layers of self-attention
        and feed-forward networks. This allows the model to learn increasingly
        abstract representations of the input.
\end{itemize}

\subsection{Lightweight Self-Attention}

Recently there has also been several works focusing on reducing the
computational cost of Self-Attention mechanisms. Sparse Transformer decomposes
attention into two categories: for a sequence of length $n$, we divide it into
$\sqrt{n}$ equal-sized blocks. Each token attends to its previous tokens inside
a $\sqrt{n}$ block it lies in, and to $\sqrt{n}$ previous blocks.\\
Transformer-XL introduces the notion of recurrence into Transformer. It divides
the input sequence into multiple segments and recurrently attends to the hidden
states of the previous segments. Transformer-XL can only model sequences in one
direction, making it hard to deal with tasks where bi-directional information is
required.
