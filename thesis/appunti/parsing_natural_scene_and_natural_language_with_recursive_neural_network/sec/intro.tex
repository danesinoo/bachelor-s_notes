\section{Introduction}

Since in the second week I am asked to develop the RNN model and to check its
accuracy over the \textit{Sentiment Penn TreeBank} dataset, I took the
\url{https://icml.cc/2011/papers/125_icmlpaper.pdf} paper, where it is
introduces for the first time. These notes are a summary of such paper.\\
It seems like the new model discovered a recursive structure that helps
identifying the units that an image or sentence contains, plus it identifies
their relationships to form a whole.\\
Super interesting: the same algorithm can be used both to provide a competitive
syntactic parser for natural language sentences and to outperform alternative
approaches for semantic scene segmentation, annotation and classification.\\
So this notes are parted in two: the first part is about the application in the
language field, which is the one I am going to use for the project, and the
second part is about the application in the image field, which I will summarize
just for completeness' sake. Finally, note that some parts of the NLP section
might be needed to understand the following sections. On the other hand, the
other sections are not needed to understand the NLP section.
