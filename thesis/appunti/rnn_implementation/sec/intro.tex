\section{Introduzione}

\subsection{Descrizione del problema}

Il task affrontato dai modelli allenati sullo Sentiment Penn Treebank riguarda 
la classificazione del sentimento di frasi e frasi parziali (nodi) all'interno 
degli alberi sintattici. In particolare, il task principale è la rilevazione del 
sentimento espresso in ogni frase, che può variare da negativo a positivo su una 
scala a cinque livelli:
\begin{itemize}
	\item negativo (- -);
	\item un po' negativo (-);
	\item neutro (0);
	\item un po' positivo (+);
	\item positivo (+ +);
\end{itemize}

In particolare il task si compone di tre sotto-task:
\begin{itemize}
	\item \textbf{Classificazione del Sentimento Fine-Grained}: I modelli devono 
		prevedere l'etichetta di sentimento per ogni frase e sottofrase 
		all'interno di un albero sintattico. Le etichette sono suddivise in 
		cinque categorie.

	\item \textbf{Composizionalità del Sentimento}: Il task richiede ai modelli 
		di catturare e comporre correttamente i sentimenti delle sottofrasi per 
		prevedere il sentimento della frase completa. Ciò include la gestione di 
		fenomeni linguistici complessi come la negazione e le congiunzioni 
		contrastive.

	\item \textbf{Analisi degli Alberi Sintattici}: Utilizzando gli alberi di 
		parsing sintattico generati dal Stanford Parser, i modelli devono 
		processare e analizzare ogni nodo per comprendere come il sentimento si 
		propaga attraverso la struttura dell'albero.
\end{itemize}

\subsection{Dataset}

La descrizione del dataset viene omessa in quanto presente in un documento a
parte.
