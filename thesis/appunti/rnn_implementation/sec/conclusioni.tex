\section{Conclusioni}

I risultati ottenuti confermano le affermazioni del paper
\cite{socher-etal-2013-recursive}. Infatti, il modello RNN dimostra una notevole
robustezza anche variando il numero di unità nascoste. Inoltre, l'accuratezza
sui singoli nodi si attesta intorno all'80\%, in linea con quanto riportato nel
paper, che indica un'accuratezza del 79.0\% per la classificazione fine-grained.
Nel nostro esperimento, il modello con 50 unità nascoste ha raggiunto
un'accuratezza del 79.4\%, un valore molto vicino a quello del paper.\\
Anche l'accuratezza relativa al label della root è comparabile: il paper riporta
un'accuratezza del 43.2\%, mentre il nostro modello con 50 unità nascoste ha
raggiunto il 42.2\%. Questi risultati suggeriscono che il modello è stato
allenato correttamente e che le prestazioni ottenute sono soddisfacenti. Le
differenze riscontrate potrebbero essere dovute all'inizializzazione dei pesi,
poiché il seed utilizzato nel paper non è stato fornito, impedendoci di valutare
se tali differenze derivino da questo aspetto.
