\documentclass{article}
\usepackage{amsmath}
\usepackage{amsfonts}
\usepackage{amssymb}
\usepackage{amsthm}
\usepackage[mathscr]{eucal}
\usepackage[italian]{babel}
\usepackage{graphicx}
\usepackage{hyperref}
\usepackage{fancyhdr}
\usepackage{geometry}
	\geometry{height=21.5 cm}
	\geometry{left=3cm}
	\geometry{right=3 cm}
	\geometry{top=3.5 cm}


\title{\vspace{4cm}\textbf{Appunti di Analisi}}
\author{\textbf{29} settembre \textbf{2021}}
\date{\vspace{2mm}\textbf{Carlo Rosso}}

\begin{document}
\setcounter{secnumdepth}{1}
\newtheorem{dimostrazione}{Dimostrazione}[section]

\begin{titlepage}
	\maketitle
	\thispagestyle{empty}
\end{titlepage}

\tableofcontents

\newpage

Raccomando di controllare wolfarm.com
%Runner.instance_.setSpeed(10000), su https://chromedino.com/
%voglio provare questa cosa
%ho installato latex sull'air
%ptrace() -> to check the security of a program
%0xCC è un'istruzione che viene inviata per fermare un momento un programma che sta venendo debuggato da un altro programma. 0xCC è un'istruzione che è mandata all'ALU. L'ALU deve rispondere in un determinato modo, così che il kernel comprenda che va tutto bene.
%Il programma può debuggare se stesso, se viene fuori -1 comprende che sta venendo debuggato a propria volta.
%patching?


\newpage
\section{Studio di Funzione}

Lo studio di funzione deve essere ordinato e svolto con ordine:
\begin{enumerate}
	\item Dominio: si enuncia il dominio, i punti in cui la funzione è definita;

	\item Limite: si calcola il limite agli estremi del dominio; si verifica l'eventuale esistenza di asintoti, obliqui e non;

	\item Zeri della funzione: studiare in quali punti la funzione si annulla;

	\item Derivata Prima: si calcola la derivata prima della funzione e si comprende in quali intervalli è crescente oppure decrescente;

	\item Derivata Seconda: si studia la concavità e la convessità del grafico ed eventuali punti di flesso.
\end{enumerate}

Gli asintoti obliqui si calcolano con le seguenti equazioni:

\begin{gather}
	\lim_{x \to x_0}f(x)= \pm \infty \\
	\lim_{x \to x_0}\frac{f(x)}{x}= m \\
	\lim_{x \to x_0}f(x)- mx= c \\
	y= mx + c
\end{gather}

La prima condizione è di esistenza del limite obliquo. Nella seconda equazione si calcola la tangente, il coefficiente angolare, della limite obliquo. Grazie alla terza equazione, calcoliamo l'intercetta del limite obliquo. Infine, mettiamo assieme tutte le informazioni e scriviamo la formula del limite obliquo. \textbf{NB}: il limite obliquo è una retta a cui la funzione tende all'infinito.\\

\section{L'Integrale di Riemann}

Si chiama scomposizione di $I0[a;b]$ un sottoinsieme $\sigma$ finito di I tale che $a,b \in \sigma$.\\

Si definisce misura di $I_k=[x_{k-1}, x_k]$ per $k=1, \dots , n$: $mis(I_k)=x_k-x_{k-1}$.\\
Si pone $|\sigma|=\max\{mis(I_k):k=1, \dots, n\}$. Il numero $|\sigma|$ si chiama finezza della scoposizione $\sigma$. Praticamente la finezza di una scomposizione è la distanza più grande tra gli elementi successivi di una scomposizione.\\
\noindent L'insieme di tutte le scomposizioni di $[a,b]$ è indicato con il simbolo $\Omega_{[a,b]}$. Se $\sigma_2 \subset \sigma_1$ si dice che $\sigma_1$ è più fine di $\sigma_2$. Notare che è sufficiente che $\sigma_1$ contenga almeno tutti gli elementi di $\sigma_2$ perchè risulti più fine.\\
Per definizione $S(f, \sigma):= \sum^n_{k=1}\sup f \cdot mis(I_k)$. Cioè la somma delle aree infinitesimali ottenute moltiplicando le distanze tra gli elementi di $\sigma$ per il maggiore dei due elementi si dice \textit{somma superiore}.\\
\noindent D'altro canto, la somma delle aree infinitesimali ottenute moltiplicando le distanze tra gli elementi di $\sigma$ per il minore dei due elementi si dice \textit{somma inferiore} e si indica con $s(f,\sigma):=\sum^n_{k=1}\inf f \cdot mis(I_k)$. Si dice che una funzione $f$ è integrabile nell'intervallo $I$ se e solo se la somma superiore e la somma inferiore del più fine degli elementi di $\Omega_{[a,b]}$ sono uguali. $f$ è integrabile in $I$ se e solo se
\begin{equation*}
	\sup\{s(f,\sigma): \sigma \in \Omega_{[a,b]}\}=\inf\{S(f,\sigma): \sigma \in \Omega_{[a,b]}\}
\end{equation*}
In particolare $\sup\{s(f,\sigma): \sigma \in \Omega_{[a,b]}\}$ è denominato integrale inferiore e $\inf\{S(f,\sigma): \sigma \in \Omega_{[a,b]}\}$ è chiamato integrale superiore. Se l'integrale inferiore e l'integrale superiore coincidono, allora il medesimo valore è definito integrale.\\
La funzione di Dirichlet è uno specimen di funzione non Riemann integrabile.

\subsection{Teorema di Riemann}
Sia $f:[a,b]\rightarrow \mathbb{R}$ limitata. Allora
\begin{equation*}
	f \in \mathcal{R}_{[a,b]} \iff \forall \epsilon > 0 \exists \sigma \in \Omega_{[a,b]}: S(f, \sigma )-s(f, \sigma )<\epsilon
\end{equation*}
$\mathcal{R}_{[a,b]}$ è l'insieme delle funzioni derivabili nell'insieme $[a,b]$.

\paragraph{Teorema 10.1.2} Se $f\in C([a,b])$ allora $f\in \mathcal{R}_{[a,b]}$. Se $f$ è continua in $I$ allora $f$ è integrabile in $I$ (non se e solo se).

\paragraph{Teorema 10.1.3} Se $f:[a,b] \rightarrow \mathbb{R}$ è una funzione limitata, tale che l'insieme $F=\{x \in [a,b]:f $ non è continua in $ x\}$ è finito. Allora $f\in \mathcal{R}_[a,b]$.

\paragraph{Proposizione 10.1.2} Sia $f:[a,b] \rightarrow \mathbb{R}$ monotona. Allora $f \in \mathcal{R}_{[a,b]}$.

\dimostrazione{Proposizione 10.1.2} Sia $f:[a,b] \rightarrow \mathbb{R}$ monotona. 
Allora $f$ è iniettiva in $[a,b]$. 
Per la suriettività è sufficiente prendere il codominio opportuno: 
l'insieme delle immagini di $f:[a,b] \rightarrow \mathbb{R}$. 
Allora $f$ è biiettiva se e solo se $f$ è invertibile. Definiamo 
$f^{-1}:[f(a),f(b)]\rightarrow [a,b]$ oppure $f^{-1}:[f(b),f(a)]\rightarrow [a,b]$, 
a seconda che $f$ sia monotona crescente oppure decrescente.
In particolare $f^{-1}$ è continua nel suo dominio, quindi $f^{-1}$ è derivabile nel 
suo dominio. Dunque l'integrale di $f$ nel suo dominio è uguale a $|(f(a)- f(b))\cdot(a-b)|$.
\\

\section{Teoremi con nome}

\subsection{Teorema della permanenza del segno}

Sia $\{a_n\}_{n \in \mathbb{N}} \in \mathbb{R}$, $\lim_{n \to \infty}a_n= l$. Se $l>0$ allora $\exists \overline n : \forall n> \overline n, \ a_n>0 $.

\subsection{Teorema dei due carabinieri}

Siano $\{a_n\}>\{b_n\}>\{c_n\}$ e sia $\lim_{n \to \infty}a_n=\lim_{n \to \infty}c_n=l$. Allora $\lim_{n \to \infty}b_n=l$.

\subsection{Teorema di Bolzano-Weierstrass}

Sia $\{a_n\}_{n \in \mathbb{N}} \in \mathbb{R}$ una successione limitata, allora ammette sottosuccessione convergente.
(Bisogna costruire una sottossuccessione monotona. Tutte le successioni monotone ammettono limite; la sottossucessione è limitata e allora è convergente).

\subsection{Definizione di successione di Cauchy}

Una successione è chiamata di Cauchy se i suoi termini sono arbitrariamente vicini tra loro purchè gli indici siano abbastanza grandi:

\begin{equation*}
	\forall \epsilon > 0 \ \exists \overline n: \forall n,m> \overline n \ |a_n-a_m|<\epsilon
\end{equation*}

\subsection{Teorema di completezza sequenziale di $ /mathbb{R}$}

Se una serie è di Cauchy in $\mathbb{R}$ allora converge in $\mathbb{R}$

\subsection{Definizione di punto di accuulazione}

Siano $A \subset \mathbb{R}$ e $x_0 \in A$, si dice che $x_0$ è un punto di accumulazione in A se $\forall W \in \mathcal{U}_{x_0}$ $A \backslash \{ x_0\} \cap W \neq \emptyset$.

\subsection{Teorema di Cauchy}

Dato $A \subset \mathbb{R}$, $x_0 \in D(A)\cap  \mathbb{\overline R}$ ed $f:A\rightarrow \mathbb{R}$

 $\lim_{x \to x_0}f(x)= \lambda \iff \forall\epsilon \exists W \in \mathcal{U_{x_0}} (\forall x,y \in W \Rightarrow |f(x)-f(y)|<\epsilon)$.

\subsection{Teorema di composizione}

Siano $f \in C(A)$ e $g\in C(f(A))$, allora $g(f(A)) \in C(A)$.

\subsection{Teorema di Weierstrass}

Sia $A \subset \mathbb{R}$ un insieme compatto. Se $f \in C(A)$ allora $f$ ha $max$ e $min$.

\subsection{Teorema di Bolzano}

Siano $a,b \in \mathbb{R}: a<b$ e sia $f:[a,b] \rightarrow \mathbb{R}$, $f \in C([a,b])$. $f(a) \cdot f(b)< 0 \Rightarrow \exists x_0 \in [a,b]: f(x_0)=0$.

\subsection{Teorema di Rolle}

Sia $f:[a,b]\rightarrow \mathbb{R}$ derivabile in $[a,b]$. $f(a)=f(b) \Rightarrow \exists x_0 \in [a,b]: f'(x_0)=0$.

\subsection{Teorema di Lagrange}

Sia $f:[a,b]\rightarrow \mathbb{R}$ derivabile in $[a,b]$. $\exists x_0 \in [a,b]: f'(x_0)=\frac{f(b)-f(a)}{b-a}$.

\subsection{Teorema di Cauchy}

Siano $f,g:[a,b]\rightarrow \mathbb{R}$ derivabili in $[a,b]$. $\frac{f(b)-f(a)}{g(b)-g(a)}=\frac{f'(x_0)}{f'(x_0)}$

\subsection{Formula di Taylor con resto di Peano}

Sia $f:I \rightarrow \mathbb{R}$, una funzione derivabile n volte nell'aperto $I$, allora
\begin{equation*}
	f(x)_n^{x_0}= \sum_{k = 0}^n \frac{f^k(x_0)}{k!}(x-x_0)^k+o((x-x_0)^n)
\end{equation*}

\subsection{Formula di Taylor con resto di Lagrange}

Sia $f:I \rightarrow \mathbb{R}$ derivabile $n+1$ volte nell'aperto $I$ e sia $x_0 \in I$, allora

\begin{equation*}
 f(x)^{x_0}_n = \sum_{k = 0}^n \frac{f^k(x_0)}{k!}(x-x_0)^k+\frac{f^{n+1}(y)}{(n+1)!}(x-x_0)^{n+1}
\end{equation*}

\subsection{Teorema di Riemann}

Sia $f:I \rightarrow \mathbb{R}$ continua e limitata nell'intervallo chiuso $I$, allora
\begin{equation*}
	f \in \mathcal{R}_I \iff \forall \epsilon >0 \exists \sigma \in \Omega : (S(f,\sigma)-s(f,\sigma)<\epsilon)
\end{equation*}

\subsection{Teorema della media integrale}

Sia $f:[a,b]\rightarrow$ integrabile nell'intervallo definito, allora $\inf_{[a,b]}f < \mu < \sup_{[a,b]}f$ tale che $\mu$ sia uguale all'integrale tra gli estremi dell'intervallo e diviso per la differenza degli estremi.

\subsection{I Teorema fondamentale del calcolo}

Sia $f \in \mathcal{R}_{[a,b]}$, posto $I_f:[a,b]\rightarrow \mathbb{R}$: integrale di $f$ nell'intervallo $[a,b]$. Allora
\begin{itemize}
	\item $I_f$ è continua in $[a,b]$;

	\item Se $f$ è continua in $x_0\in [a,b]$ allora $f(x_0)$ è la derivata di $I_f$ in $x_0$.
\end{itemize}

\subsection{II Teorema fondamentale del calcolo}

Sia $f \in \mathcal{R}_{[a,b]}$, posto $I_f:[a,b]\rightarrow \mathbb{R}$: integrale di $f$ nell'intervallo $[a,b]$. Allora l'integrale nell'intervallo $[a,b]$ di $f$ è uguale a $I_f(b)-I_f(a)$.

\subsection{Teorema del confronto}

Sia $f(x)=O(g(x))$ per $x\rightarrow$ allora se $g(x)\rightarrow l\in \mathbb{\overline R}$ per $x\rightarrow x_0$ $f(x)\rightarrow kl$ per $x\rightarrow x_0$ tale che $k=\frac{f(x)}{g(x)}$ per $x\rightarrow x_0$.

\subsection{Teorema della convergenza degli integrali oscillanti}

Dato l'integrale $I_{f\cdot g}:=$ integrale di $f\cdot g$, $f,g:[a,b[\rightarrow \mathbb{R}$ continue nel dominio e la derivata di g è continua nel dominio. $I_{f\cdot g}\rightarrow l\in \mathbb{R}$ per $x\rightarrow \infty$ se:
\begin{enumerate}
	\item l'integlrale di $f$ è limitato nel dominio;

	\item $g$ è monotona;

	\item $g\rightarrow 0$ per $x\rightarrow \infty$.
\end{enumerate}

Ovvero se le ipotesi sono vere l'integrale del prodotto tra $g$ ed $f$ è integrabile in senso generalizzato.

\end{document}
